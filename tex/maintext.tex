% ------------------------- %
%  Layout
% ------------------------- %

\documentclass[10pt]{article}
\linespread{0.99}
\setlength{\parskip}{0.3em}
\hfuzz=5002pt

% ------------------------- %
%  Packages
% ------------------------- %

\usepackage[utf8]{inputenc}
\usepackage[T2A]{fontenc}
\usepackage[russian, english]{babel}
\usepackage{fancyhdr}
\usepackage{listings}
\usepackage{booktabs}
\usepackage{xcolor}
\usepackage[margin=1.0in, headsep=0.3in]{geometry}
\usepackage{xspace}
\usepackage{graphicx}
\usepackage[font=small, labelfont=bf, width=0.95\linewidth]{caption}
\usepackage{floatrow}
\usepackage{amsmath}
\usepackage{fouriernc}
\usepackage{tempora}
\usepackage{mathtools}
\usepackage[unicode=true,hidelinks]{hyperref}
\usepackage[fixlanguage]{babelbib}
\usepackage[oldsyntax]{stackengine}
\usepackage[labelformat=simple, labelsep=colon]{subfig}
\usepackage{titling}

% ------------------------- %
%  Custom components
% ------------------------- %

% ------------------------- %

\newcommand{\shifthat}[2]{%
    \stackengine{\Sstackgap}{$#2$}{\(\hspace{#1}\hat{}\)}{O}{l}{F}{T}{S}
}

% ------------------------- %

\newcommand{\operator}[2][operator]{
    \if H#2\shifthat{0.5em}{#2}\else
    \if d#2\shifthat{0.49em}{#2}\else
    \if q#2\shifthat{0.35em}{#2}\else
    \if \mu#2\shifthat{0.35em}{#2}\else
    \shifthat{0.45em}{#2}
    \fi
    \fi
    \fi
    \fi
}

% ------------------------- %

\newcommand{\vectoperator}[2][operator]{
    \if d#2\shifthat{0.367em}{\textbf{#2}}\else
    \if m#2\shifthat{0.4em}{\textbf{#2}}\else
    \shifthat{0.275em}{\textbf{#2}}
    \fi
    \fi
}

% ------------------------- %

\newcommand{\vect}[3][vector]{
    \overrightarrow{#2_{#3}}
}

% ------------------------- %

\newcommand{\vectbf}[2][bold vector]{
    \vect{\textbf{#2}}
}

% ------------------------- %

\newcommand{\pd}[3][empty]{
    \frac{\partial {#2}}{\partial {#3}}
}

% ------------------------- %

\newcommand{\func}[5][empty]{
    {#2}_{#3}^{#4} \left({#5} \right)
}

% ------------------------- %

\newcommand{\underrel}[3][]{
    \mathrel{\mathop{#3}\limits_{
        \ifx c#1\relax\mathclap{#2}\else#2\fi
    }}
}
% ------------------------- %

\makeatletter

\newcommand{\Autoref}[1]{\@first@ref#1,@}
\def\@throw@dot#1.#2@{#1}% discard everything after the dot
\def\@set@refname#1{%    % set \@refname to autoefname+s using \getrefbykeydefault
    \edef\@tmp{\getrefbykeydefault{#1}{anchor}{}}%
    \xdef\@tmp{\expandafter\@throw@dot\@tmp.@}%
    \ltx@IfUndefined{\@tmp autorefnameplural}%
        {\def\@refname{\@nameuse{\@tmp autorefname}}}%
        {\def\@refname{\@nameuse{\@tmp autorefnameplural}}}%
}
\def\@first@ref#1,#2{%
\ifx#2@\autoref{#1}\let\@nextref\@gobble% only one ref, revert to normal \autoref
\else%
    \@set@refname{#1}%  set \@refname to autoref name
    \@refname\ref{#1}% add autoefname and first reference
    \let\@nextref\@next@ref% push processing to \@next@ref
\fi%
\@nextref#2%
}
\def\@next@ref#1,#2{%
\ifx#2@,~\ref{#1}\let\@nextref\@gobble% at end: print and+\ref and stop
\else, \ref{#1}% print  ,+\ref and continue
\fi%
\@nextref#2%
}

\makeatother

% ------------------------- %
% ------------------------- %

\newcommand{\img}[4][anything]{
    \begin{figure}[H]{
        \center{\includegraphics[width={#4}]{{#1}}}
        \caption{#2}\label{#3}}
    \end{figure}
}

% ------------------------- %

\newcommand{\floatimg}[4][anything]{
    \begin{figure}[ht]{
        \center{\includegraphics[width={#4}]{{#1}}}
        \caption{#2}\label{#3}}
    \end{figure}
}

% ------------------------- %

\newcommand{\subimg}[2][anything]{
    \begin{minipage}[h]{{#2}} % 0.4\textwidth
        \center{\includegraphics[width=1\linewidth]{{#1}}}
    \end{minipage}
}

% ------------------------- %

\newcommand{\subimgtwo}[4][anything]{
    \subfloat[{#2}]{\includegraphics[width={#4}]{{#1}}\label{#3}}
}

% ------------------------- %

% ------------------------- %
%  Document
% ------------------------- %

\begin{document}

	% declare specific options

	\selectlanguage{russian}

	\DeclareGraphicsExtensions{.pdf,.png,.jpg}

	\pagestyle{fancy}
	\fancyhf{}
	\fancyhead[L]{\textit{\nouppercase{\leftmark}}}
	\fancyfoot[C]{\thepage}

	\thinmuskip=1mu
	\thickmuskip=6mu
	\def\stacktype{S}\Sstackgap=-4.3pt
	\floatsetup[figure]{style=plain,subcapbesideposition=top}
	\captionsetup[subfigure]{margin=0.05\textwidth}

	% renew or new commands

	%\renewcommand{\equationautorefname}{}
	\newcommand{\subfigureautorefname}{\figureautorefname}
	\renewcommand{\thesubfigure}{\asbuk{subfigure}}

	% add components

	\title{Усиление угловой дисперсии лазерных гармоник высокого порядка при взаимодействии с плотными плазменными кластерами}
	\author{
		Л.А. Литвинов\textsuperscript{1}, А.А. Андреев\textsuperscript{1, 2}
	}
	\date{
		\normalsize{\textit{\textsuperscript{1}Санкт-Петербургский государственный университет, Санкт-Петербург \\ \textsuperscript{2}Физико-технический институт имени А.Ф.Иоффе, Санкт-Петербург}}
	}
	\maketitle

	\begin{abstract}
		Мы предлагаем мишень из массива наносфер в плазменной фазе в качестве эффективной дисперсионной среды для интенсивного экстремально ультрафиолетового излучения, возникающего в результате лазерно-плазменных взаимодействий, где происходят различные процессы генерации высоких гармоник. Процесс рассеяния исследуется с помощью численного моделирования с использованием условий резонанса, полученных из аналитической модели. Показано, что угловое распределение различных гармоник после рассеяния хорошо описывается простой интерференцией, в частности, для прямоугольной симметрии угловое распределение соответствует теории дифракции Брэгга-Вульфа.
	\end{abstract}

	%\section{Introduction}

Limited size targets interacting with high-intensity coherent radiation is well-studied phenomenon of linear excited surface plasmonic oscillations. Absorption and scattering of incident light in this case good described with Mie theory predicting exist of resonance corresponding to multipole oscillations of part of the target free electrons relative to positive charged ions. In resonance mode efficient exciting of surface plasmons can lead to significant boost internal and external field on fundamental cluster frequency (eigenfrequency). In turn, this can cause enhancement of field scattered on large angles relative to the direction of incident wave. 

In micrometer wavelengths photon crystals and lattices can be used for direction or diffraction electromagnetic waves~\cite{lin_zhang}, while for x-ray radiation it is possible to use real crystals with regularly placed scattering centers (atoms) with distance of few nanometers~\cite{batterman_cole}. At the same time, large interval between these wavelength orders named XUV (extreme-ultraviolet) is hard to manipulate.

Within the present work we consider the possibility of directed scattering of short wavelength radiation in the XUV range by scattering on suitable spherical clusters. Similar case with cylindrical symmetry (arrays of nanocylinders as scatterers) was researched earlier~\cite{andreev_lecz}. Of course, nanocylinders are more suitable regarding the control of size an distance parameters at the target manufacturing stage, but arrays of spherical clusters can make possible to manipulate with light direction in three-dimensional space and give a more optimal spatial configuration.

\img[components/img/gas_cluster4]{Clusters generation process by adiabatic expansion of gas.}{cluster_gas_sheme:image}{}{0.7\textwidth}

It is known that a short intense laser pulse can generate high-order harmonics by interacting with dense solid surfaces. But intensity of high-order harmonics generated in gases is at least 4 orders of magnitude less that is not enough to ionize the target and generate a plasma with fully imaginary refractive index that we need --- in our case, spherical clusters are ionized cluster gas (\autoref{cluster_gas_sheme:image}). To solve this problem we propose to use intense preceding pulse to pre-ionize the target and reach required plasma generation.

Common interaction scheme is shown in \autoref{plasma_area1:image}. Harmonics in the main pulse have different intensity depending on the angle, that leads to the angle dependence of output radiance shape. The scattering by a single cluster can be completely described in spherical symmetry and the interaction can be easily modeled with the help of particle in cell simulations. We propose to use linear approximation by Mie theory as assessment for further modeling. In general, we concentrate on a theoretical investigation, supported by simulations, and we point out the applicability for experimental realisation.

\img[components/img/plasma_area2]{Interaction scheme. The plane of polarization is parallel to one of the faces of cubic region. The dimensions of spherical clusters are about a few nanometers, distance between them is at least wavelength. In general, the distribution of clusters within a cubic region is random, clusters do not intersect the edges of the region and each other.}{plasma_area1:image}{}{0.8\textwidth}

	\section{Введение}
	%\section{Base model}

Let us consider a single cluster with radius $a$ irradiated by short femtosecond pulse with intensity about $I_{h} \approx 10^{14}$ $\textrm{W/cm}^2$. The Drude model yields the dielectric function of the plasma:

    \eq
		\varepsilon (\w) = 1 - \left( \frac{\w_{pe}}{\w} \right)^2 \frac{1}{1 + i \beta_{e}}, \qquad \w_{pe} = \sqrt{\frac{4 \pi e^2 n_e}{m_e}},
		\label{eps_plasma}
	\qe

\noindent where $\w$ --- harmonic (angular) frequency under consideration; $\w_{pe}$ --- the electron plasma frequency; $e$, $m_e$ --- electron charge and mass; $n_e = Z n_i$ --- the electron number density, where $Z$ --- average ionization degree, $n_i$ --- ion density. $\beta_{e} = v_e / \w$ and $v_e$ --- electron-ion collision rate in Spitzer approximation. As we are going to consider scattering of harmonic radiation, the cluster should have a density above the critical one for this harmonic: $n_c = \w^2 m_e / 4 \pi e^2$. Thus for example, for 10-th laser harmonic with wavelength $\lambda_{L} = 830$ nm one obtains condition $n_e > 1.3 \cdot 10^{23}$ $\textrm{cm}^{-3}$.

The Mie theory can be used for the description of elastic electromagnetic wave scattering by arbitrary sized particles in case of linear interactions and let obtain scattered and internal field. A main step is to solve the scalar Helmholtz Equation in suitable coordinate system andgain the vector solutions. For spherical cluster the solution of corresponding equation can be written in the form of Bessel and Hankel functions of $n$-th order~\cite{boren_huffman}.

Assume an incident plane wave propagating along $z$ axis of cartesian coordinate system and polarized along $x$ axis:

    \eq
        \vectbf{E}{i} = E_0\:e^{i\w t - ikz}\:\vectbf{e}{x},
        \label{E_i_sph}
    \qe

\noindent where $k = \w/c$ --- wavenumber, $\vectbf{e}{x}$ --- the unit vector of $x$ axis direction and polarization vector:

    \img[components/img/single_sph_scheme]{Base model scheme.}{single_sph_scheme:image}{0.73\textwidth}

Now we can expand the plane wave into series using generalized Fourier expantions. Assuming our media is isotropic we obtain following form of scattered field~\cite{boren_huffman}:

    \eq
		\vectbf{E}{s} = \sum_{n = 1}^{\infty}E_n \left[ i a_n\left(ka, m\right) \vectbf{N}{}^{(3)}_{e1n} - b_n\left(ka, m\right) \vectbf{M}{}^{(3)}_{o1n} \right], \qquad E_n = i^{n} E_0 \frac{2n + 1}{n \left( n + 1\right)}
        \label{E_s_sph}
	\qe

$n$ --- vector harmonic number after cartesian-spherical coordinate system transformation, $m = \sqrt{\varepsilon\left(\w\right)}$ --- refractive index of the target. Vector harmonics coefficients have the following form~\cite{boren_huffman}:


    \eq
		a_n(x,\:m) = \frac{m \func{\psi}{n}{\prime}{x} \func{\psi}{n}{}{mx} - \func{\psi}{n}{\prime}{mx} \func{\psi}{n}{}{x}}{m \func{\xi}{n}{\prime}{x} \func{\psi}{n}{}{mx} - \func{\psi}{n}{\prime}{mx} \func{\xi}{n}{}{x}},
		\label{an_bessel}
	\qe

    \eq
        b_n(x,\:m) = \frac{\func{\psi}{n}{\prime}{x} \func{\psi}{n}{}{mx} - m \func{\psi}{n}{\prime}{mx} \func{\psi}{n}{}{x}}{\func{\xi}{n}{\prime}{x} \func{\psi}{n}{}{mx} - m \func{\psi}{n}{\prime}{mx} \func{\xi}{n}{}{x}},
        \label{bn_bessel}
    \qe
    \eqc % artificial indent after the equation
    \cqe %

\noindent $\func{\psi}{n}{}{z} = z \func{j}{n}{}{z}$, $\func{\xi}{n}{}{z} = z \func{h}{n}{}{z}$ --- Riccati-Bessel functions, $h_n = j_n + i \gamma_n$ --- spherical Hankel functions of the first kind. 

In case of spherical symmetry amplitude of the scattered field is maximum for $m^2 = - (n+ 1) / n$ when $ka \ll 1$, that gain corresponding set of resonance densities in collision-less case: $n_e = n_c(2n + 1) / n$. It can be obtained using zero-order (asymptotic) approximation of Bessel functions, after which coefficients (\Autoref{an_bessel, bn_bessel}) are significant simplified:

    \eq
        a_n\left( x \to 0,\:m \right) = \left( 1 + 2i \frac{ (2n - 1)! (2n + 1)!}{4^n \: n! (n + 1)!} \frac{\left(m^2 + \frac{n + 1}{n} \right)}{(m^2 - 1)} \frac{1}{x^{2n+1}} \right)^{-1}, \qquad b_n\left( x \to 0,\:m \right) = 0
        \label{ab_asymp}
    \qe

This approximation can be used instead of (\Autoref{an_bessel, bn_bessel}) for scatterers with quite small radius, but for $ka \sim 1$ the approximation ceases to be reasonable already, particularly for large $n$. Instead, in this case, the first-order approximation is better suited:

    \eq
		a_n\left( x ,\:m \right) = \left( 1 + i \frac{ C_n x^{-1 -2n} \left( (4(1 + n + m^2 n) (-3 + 4n (1 + n)) - 2(m^2 - 1)(3 + n(5 + 2n + m^2 (2n - 1))) x^2) \right)}{\pi (m^2 - 1)(2n + 3)(n + 1)(4(2n + 3) - 2(m^2 + 1)x^2)} \right)^{-1}
		\label{an_sph_asymp1}
	\qe
	\eqc
		C_n = 2^{1 + 2n} \Gamma(n - \frac{1}{2}) \Gamma(n + \frac{5}{2})
	\cqe

\autoref{ab_asymp:image} shows dependence of the scattering coefficient on the electron density for two different values of the radius in zero-order approximation. We can compare it with the first-order for $ka = 1.5$ (\autoref{ab1:image}). We can see, that with increase of $n$ width of the resonance peak decreases rapidly. Larger $ka$ corresponds to larger peak width. Besides that, with increase of $ka$ value of the resonance density increases, that shown in \autoref{nenc_123:image}.

    \begin{figure}[H]
        \subimgtwo[components/img/sph_base/sph_ka0.5_123]{$ka = 0.5$.}{ab_asymp:a}{0.66\textwidth}\\
        \subimgtwo[components/img/sph_base/sph_ka1.5_123]{$ka = 1.5$.}{ab_asymp:b}{0.66\textwidth}
		\caption{Spherical harmonics coefficients in zero-order approximation, $\beta_e = 0$. "Exact" curves were plotted using full expansions of the Bessel and Hankel fucntions.}
		\label{ab_asymp:image}
	\end{figure}

    \img[components/img/sph_base/sph_ka1.5_123_1st]{$ka = 1.5$ in first-order approximation. $\beta_e = 0$. "Exact" curves were plotted using full expansions of the Bessel and Hankel fucntions.}{ab1:image}{0.66\textwidth}

Such approximations allow us to estimate the resonance cases for a material with pre-defined refractive index $m$ as well as estimate refractive index corresponding to the required wavelength. As we consider XUV range radiation (20-120 nm), radiuses of spherical scatterers should be about few nanometers, that causes $ka \sim 1$. Obviously, for such $ ka $ the resonance values of the electron density can be large in considering $n = 1$ as term with the largest contribution to the scattered field. Staying within high-temperature plasma we should use only $n_e < 10^{24}$ $\textrm{cm}^{-3}$. Thus for $ka > 0.9$ it is more reasonable to estimate the resonance electon density using $n = 2$.

Using first-order approximation (\ref{an_sph_asymp1}) with wavelength $\lambda_{10} = \lambda_{L} / 10 = 83$ nm we get $n_e \approx 5 \cdot 10^{23}$ $\textrm{cm}^{-3}$ for $ka \approx 0.5$ and $n_e \approx 5.7 \cdot 10^{23}$ $\textrm{cm}^{-3}$ for $ka \approx 0.7$ to reach efficient scattering.

    \img[components/img/sph_base/nenc_123]{Resonance electron density depending on radius. Curves were calculated in maximum points of (\ref{an_sph_asymp1}), $\beta_e = 0$.}{nenc_123:image}{0.66\textwidth}




	\section{Аналитическая модель}
	\section{Стационарные вычисления}
	%\section{Single cluster}

Within the Mie theory, it is well-known that we can significantly enhance the field amplitude near the target. To check this, $m$ values corresponding previously considered conditions $\lambda = \lambda_{10}$, $ka = 0.5,\:0.7$ was caltulated: $m_{0.5} = 1.635i$, $m_{0.7} = 1.851i$.

Total near- and far-field waa calculated in two cases: for $\lambda = \lambda_{L}$ and $\lambda = \lambda_{10}$ --- to compare their amplitudes and scattering profiles. We can see, that the scattering of the laser harmonic (first harmonic) is very close to Rayleigh scattering (\autoref{1h_ka0.5:image}b, \ref{1h_ka0.7:image}b) --- the incident plane wave profile almost does not change. Also scattering indicatrices in the plane of polarization correspond to the Rayleigh dependency~\cite{boren_huffman} (\autoref{ka0.5_far_field:image}b, \ref{ka0.7_far_field:image}b).

A completely different situation in the case of $\lambda = \lambda_{10}$ --- the incident plane wave profile is distorted as a result of scattering and becomes like a diverging spherical wave (\autoref{10h_ka0.5:image}b, \ref{10h_ka0.7:image}b). The near-field amplitude is higher than for $\lambda = \lambda_{L}$ about 5 times for both radius cases (\autoref{10h_ka0.5:image}a, \ref{10h_ka0.7:image}a). Also scattering indicatrices in the plane of polarization have larger amplitudes, which suggests more efficient far-field scattering (\autoref{ka0.5_far_field:image}a, \ref{ka0.7_far_field:image}a). We can see back-scattering enhancement on angles $\theta \approx 180^{\circ},\:120^{\circ},\:-240^{\circ}$ relative to the direction of the incident wave propagation.

The case $ka = 0.7$ compared with similar situation for scattering by a single nanocylinder~\cite{andreev_lecz} (\autoref{10h_ka0.7:image}c). We can see, that field distributions are similar include spherical outgoing far-field wave and localized near-field area in $0^{\circ}$ scattering direction relative to the direction of the incident wave propagation.

    \begin{figure}[H]
        (a)\:\subimg[components/img/mph/830nm_ka0.5_near]{0.42\textwidth}
        (b)\:\subimg[components/img/mph/830nm_ka0.5_far]{0.42\textwidth}
        \caption{Laser harmonic scattering by a single cluster. $\lambda = \lambda_{L}$, $a \approx 6.4$~nm ($ka = 0.5$); $|\vectbf{E}{}|$ plotted in the plane of polarization, near field (a) and far field (b).}
        \label{1h_ka0.5:image}
    \end{figure}

    \begin{figure}[H]
        (a)\:\subimg[components/img/mph/83nm_ka0.5_near_k_broken]{0.42\textwidth}
        (b)\:\subimg[components/img/mph/83nm_ka0.5_far_k_broken]{0.42\textwidth}
        \caption{$10$-th harmonic scattering by a single cluster. $\lambda = \lambda_{L}$, $a \approx 6.4$~nm ($ka = 0.5$); $|\vectbf{E}{}|$ plotted in the plane of polarization, near field (a) and far field (b).}
        \label{10h_ka0.5:image}
    \end{figure}

    \begin{figure}[H]
        (a)\:\subimg[components/img/mph/830nm_ka0.7_near]{0.42\textwidth}
        (b)\:\subimg[components/img/mph/830nm_ka0.7_far]{0.42\textwidth}
        \caption{Laser harmonic scattering by a single cluster. $\lambda = \lambda_{L}$, $a \approx 8.9$~nm ($ka = 0.5$); $|\vectbf{E}{}|$ plotted in the plane of polarization, near field (a) and far field (b).}
        \label{1h_ka0.7:image}
    \end{figure}

    \begin{figure}[H]
        (a)\:\subimg[components/img/mph/83nm_ka0.7_near_k_broken]{0.42\textwidth}
        (b)\:\subimg[components/img/mph/83nm_ka0.7_far_k_broken]{0.42\textwidth}
        \\(c)\:\subimg[components/img/external/oe-28_screen]{0.46\textwidth}
        \caption{$10$-th harmonic scattering by a single cluster. $\lambda = \lambda_{L}$, $a \approx 8.9$~nm ($ka = 0.5$); $|\vectbf{E}{}|$ plotted in the plane of polarization, near field (a) and far field (b). For qualitative assessment field scattered by a single nanocylinder \cite{andreev_lecz} with the same $ka$ added (c) --- here the incident wave propagates from right to left (along negative $x$ axis direction), polarization is along $y$ axis.}
        \label{10h_ka0.7:image}
    \end{figure}

    \img[components/img/mph/ka0.5_far_field]{Single cluster scattering indicatrices. $a \approx 6.4$~nm ($ka = 0.5$).}{ka0.5_far_field:image}{\textwidth}

    \img[components/img/mph/ka0.7_far_field]{Single cluster scattering indicatrices. $a \approx 8.9$~nm ($ka = 0.7$).}{ka0.7_far_field:image}{\textwidth}
	\subsection{Множество кластеров}

В рамках в рамках стационарной теории рассеяния Ми было рассмотрено множество кластеров в виде протяженной цилиндрической газовой струи (в дальнейшем мишень) с регулярной и квазирегулярной пространственной конфигурацией для исследования возможности рассеяния такими структурами на большие углы жёсткого ультрафиолетового излучения, в частности соответствующего гармоникам высокого порядка.

В качестве регулярного распределения была выбрана примитивная кубическая решетка c расстоянием между соседними узлами $d$. Квазирегулярное распределение было построено при помощи внесения случайных сдвигов координат узлов с произвольной нормой сдвига в диапазоне $0 \leq |\Delta d| \leq \eta d$, где $0 \leq \eta < 0.5$ --- степень нерегулярности. Тогда при кратном $d = b\lambda$, $b \in \EuScript{N}$ расстояние между соседними узлами после внесения сдвигов:

    \begin{equation}
        b\left(1 - \eta\right)\lambda \le d_{\textrm{irreg}} \le b\left(1 + \eta\right)\lambda
    \end{equation}

В квазирегулярном случае моделирование было проведено несколько раз с целью усреднения и получения обобщенной картины рассеянного поля. Для вычислений был использован программный код CELES~\cite{celes}.

\subsubsection{Условие дифракции для решетки в пространстве}

%! move to single cluster section
% Наиболее интенсивное излучения ожидается в направлении распространения падающего поля, так как в данном случае моды находятся в фазе и происходит конструктивная интерференция, как и в случае одиночного кластера~\cite{boren_huffman}. Конечно, если плотность такова, чтобы быть достаточно близко к резонансному значению для гармоник высокого порядка, то рассеяние на большие углы также возможно.

Условие дифракции в случае трехмерной регулярной решетки при упругом рассеянии принимает вид~\cite{Kittel86}:

    \begin{equation}
        \begin{cases}
        \begin{aligned}
            \left( \vectbf{D}{x},\: \vectbf{e}{\textrm{out}} - \vectbf{e}{\textrm{in}}\right) &= h \lambda
            \\
            \left( \vectbf{D}{y},\: \vectbf{e}{\textrm{out}} - \vectbf{e}{\textrm{in}}\right) &= k \lambda
            \\
            \left( \vectbf{D}{z},\: \vectbf{e}{\textrm{out}} - \vectbf{e}{\textrm{in}}\right) &= l \lambda
        \end{aligned}
        \end{cases}
        \label{bragg_wolf_order}
    \end{equation}

\noindentгде $h,\:k,\:l$ --- индексы Миллера представленные целыми числами, $\vectbf{D}{i}$ --- вектор, соединяющий соседние узлы решетки вдоль направления $i$, $\vectbf{e}{\textrm{in}}$ --- единичный вектор направления падающего излучения, $\vectbf{e}{\textrm{out}}$ --- единичный вектор направления прошедшего излучения. Переходя к сферическим координатам, связанными с $\vectbf{e}{\textrm{in}}$ так, что в декартовом представлении $\vectbf{e}{\textrm{in}} = \vectbf{e}{z}$, \autoref{bragg_wolf_order} можно преобразовать следующим образом, учитывая, что $|\vectbf{D}{x}| = |\vectbf{D}{y}| = |\vectbf{D}{z}| = d$ для рассматриваемой кубической решетки:

    \begin{equation}
        \begin{cases}
            \cos{\theta_0}\sin{\Delta \theta}\cos{\left( \Delta \varphi - \varphi_0 \right)} - \sin{\theta_0} \left( \cos{\Delta \theta} - 1 \right) = \cfrac{h^{\prime} \lambda}{d}
            \\
            \sin{\Delta \theta} \sin{\left( \Delta \varphi - \varphi_0 \right)} = \cfrac{k^{\prime} \lambda}{d}
            \\
            \sin{\theta_0}\sin{\Delta \theta}\cos{\Delta \varphi} + \cos{\theta_0} \left( \cos{\Delta \theta} - 1 \right)= \cfrac{l^{\prime} \lambda}{d}
        \end{cases}
        \label{bragg_wolf_order_spherical}
    \end{equation}

\noindentгде $\Delta \theta,\:\Delta \varphi$ --- углы, характеризующие отклонение направления дифрагировавшего излучения относительно падающего, $\theta_0,\:\varphi_0$ --- углы, характеризующие поворот мишени (решётки) в пространстве, $h^\prime,\:k^\prime\:,l^\prime$ --- новые индексы Миллера (\autoref{3ddiffr:image}). Используя \autoref{bragg_wolf_order_spherical}, можем получить угловое распределение дифрагировавшего излучения при заданных начальных параметрах $d$, $\lambda$, $\theta_0$, $\varphi_0$. 

    \begin{figure}[ht]
        \subimgtwo[components/img/3ddiffrxzgas]{Проекция на плоскость $xz$.}{3ddiffr:a}{0.4\textwidth}
        \hfil
        \subimgtwo[components/img/3ddiffrxygas]{Проекция на плоскость $xy$.}{3ddiffr:b}{0.4\textwidth}
        \caption{Общая схема взаимодействия падающего излучения с решеткой. $\theta_0$, $\varphi_0$ --- характеризуют углы покорота мишени в пространстве, $\Delta \theta$, $\Delta \varphi$ --- углы отклонения направления дифрагировавшего излучения относительно падающего, $r_{\textrm{gas}}$ --- радиус газовой струи, представляюшей мишень, $w$ --- диаметр гауссова пучка падающего излучения. $\Delta \theta$ отсчитывается вокруг $y$ против часовой стрелки, $\Delta \varphi$ --- вокруг $z$ против часовой стрелки.}\label{3ddiffr:image}
    \end{figure}


    \img[components/img/eint_scheme]{Схематическое изображение области $V$ (\autoref{V_for_e_int}).}{eint_scheme:image}{0.45\textwidth}

% Наиболее интенсивные направления дифракции будут соответствовать минимальным по модулю индексам Миллера, тогда пусть $k^\prime = 0$:

%     \begin{equation}
%         \begin{cases}
%             \Delta \varphi = \varphi_0
%             \\
%             \Delta \theta = \theta_0 + \arcsin{\left( \cfrac{h^{\prime} \lambda}{d} - \sin{\theta_0} \right)},
%             \\
%             l^{\prime} = \cfrac{\lambda}{d}\left(\sin{\theta_0}\sin{\Delta \theta}\cos{\Delta \varphi} + \cos{\theta_0} \left( \cos{\Delta \theta} - 1 \right)\right)
%         \end{cases}
%         \label{bragg_wolf_sol_0}
%     \end{equation}

% Используя \autoref{bragg_wolf_sol_0}, можно построить решения, соответствующие целым значениям $l^\prime$, которые отвечают различным порядкам прошедшего и отраженного излучения (\autoref{phi0_theta0_lprime:image}).

    \begin{figure}[ht]
        \subimgtwo[components/img/celes/dphi_dtheta_kprime_d_3l_phi0_0_theta0_15]{$\Delta \theta \in \left[ 0,\:\cfrac{\pi}{2} \right]$.}{1st_check_diffrth:a}{0.35\textwidth}
        \hfil
        \subimgtwo[components/img/celes/e_int_cylinder_15edge_theta0_15_phi0_0_249gap_rad50nm]{$\Delta \theta \in \left[ 0,\:\cfrac{\pi}{2} \right]$.}{1st_check_diffrth:b}{0.45\textwidth}
        \subimgtwo[components/img/celes/dphi_dtheta_kprime_d_3l_phi0_0_theta0_15_refr]{$\Delta \theta \in \left[ \cfrac{\pi}{2},\:\pi \right]$.}{1st_check_diffrth:c}{0.35\textwidth}
        \hfil
        \subimgtwo[components/img/celes/e_int_cylinder_15edge_theta0_15_phi0_0_249gap_rad50nm_refr]{$\Delta \theta \in \left[ \cfrac{\pi}{2},\:\pi \right]$.}{1st_check_diffrth:d}{0.45\textwidth}
        \caption{Вычисление $E_{\textrm{int}}$ по \autoref{e_int} (а, в) и решение \autoref{bragg_wolf_order_spherical} в целых индексах Миллера для $a = 50$ nm и $d = 3\lambda_{10}$ при $\varphi_0 = 0^{\circ}$, $\theta_0 = 15^{\circ}$, $\lambda = \lambda_{10} = 83$ nm, радиус цилиндрической области $\rho = 200$ nm (б, г). Для наглядности построение было поделено на две проекции полусферических областей по $\Delta \theta$ с полюсами в $0$ и $\pi$ соответственно.}\label{1st_check_diffrth:image}
    \end{figure}

    % \begin{figure}[ht]
    %     \subimgtwo[components/img/celes/phi0_theta0_lprime_d_2l_h_1]{$d = 2\lambda$, $h^\prime = 1$.}{phi0_theta0_lprime:a}{0.4\textwidth}
    %     \hfil
    %     \subimgtwo[components/img/celes/phi0_theta0_lprime_d_3l_h_1]{$d = 3\lambda$, $h^\prime = 1$.}{phi0_theta0_lprime:b}{0.4\textwidth}
    %     \caption{Кривые, отвечающие различным дифракционным порядкам по $l^\prime$ при $k^\prime = 0$, $\Delta \varphi = \varphi_0$.}\label{phi0_theta0_lprime:image}
    % \end{figure}

Для того, чтобы проверить достоверность полученной теории, смоделируем стационарное взаимодействие в случае регулярной решётки с радиусом кластеров $a = 50$ nm и $d = 3\lambda_{10}$ при $\varphi_0 = 0^{\circ}$, $\theta_0 = 15^{\circ}$, $\lambda = \lambda_{10} = 83$ нм, ширина гауссова пучка $w = 800$ nm, радиус цилиндра, ограничивающего решетку (радиус газовой струи) $r_{\textrm{gas}} = a + 12d \approx 2$ $\upmu\textrm{m}$, где множитель при $d$ --- количество узлов решетки между центральной осью и границей цилиндра. Несмотря на то, что в реальных условиях гауссов пучок $10$-ой гармоники Ti:Sa лазера с шириной 800 nm получить практически невозможно, в силу стационарности вычислений отношение $w\:/\:r_{\textrm{gas}}$ может быть корректно масштабировано при $w \ll 2r_{\textrm{gas}}$. Использованное малое значение $w$ в таком случае ускоряет вычисления, но принципиально не изменяет их результат.

Определим наиболее интенсивные направления рассеяния при помощи следующей интегральной характеристики: %\autoref{1st_check_diffrth:b}:

    \begin{equation}
        E_{\textrm{int}} \left(V, \:\eta\right) = \int\limits_{V} \left( |\vectbf{E}{s}|^2_{\eta \:= \eta} \right) dV,
        \label{e_int}
    \end{equation}

В данном случае \autoref{e_int} представляет собой интегрирование интенсивности рассеянного поля, вычисленного как квадрат модуля напряженности рассеянного поля (без учета нормирующего множителя) в области пространства $V$ для решётки, обладающей нерегулярностью $\eta$, то есть является энергией, рассеянной решёткой в область $V$. Эта область должна быть задана так, чтобы характеризовать некоторое направление в пространстве, и, как правило, для этой цели используется область в виде конуса, образованного некоторым раствором телесного угла $\delta \Omega$ и направлением при помощи углов $\Delta \theta$, $\Delta \varphi$. Для рассматриваемой задачи необходимо исключить из вычисления ближнее поле, ввиду чего накладывается дополнительное ограничение $V$ внутренностью сферического слоя пространства с границами $b_1$ и $b_2$, где $b_2$ --- граница области численного моделирования, $b_1$ --- больше радиуса сферы, описанной вокруг мишени. Хотя газовая струя является протяженным объектом, в моделировании используется только её сегмент, так как падающий пучок ограничен и рассеяное поле слабо зависит от частей струи, удаленных от области падения пучка, что позволяет описать вокруг такого сегмента соответствующую окружность.

Пересечение конической области и сферического слоя вдали от мишени можно приблизить цилиндром, считая $\rho \approx 0.5\,b_2\cdot\delta \Omega$, где $\rho$ --- радиус цилиндра. В таком случае, при помощи вспомогательного вектора $\vectbf{c}{}$ (\autoref{c_for_e_int}) получаем область $V$ (\autoref{eint_scheme:image}):

    \begin{align}
        \vectbf{c}{} = \vectbf{c}{}\left(x,\:y,\:z,\:\Delta \theta,\:\Delta \varphi \right) = \begin{pmatrix}c_{x}\\c_{y}\\c_{z}\end{pmatrix} = M_y(\Delta \theta)\,M_z(\Delta \varphi)\begin{pmatrix}x\\y\\z\end{pmatrix},
        \label{c_for_e_int}
    \end{align}
    \begin{align}
        V\left(\:\rho, \:b_1, \:b_2, \:\vectbf{c}{} \right) = \left\{\:x,\:y,\:z : c_{x}^2 + c_{y}^2 \leq \rho^2, \:\: b_1^2 \leq x^2 + y^2 + z^2 \leq b_2^2 \right\},
        \label{V_for_e_int}
    \end{align}

\noindentгде $M_y(\Delta \theta)$ --- матрица поворота вокруг декартовой оси $y$ на угол $\Delta\theta$ против часовой стрелки, $M_z(\Delta\varphi)$ --- матрица поворота вокруг декартовой оси $z$ на угол $\Delta\varphi$ против часовой стрелки. В дальнейшем взяты значения $\rho = w\:/\:4$, $b_1 = 4r_{\textrm{gas}}$ где $w$ --- ширина гауссова пучка падающего поля, $r_{\textrm{gas}}$ --- радиус газовой струи, формирующей мишень.

Также построим пересечения целочисленных решений для $h^\prime,\:k^\prime,\:l^\prime$ с заданными $\theta_0$, $\varphi_0$ в осях $\Delta \varphi$, $\Delta \theta$ при помощи \autoref{bragg_wolf_order_spherical} (\Autoref{1st_check_diffrth:a, 1st_check_diffrth:c}). Сравнивая полученные результаты можно заметить, что наиболее интенсивные направления дифракции по $E_{\textrm{int}}$ отвечают наиболее близкому раположению кривых, соответствующих целочисленным значениям индексов Миллера. 


    % \img[components/img/celes/20_rad_1st_check]{Рассеяние гауссового пучка ширины $w = 1700$ нм на слое регулярной решетке кластеров размера $a = 20$ нм, $\theta_0 = 20^{\circ}$. Границы газового слоя обозначены пурпурным цветом. Амплитуда $|\vectbf{E}{s}|^2$ построена в плоскости поляризации падающей волны, нормированая на собственный максимум.}{20rad_1st_check:image}{0.4\textwidth}

\begin{comment}
\subsubsection{Резонансное рассеяние лазерной гармоники}

%Для того, чтобы найти оптимальный угол рассеяния при помощи численного моделирования, введена следующая интегральная характеристика:

Варьируя $\theta_0$, был обнаружен оптимальный для резонансного рассения угол $\theta_0 = 14.32^{\circ}$, соответствующий наиболее интенсивному рассеянию в направлении дифракционного максимума $100$ по $h^\prime k^\prime l^\prime$ при $d = 2\lambda_{10}$, $w = 1700$ нм (\autoref{energy_vs_theta:image}).

    \img[components/img/celes/energy_vs_theta]{Зависимость относительной характеристики \autoref{integrate_sc_E} от угла падения $\theta_0$ при $\eta = 0$, $w = 1700$ нм.}{energy_vs_theta:image}{0.55\textwidth}

Рассеянные поля, полученные при моделировании, представлены на \autoref{random_ka0.7:image}. В этом случае мишень более реалистична, так как состоит из материала с реалистичной электронной плотностью $n_{el} = 5.7 \cdot 10^{23}\:\,\textrm{см}^{-3} \approx 4.4 n_{c}$ для $\lambda_{10} = 83$ нм. В качестве падающего поля был использован гауссов пучок с той же интенсивностью, что и в случае с одиночным кластером $I_{L} \approx 10^{18}\:\,\textrm{Вт/см}^2$, $I_h = I_{10} \approx 10^{14}\:\,\textrm{Вт/см}^2$, параметром ширины $w = 1700$ нм, направленный вдоль оси $z$ и поляризованный вдоль оси $x$.

На \Autoref{random_ka0.7:a, random_ka0.7:b} видна значительная разница между резонансным и нерезонансным случаем --- рассеяное поле $10$-ой гармоники четко ограничено, хорошо видно рассеяние в двух направлениях, соответствующих порядкам дифракции $000$ и $100$ по $h^\prime k^\prime l^\prime$ (\autoref{bragg_wolf_sol_0}), амплитуда поля превышает таковую в отсутствии резонанса более, чем в 10 раз при найденном угле $\Delta\theta$, что соответствует описанной ранее теории дифракции.

    \begin{figure}[ht]
        \subimgtwo[components/img/celes/mean_field_0_42]{Рассеяние $10$-ой гармоники, $\lambda_{10} = 83$ нм, $\eta = 0.43$.}{random_ka0.7:a}{0.4\textwidth}
        \hfil
        \subimgtwo[components/img/celes/mean_field_0_42_1harm]{Рассеяние лазерной гармоники, $\lambda_{L} = 830$ нм, $\eta = 0.43$.}{random_ka0.7:b}{0.4\textwidth}
        \subimgtwo[components/img/celes/reference_regular_14.324]{Рассеяние $10$-ой гармоники, $\lambda_{10} = 83$ нм, $\eta = 0$.}{random_ka0.7:c}{0.4\textwidth}
        % \hfil
        % \subimgtwo[components/img/celes/check20_rad]{Рассеяние $10$-ой гармоники, $\lambda_{10} = 83$ нм, $|\Delta d|_{\max} = 0$.}{random_ka0.7:c}{0.4\textwidth}
        \caption{Рассеяние гауссового пучка ширины $w = 1700$ нм на слое квазирегулярно расположенных кластеров размера $ka = 0.7$ ($a \approx 8.9$ нм), $\theta_0 = 14.32^{\circ}$. Границы газового слоя обозначены пурпурным цветом. Амплитуда $|\vectbf{E}{s}|^2$ построена в плоскости поляризации падающей волны, нормирована на максимальную амплитуду в случае рассеяния 10 гармоники.}\label{random_ka0.7:image}
    \end{figure}

    \img[components/img/celes/energy_vs_nonreg]{Зависимость относительной характеристики \autoref{integrate_sc_E} от нерегулярности $\eta$ при $w = 1700$ нм, $\theta_0 = 14.32^{\circ}$.}{energy_vs_nonreg:image}{0.55\textwidth}

% \subsubsection{Учет квазимонохроматичности падающего поля}

% Гармоническое излучение состоит из множества частот с хорошо определенными фазами, зависящими от природы излучающей среды. Для каждой гармоники условия рассеяния разные, так как нормированные величины определяют картину рассеянного поля. Была получена обобщенная картина рассеяного поля в случае волнового пакета, включающего в себя гармоники с 8 по 12.

\subsubsection{Направленная энергия в зависимости от нерегулярности расположения кластеров}

Для того, чтобы определить, как нерегулярность расположения кластеров в слое влияет на количество излучения, отклоненного от направления падения, было смоделировано рассеяние на множествах кластеров с различным показателем нерегулярности $\eta$ в соответствии с \autoref{random_shifts} и посчитана нормированная характеристика \autoref{e_int} на прямоугольной области с шириной $w$, соответствующей ширине падающего пучка, вне газового слоя в направлении дифракционного максимума $\Delta\theta = 2\theta_0$ (\autoref{energy_vs_nonreg:image}):

    \begin{equation}
        E_{\textrm{int}}^{\textrm{norm}} \left( 2\theta_0,\: 0, \:w, \:\eta\right) = \cfrac{E_{\textrm{int}} \left( 2\theta_0,\: 0, \:w, \:\eta\right)}{E_{\textrm{int}} \left( 2\theta_0,\: 0, \:w, \:0\right)}
        \label{integrate_sc_E}
    \end{equation}
\end{comment}
	\subsection{Оправдание стационарной модели}

В общем случае расчет взаимодействия высокоинтенсивного импульса лазерного излучения с группой плотных сферических кластеров, расположенных в трехмерном пространстве, требует длительных и сложных нестационарных вычислений ввиду того, что распределение электронной плотности кластеров в результате взаимодействия с лазерным импульсом изменяется с течением времени.

Для проверки масштабов изменения электронной плотности в рассматриваемом случае было проведено моделирование эволюции распределения электронной плотности в одномерном пространстве отдельного кластера. Для моделирования был взят код LPIC++~\cite{Pfund1998}.

В качестве источника был задан фронтальный линейно поляризованный лазерный импульс с длиной волны $\lambda_{10} = 83$ nm и длительностью $\tau$. Период лазерного излучения, соответствующий лазерной гармонике, равен $T = \lambda_{L} / c \approx 2.8$ fs, поэтому длина импульса в моделировании была взята $\tau = 10T = 28$ fs, время моделирования $t = 20T = 56$ fs. Плазма представлена 2000 частицами в каждой ячейке, занятой мишенью, расположенной в центре бокса шириной $w_{box} \approx 2\lambda_{10}$; электронная плотность мишени в критических единицах равна $n_{el} = 4.4 n_c$. Относительная амплитуда импульса $a_{0}$ равна:

    \begin{align}
        I_h \lambda_{10}^2 = a_0^2 \times 1.37 \cdot 10^{14}\:\rm{W}\cdot\rm{\upmu m}^2/\rm{cm}^2
    \end{align}
    \begin{equation*}
        a_0 = \frac{\lambda_{10}}{\sqrt{1.37} \cdot 1\:\rm{\upmu m}} \approx 7 \cdot 10^{-4}
    \end{equation*}

В качестве мишеней были взяты одиночные кластеры радиуса $a$ от 9 до 50 nm (\autoref{lpic_low_high:image}).

    \begin{figure}[htbp]
        \subimgtwo[components/img/lpic/9nm_rad_1nm_grid]{$a = 9$ nm.}{lpic_low_high:a}{0.42\textwidth}
        \hfil
        \subimgtwo[components/img/lpic/20nm_rad_1nm_grid]{$a = 20$ nm.}{lpic_low_high:b}{0.42\textwidth}
        \subimgtwo[components/img/lpic/htr_a]{Завимость средней суммарной толщины переходного слоя при $0 \leq t \leq 10T$ в зависимости от радиуса мишени.}{lpic_low_high:c}{0.55\textwidth}
        \caption{Взаимодействие одномерной мишени с $10$-ой гармоникой, $\lambda_{10} = 83$ nm.}\label{lpic_low_high:image}
    \end{figure}

По полученным результатам моделирования была расчитана средняя суммарная толщина переходного слоя в процессе взаимодействия со внешним импульсом $h_{tr}$ в зависимости от радиуса мишени $a$ (\autoref{lpic_low_high:c}). Условие квазистационарности в таком случае принимает вид $h_{tr} \ll 2a$, что соблюдается при $a \geq 20$ nm. Для ближайших по порядку гармоник величины $h_{tr}$ при аналогичных радиусах слабо отличаются.

	\newpage
\section*{Приложение}
\subsection*{Случайный сдвиг кластера в пространственной решетке}

Процесс вычисления сдвига для отдельного кластера описывается следующим образом:

    \begin{equation}
        P_0 = (x_0,\:y_0,\:z_0)
        \label{random_shifts}
    \end{equation}
    \begin{equation*}
        \Delta_{xyz} = (\Delta_x,\: \Delta_y\:, \Delta_z) = \textrm{rand.uniform}\left( -1,\:\,1 \right)|_{size=3}
    \end{equation*}
    \begin{equation*}
        \Delta_{xyz} = \textrm{rand.uniform}\left(0,\:\,|\Delta d|_{\max} \right)\:\frac{\Delta_{xyz}}{|\Delta_{xyz}|}
    \end{equation*}
    \begin{equation*}
        P_1 = P_0 + \Delta_{xyz}
    \end{equation*}

\subsection*{Резонансная электронная плотность в первом приближении}

В зависимости от нормированного радиуса сферического кластера $x = ka$ и порядка сферической гармоники $n$:

\begin{equation}
    m^2 \left(x,\:n \right) = \frac{8n^2 (n + 1) - (6n + 3)x^2 + 6n}{2n x^2 (2n-1)} \left[ 1 + \sqrt{ 1 - \frac{4n (n-3 + 4n^2 (n + 2)) (x^2 + 4n-2) x^2}{{\left(8n^2 (n + 1) - (6n + 3)x^2 + 6n \right)}^{2}} } \right]
    \label{m2_resonance}
\end{equation}

\begin{equation}
    \frac{n_e}{n_c} = (1 - m^2) (1 + i \beta_e)
    \label{nenc_resonance}
\end{equation}

	% add bibliography
	\newpage
	\selectbiblanguage{english}
	\bibliographystyle{ieeetr}
	\bibliography{components/bibliography.bib}

\end{document}