% ------------------------- %
%  Layout
% ------------------------- %

\documentclass[10pt]{article}
\linespread{0.99}
\setlength{\parskip}{0.3em}
\hfuzz=5002pt

% ------------------------- %
%  Packages
% ------------------------- %

\usepackage[utf8]{inputenc}
\usepackage[T2A]{fontenc}
\usepackage[russian, english]{babel}
\usepackage{fancyhdr}
\usepackage{listings}
\usepackage{booktabs}
\usepackage{xcolor}
\usepackage[margin=1.0in, headsep=0.3in]{geometry}
\usepackage{xspace}
\usepackage{graphicx}
\usepackage[font=small, labelfont=bf, width=0.95\linewidth]{caption}
\usepackage{floatrow}
\usepackage{amsmath}
\usepackage{fouriernc}
\usepackage{tempora}
\usepackage{mathtools}
\usepackage[unicode=true,hidelinks]{hyperref}
\usepackage[fixlanguage]{babelbib}
\usepackage[oldsyntax]{stackengine}
\usepackage[labelformat=simple, labelsep=colon]{subfig}
\usepackage{titling}

% ------------------------- %
%  Custom components
% ------------------------- %

\input{custom/equation.tex}
\input{custom/autoref_multiple.tex}
\input{custom/figure.tex}

% ------------------------- %
%  Document
% ------------------------- %

\begin{document}

	% declare specific options

	\selectlanguage{russian}

	\DeclareGraphicsExtensions{.pdf,.png,.jpg}

	\pagestyle{fancy}
	\fancyhf{}
	\fancyhead[L]{\textit{\nouppercase{\leftmark}}}
	\fancyfoot[C]{\thepage}

	\thinmuskip=1mu
	\thickmuskip=6mu
	\def\stacktype{S}\Sstackgap=-4.3pt
	\floatsetup[figure]{style=plain,subcapbesideposition=top}
	\captionsetup[subfigure]{margin=0.05\textwidth}

	% renew or new commands

	%\renewcommand{\equationautorefname}{}
	\newcommand{\subfigureautorefname}{\figureautorefname}
	\renewcommand{\thesubfigure}{\asbuk{subfigure}}

	% add components

	\title{Усиление угловой дисперсии лазерных гармоник высокого порядка при взаимодействии с плотными плазменными кластерами}
	\author{
		Л.А. Литвинов\textsuperscript{1}, А.А. Андреев\textsuperscript{1, 2}
	}
	\date{
		\normalsize{\textit{\textsuperscript{1}Санкт-Петербургский государственный университет, Санкт-Петербург \\ \textsuperscript{2}Физико-технический институт имени А.Ф.Иоффе, Санкт-Петербург}}
	}
	\maketitle

	\begin{abstract}
		Мы предлагаем мишень из массива наносфер в плазменной фазе в качестве эффективной дисперсионной среды для интенсивного экстремально ультрафиолетового излучения, возникающего в результате лазерно-плазменных взаимодействий, где происходят различные процессы генерации высоких гармоник. Процесс рассеяния исследуется с помощью численного моделирования с использованием условий резонанса, полученных из аналитической модели. Показано, что угловое распределение различных гармоник после рассеяния хорошо описывается простой интерференцией, в частности, для прямоугольной симметрии угловое распределение соответствует теории дифракции Брэгга-Вульфа.
	\end{abstract}

	%\noindent(- - 2 - -)\\~\\

\section{Introduction}

\noindent(- - 3 - -)\\~\\
Мишени ограниченного размера, взаимодействующие с мощным когерентным излучением, представляют собой хорошо изученное явление линейных возбужденных поверхностных плазмонных колебаний. Поглощение и рассеяние падающего света в таком случае хорошо описывается теорией Ми, предсказывающей существование резонанса, соответствующего мультипольным колебаниям части свободных электронов мишени относительно положительно заряженных ионов. В резонансном режиме эффективное возбуждение поверхностных плазмонов может привести к значительному увеличению внутреннего и внешнего поля на основной частоте кластера (собственной частоте). В свою очередь, это может вызвать усиление поля, рассеянного на больш\'{и}е углы относительно направления падающей волны.

\noindent(- - 4 - -)\\~\\
В микрометровых длинах волн фотонные кристаллы и решетки могут использоваться для направления или дифракции электромагнитных волн, в то время как для рентгеновского излучения можно использовать реальные кристаллы с регулярно расположенными центрами рассеяния (атомами) на расстоянии нескольких нанометров. В то же время большим интервалом между этими порядками длин волн, называемым XUV (жесткий или экстремальный ультрафиолет), достаточно трудно управлять.

\noindent(- - 5 - -)\\~\\
В рамках данной работы мы рассматриваем возможность направленного рассеяния коротковолнового излучения в XUV-диапазоне за счет рассеяния на сферических кластерах. Подобный случай с цилиндрической симметрией (массивы наноцилиндров в качестве рассеивателей) исследовался ранее. Конечно, наноцилиндры лучше подходят для управления параметрами размера и расстояния на стадии изготовления мишени, но массивы сферических кластеров могут позволить управлять направлением света в трехмерном пространстве и обеспечить более оптимальную пространственную конфигурацию за счет своей геометрии.

\noindent(- - 6 - -)\\~\\
Известно, что короткий интенсивный лазерный импульс может генерировать гармоники высокого порядка, взаимодействуя с плотными твердыми поверхностями. Но интенсивность высоких гармоник, генерируемых в газах, как минимум на 4 порядка меньше, что недостаточно для ионизации мишени и генерации плазмы с полностью мнимым показателем преломления, который нам нужен - в нашем случае сферические кластеры - это ионизированный кластерный газ (рисунок 1). Для решения этой проблемы мы предлагаем использовать интенсивный предимпульс для предварительной ионизации мишени и достижения необходимых условий генерации.

\noindent(- - 7 - -)\\~\\
Общая схема взаимодействия представлена на рисунке 2. Гармоники в основном импульсе имеют разную интенсивность в зависимости от угла, что приводит к угловой зависимости формы выходного излучения. Рассеяние на отдельном кластере можно полностью описать в сферической симметрии, а взаимодействие можно легко смоделировать с помощью particle-in-cell метода. Мы предлагаем использовать линейное приближение в рамках теории Ми в качестве оценки для дальнейшего моделирования. В частности, мы концентрируемся на теоретическом исследовании, подкрепленном компьютерным моделированием, и указываем на применимость для экспериментальной реализации.

	\section{Введение}
	%\section{Base model}

Let us consoider a single cluster with radius $a$ irradiated by short femtosecond pulse with intensity about $I_{h} \approx 10^{14}$ $\textrm{W/cm}^2$. The Drude model yields the dielectric function of the plasma:

    \eq
		\varepsilon (\omega) = 1 - \left( \frac{\omega_{pe}}{\omega} \right)^2 \frac{1}{1 + i \beta_{e}}, \qquad \omega_{pe} = \sqrt{\frac{4 \pi e^2 n_e}{m_e}},
		\label{eps_plasma}
	\qe

\noindent where $\omega$ --- harmonic (angular) frequency under consideration; $\omega_{pe}$ --- the electron plasma frequency; $e$, $m_e$ --- electron charge and mass; $n_e = Z n_i$ --- the electron number density, where $Z$ --- average ionization degree, $n_i$ --- ion density. $\beta_{e} = v_e / \omega$ and $v_e$ --- electron-ion collision rate in Spitzer approximation. As we are going to consider scattering of harmonic radiation, the cluster should have a density above the critical one for this harmonic: $n_c = \omega^2 m_e / 4 \pi e^2$. Thus for example, for 10-th laser harmonic with wavelength $\lambda_{L} = 830$ nm one obtains condition $n_e > 1.3 \cdot 10^{23}$ $\textrm{cm}^{-3}$.

The Mie theory can be used for the description of elastic electromagnetic wave scattering by arbitrary sized particles in case of linear interactions and let obtain scattered and internal field. A main step is to solve the scalar Helmholtz Equation in suitable coordinate system andgain the vector solutions. For spherical cluster the solution of corresponding equation can be written in the form of Bessel and Hankel functions of $n$-th order~\cite{boren_huffman}.

Assume an incident plane wave propagating along $z$ axis of cartesian coordinate system and polarized along $x$ axis:

    \eq
        \vectbf{E}{i} = E_0\:e^{i\omega t - ikz}\:\vectbf{e}{x},
        \label{E_i_sph}
    \qe

\noindent where $k = \omega/c$ --- wavenumber, $\vectbf{e}{x}$ --- the unit vector of $x$ axis direction and polarization vector:

    \img[components/img/single_sph_scheme]{Base model scheme.}{single_sph_scheme:image}{}{0.73\textwidth}

Now we can expand the plane wave into series using generalized Fourier expantions. Assuming our media is isotropic we obtain following form of scattered field~\cite{boren_huffman}:

    \eq
		\vectbf{E}{s} = \sum_{n = 1}^{\infty}E_n \left[ i a_n\left(ka, m\right) \vectbf{N}{}^{(3)}_{e1n} - b_n\left(ka, m\right) \vectbf{M}{}^{(3)}_{o1n} \right], \qquad E_n = i^{n} E_0 \frac{2n + 1}{n \left( n + 1\right)}
        \label{E_s_sph}
	\qe

$n$ --- vector harmonic number after cartesian-spherical coordinate system transformation, $m = \sqrt{\varepsilon\left(\omega\right)}$ --- refractive index of the target. Vector harmonics coefficients have the following form~\cite{boren_huffman}:


    \eq
		a_n(x,\:m) = \frac{m \func{\psi}{n}{\prime}{x} \func{\psi}{n}{}{mx} - \func{\psi}{n}{\prime}{mx} \func{\psi}{n}{}{x}}{m \func{\xi}{n}{\prime}{x} \func{\psi}{n}{}{mx} - \func{\psi}{n}{\prime}{mx} \func{\xi}{n}{}{x}},
		\label{an_bessel}
	\qe

    \eq
        b_n(x,\:m) = \frac{\func{\psi}{n}{\prime}{x} \func{\psi}{n}{}{mx} - m \func{\psi}{n}{\prime}{mx} \func{\psi}{n}{}{x}}{\func{\xi}{n}{\prime}{x} \func{\psi}{n}{}{mx} - m \func{\psi}{n}{\prime}{mx} \func{\xi}{n}{}{x}},
        \label{bn_bessel}
    \qe
    \eqc % artificial indent after the equation
    \cqe %

\noindent $\func{\psi}{n}{}{z} = z \func{j}{n}{}{z}$, $\func{\xi}{n}{}{z} = z \func{h}{n}{}{z}$ --- Riccati-Bessel functions, $h_n = j_n + i \gamma_n$ --- spherical Hankel functions of the first kind. 

In case of spherical symmetry amplitude of the scattered field is maximum for $m^2 = - (n+ 1) / n$ when $ka \ll 1$, that gain corresponding set of resonance densities in collision-less case: $n_e = n_c(2n + 1) / n$.

% В случае сферической симметрии амплитуда рассеянного поля максимальна для $m^2 = - (n+ 1) / n$ при $ka \ll 1$, что даёт соответствующий набор резонансных плотностей в бесстолкновительном случае $n_e = n_c(2n + 1) / n$. Это можно получить, используя нулевое асимптотическое приближение функций Бесселя, в результате чего коэффициенты (\Autoref{an_bessel, bn_bessel}) значительно упрощаются:

    \eq
        a_n\left( x \to 0,\:m \right) = \left( 1 + 2i \frac{ (2n - 1)! (2n + 1)!}{4^n \: n! (n + 1)!} \frac{\left(m^2 + \frac{n + 1}{n} \right)}{(m^2 - 1)} \frac{1}{x^{2n+1}} \right)^{-1}, \qquad b_n\left( x \to 0,\:m \right) = 0
        \label{ab_asymp}
    \qe

% Такое приближение можно использовать вместо (\Autoref{an_bessel, bn_bessel}) для объектов достаточно маленького радиуса, но уже при $ka \sim 1$ оно перестаёт быть разумным, особенно для больших $n$. Вместо него в таком случае лучше подходит аппроксимация первого порядка:

    \eq
		a_n\left( x ,\:m \right) = \left( 1 + i \frac{ C_n x^{-1 -2n} \left( (4(1 + n + m^2 n) (-3 + 4n (1 + n)) - 2(m^2 - 1)(3 + n(5 + 2n + m^2 (2n - 1))) x^2) \right)}{\pi (m^2 - 1)(2n + 3)(n + 1)(4(2n + 3) - 2(m^2 + 1)x^2)} \right)^{-1}
		\label{an_sph_asymp1}
	\qe
	\eqc
		C_n = 2^{1 + 2n} \Gamma(n - \frac{1}{2}) \Gamma(n + \frac{5}{2})
	\cqe

% На \autoref{ab_asymp:image} показана зависимость коэффициента рассеяния от электронной плотности для двух различных значений радиуса в рамках нулевого асимптотического приближения и сравнение первого и нулевого приближений. Видно, что с ростом $n$ ширина резонансного пика быстро уменьшается, а также б\'{о}льшим радиусам (безразмерным) $ka$ соответствует б\'{о}льшая их ширина. Помимо этого, с ростом радиуса растет и значение резонансной плотности, что видно на \autoref{nenc_123:image}.

%     \begin{figure}[ht]
% 		(a)\qquad \subimg[components/img/sph_base/sph_ka0.5_123]{0.66\textwidth}
% 		\\ (б)\qquad \subimg[components/img/sph_base/sph_ka1.5_123]{0.66\textwidth}
%         \\ (в)\qquad \subimg[components/img/sph_base/sph_ka1.5_123_1st]{0.66\textwidth}
% 		\caption{\textbf{Коэффициенты сферических гармоник.} $ka = 0.5$ (а), $ka = 1.5$ (б) в нулевом приближении; $ka = 1.5$ в первом приближении (в). $\beta_e = 0$. Кривые exact были построены с использованием полных разложений функций Бесселя и Ханкеля.}
% 		\label{ab_asymp:image}
% 	\end{figure}


% Подобные аппроксимации позволяют оценить резонансные случаи для материала с заданным коэффициентом преломления $m$, равно как и оценить $m$, отвечающий необходимой длине волны. Так как рассматривается XUV излучение, охватывающее длины волн порядка $20-120$ нм, радиусы сферических рассеивателей должны быть порядка нескольких нанометров, что обуславливает $ka \sim 1$. Очевидно, что для таких $ka$ резонансные значения электронной плотности могут быть велики в рассмотрении $n = 1$ как слагаемого, дающего наибольший вклад в результирующее поле (\autoref{nenc_123:image}). Не выходя за рамки высокотемпературной плазмы, мы можем использовать только $n_e < 10^{24}$ $\textrm{см}^{-3}$. Тогда для $ka > 0.9$ разумнее оценивать резонансную плотность, используя $n = 2$.

% Используя первое приближение (\ref{an_sph_asymp1}), для качественного рассеяния при $\lambda_{10} = \lambda_{L} / 10 = 83$ нм получаем $n_e \approx 5 \cdot 10^{23}$ $\textrm{см}^{-3}$ в случае $ka \approx 0.5$ и $n_e \approx 5.7 \cdot 10^{23}$ $\textrm{см}^{-3}$ в случае $ka \approx 0.7$.

%     \img[components/img/sph_base/nenc_123]{\textbf{Резонансная электронная плотность в зависимости от радиуса.} Кривые посчитаны в точках максимума коэффициента (\ref{an_sph_asymp1}), $\beta_e = 0$.}{nenc_123:image}{}{0.66\textwidth}




	\section{Аналитическая модель}
	\section{Стационарные вычисления}
	%\section{Одиночный кластер}

    \begin{figure}[H]
        \subimg[components/img/mph/830nm_6.37nm_sphere_n0_k1.73]{0.6\textwidth}
        \subimg[components/img/mph/830nm_6.37nm_sphere_n0_k1.73_log]{0.33\textwidth}
        \caption{\textbf{Рассеяние первой гармоники на одиночном кластере.} $\lambda_1 = \lambda_L = 830$ нм, $a \approx 6.4$ нм;~слева $|\vectbf{E}{}|$ в плоскости поляризации, ближнее поле рассеивающего объекта;~справа $\ln^2 ( |\vectbf{E}{}|^2 + 1)$, дальнее поле.}
        \label{1st_harm:image}
    \end{figure}

    \begin{figure}[H]
        \subimg[components/img/mph/83nm_6.37nm_sphere_n0_k1.73]{0.6\textwidth}
        \subimg[components/img/mph/83nm_6.37nm_sphere_n0_k1.73_log]{0.35\textwidth}
        \caption{\textbf{Рассеяние десятой гармоники на одиночном кластере.} $\lambda_{10} = 83$ нм, $a \approx 6.4$ нм;~слева $|\vectbf{E}{}|$ в плоскости поляризации, ближнее поле рассеивающего объекта;~справа $\ln ( |\vectbf{E}{}|^2 + 1)$, дальнее поле.}
        \label{10th_harm:image}
    \end{figure}

По полученным резульатам видно, что амплитуда ближнего электрического поля частицы в случае 10-ой гармоники (\autoref{10th_harm:image}) относительно внешнего поля значительно превышает таковую для 1-ой гармоники (\autoref{1st_harm:image}), что можно заметить по контрасту амплитуд вблизи и вдали от рассеивающего объекта.
	\section{Множество кластеров в рамках кубической области}

Возвращаясь к \autoref{plasma_area1:image}
	\subsection{Оправдание стационарной модели}

В общем случае расчет взаимодействия высокоинтенсивного импульса лазерного излучения с группой плотных сферических кластеров, расположенных в трехмерном пространстве, требует длительных и сложных нестационарных вычислений ввиду того, что распределение электронной плотности кластеров в результате взаимодействия с лазерным импульсом изменяется с течением времени.

Для проверки масштабов изменения электронной плотности в рассматриваемом случае было проведено моделирование эволюции распределения электронной плотности в одномерном пространстве отдельного кластера. Для моделирования был взят код LPIC++~\cite{Pfund1998}.

В качестве источника был задан фронтальный линейно поляризованный лазерный импульс с длиной волны $\lambda_{10} = 83$ nm, длительностью $\tau$ и интенсивностью $I_h = 10^{14}\:\rm{W}/\rm{cm}^2$. Период лазерного излучения, соответствующий лазерной гармонике, равен $T = \lambda_{L}\:/\:c \approx 2.8$ fs, поэтому длина импульса в моделировании была взята $\tau = 10T = 28$ fs, время моделирования $t = 20T = 56$ fs. Плазма представлена 2000 частицами в каждой ячейке, занятой мишенью, расположенной в центре бокса шириной $w_{box} \approx 3\lambda_{10}$; электронная плотность мишени в критических единицах равна $n_{el} = 4.4 n_c$. Относительная амплитуда импульса $\alpha_{h}$ равна:

    \begin{align}
        I_h \lambda_{10}^2 = \alpha_{h}^2 \times 1.37 \cdot 10^{14}\:\rm{W}\cdot\rm{\upmu m}^2/\rm{cm}^2
    \end{align}
    \begin{equation*}
        \alpha_{h} = \frac{\lambda_{10}}{\sqrt{1.37} \cdot 1\:\rm{\upmu m}} \approx 0.071
    \end{equation*}

Также было рассмотрено взаимодействие с лазерной гармоникой, для которой $\lambda_L = 830$ nm, $I_L = 10^{18}\:\rm{W}/\rm{cm}^2$, что дает относительную амплитуду импульса $\alpha_{L} \approx 71$. В качестве мишеней были взяты одиночные кластеры радиуса $a$ от 9 до 60 nm.

    \begin{figure}[htbp]
        \subimgtwo[components/img/lpic/9nm_rad_1nm_grid]{$a = 9$ nm, $\lambda = \lambda_{10}$.}{lpic_low_high:a}{0.42\textwidth}
        \hfil
        \subimgtwo[components/img/lpic/20nm_rad_1nm_grid]{$a = 20$ nm, $\lambda = \lambda_{10}$.}{lpic_low_high:b}{0.42\textwidth}
        \caption{Взаимодействие одномерной мишени с $10$-ой гармоникой, $\lambda_{10} = 83$ nm.}\label{lpic_low_high:image}
    \end{figure}

    \img[components/img/lpic/htr_over_2a_a]{Асимптотика средней суммарной толщины переходного слоя при $0 \leq t \leq 10T$ относительно радиуса мишени.}{lpic_htr:image}{0.55\textwidth}

По полученным результатам моделирования была расчитана средняя суммарная толщина переходного слоя в процессе взаимодействия со внешним импульсом $h_{tr}$ в зависимости от радиуса мишени $a$ (\autoref{lpic_htr:image}). Условие квазистационарности в таком случае принимает вид $h_{tr} \ll 2a$, что соблюдается при $a \geq 50$ nm. Для ближайших по порядку гармоник величины $h_{tr}$ при аналогичных радиусах слабо отличаются.

	\section*{Приложение}

\subsection*{A\quadНулевое и первое приближение коэффициентов рассеянного поля}

\begin{figure}[H]
    \subimgtwo[../img/sph_base/sph_ka0.5_123]{$ka = 0.5$.}{ab_asymp:a}{0.45\textwidth}
    \hfil
    \subimgtwo[../img/sph_base/sph_ka1.5_123]{$ka = 1.5$.}{ab_asymp:b}{0.45\textwidth}
    \subimgtwo[../img/sph_base/sph_ka1.5_123_1st]{$ka = 1.5$ в приближении первого порядка.}{ab_asymp:b}{0.45\textwidth}
    \caption{Коэффициенты сферических гармоник в приближении нулевого и первого порядка, $\beta_e = 0$. Кривые ``exact'' построены с использованием полных разложений функций Бесселя и Ханкеля в ряд.}\label{ab_asymp:image}
\end{figure}

На \autoref{ab_asymp:image} показана зависимость коэффициента рассеяния от электронной плотности для двух различных значений радиуса в рамках нулевого асимптотического приближения и сравнение первого и нулевого приближений. Видно, что с ростом $n$ ширина резонансного пика быстро уменьшается, а также б\'{о}льшим радиусам (безразмерным) $ka$ соответствует б\'{о}льшая их ширина. Помимо этого, с ростом радиуса растет и значение резонансной электронной плотности.
\subsection{Оправдание стационарной модели}

В общем случае расчет взаимодействия высокоинтенсивного импульса лазерного излучения с группой плотных сферических кластеров, расположенных в трехмерном пространстве, требует длительных и сложных нестационарных вычислений ввиду того, что распределение электронной плотности кластеров в результате взаимодействия с лазерным импульсом изменяется с течением времени.

Для проверки масштабов изменения электронной плотности в рассматриваемом случае было проведено моделирование эволюции распределения электронной плотности в одномерном пространстве отдельного кластера. Для моделирования был взят код LPIC++~\cite{Pfund1998}.

В качестве источника был задан фронтальный линейно поляризованный лазерный импульс с длиной волны $\lambda_{10} = 83$ nm, длительностью $\tau$ и интенсивностью $I_h = 10^{14}\:\rm{W}/\rm{cm}^2$. Период лазерного излучения, соответствующий лазерной гармонике, равен $T = \lambda_{L}\:/\:c \approx 2.8$ fs, поэтому длина импульса в моделировании была взята $\tau = 10T = 28$ fs, время моделирования $t = 20T = 56$ fs. Плазма представлена 2000 частицами в каждой ячейке, занятой мишенью, расположенной в центре бокса шириной $w_{box} \approx 3\lambda_{10}$; электронная плотность мишени в критических единицах равна $n_{el} = 4.4 n_c$. Относительная амплитуда импульса $\alpha_{h}$ равна:

    \begin{align}
        I_h \lambda_{10}^2 = \alpha_{h}^2 \times 1.37 \cdot 10^{14}\:\rm{W}\cdot\rm{\upmu m}^2/\rm{cm}^2
    \end{align}
    \begin{equation*}
        \alpha_{h} = \frac{\lambda_{10}}{\sqrt{1.37} \cdot 1\:\rm{\upmu m}} \approx 0.071
    \end{equation*}

Также было рассмотрено взаимодействие с лазерной гармоникой, для которой $\lambda_L = 830$ nm, $I_L = 10^{18}\:\rm{W}/\rm{cm}^2$, что дает относительную амплитуду импульса $\alpha_{L} \approx 71$. В качестве мишеней были взяты одиночные кластеры радиуса $a$ от 9 до 60 nm.

    \begin{figure}[htbp]
        \subimgtwo[components/img/lpic/9nm_rad_1nm_grid]{$a = 9$ nm, $\lambda = \lambda_{10}$.}{lpic_low_high:a}{0.42\textwidth}
        \hfil
        \subimgtwo[components/img/lpic/20nm_rad_1nm_grid]{$a = 20$ nm, $\lambda = \lambda_{10}$.}{lpic_low_high:b}{0.42\textwidth}
        \caption{Взаимодействие одномерной мишени с $10$-ой гармоникой, $\lambda_{10} = 83$ nm.}\label{lpic_low_high:image}
    \end{figure}

    \img[components/img/lpic/htr_over_2a_a]{Асимптотика средней суммарной толщины переходного слоя при $0 \leq t \leq 10T$ относительно радиуса мишени.}{lpic_htr:image}{0.55\textwidth}

По полученным результатам моделирования была расчитана средняя суммарная толщина переходного слоя в процессе взаимодействия со внешним импульсом $h_{tr}$ в зависимости от радиуса мишени $a$ (\autoref{lpic_htr:image}). Условие квазистационарности в таком случае принимает вид $h_{tr} \ll 2a$, что соблюдается при $a \geq 50$ nm. Для ближайших по порядку гармоник величины $h_{tr}$ при аналогичных радиусах слабо отличаются.


	% add bibliography
	\newpage
	\selectbiblanguage{english}
	\bibliographystyle{ieeetr}
	\bibliography{components/bibliography.bib}

\end{document}