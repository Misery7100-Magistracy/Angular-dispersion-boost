Within the framework of the stationary Mie scattering theory, many clusters were considered in the form of an extended cylindrical gas jet (hereinafter referred to as a target) with a regular and quasi-regular spatial configuration in order to study the possibility of scattering by such structures at large angles of hard ultraviolet radiation, in particular, corresponding to high-order harmonics.

A primitive cubic lattice with a distance between neighboring nodes $d$ was chosen as a regular distribution. The quasi-regular distribution was constructed by introducing random shifts of the coordinates of nodes with an arbitrary shift norm in the range $0 \leq |\Delta d| \leq \eta d$, where $0 \leq \eta < 0.5$ is the degree of irregularity. Then, for a multiple of $d = b\lambda$, $b \in \EuScript{N}$, the distance between adjacent nodes after making shifts:

    \begin{equation}
        b\left(1 - \eta\right)\lambda \le d_{\textrm{irreg}} \le b\left(1 + \eta\right)\lambda
    \end{equation}
    \begin{equation*}
    \end{equation*}

In the quasi-regular case, the simulation was carried out several times in order to average and obtain a generalized picture of the scattered field. The program code CELES~\cite{celes} was used for calculations.