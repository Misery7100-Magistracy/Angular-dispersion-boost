Periodic surface gratings or photonic crystals are excellent tools for light manipulation. However, this method is less effective in the case of extreme ultraviolet (XUV) light due to the high absorption of any material in this frequency range. In the paper we research the possibility of angluar boost of a radiation in the XUV range by scattering on suitable spherical clusters. Within the work the analytical model was developed with help of the Drude dielectric function of the plasma and the Mie scattering theory. The model was constructed in the quasi-static approximation since the ionization time is shorter than the pulse duration, which is much shorter than the plasma expansion time. Within the model we use the limiting forms of the Bessel functions since we are only interested in particle sizes smaller than the incident wavelength. The resonance parameters of the target was estimated using the tenth harmonic of titan:sapphire laser and the scattered field enhancement in the resonance case in comparison with the first harmonic was found. Using the same resonance conditions for a single cluster, we simulate diffraction by an array of such clusters using code CELES. Obtained results show a significant boost of the scattered field in the resonance case for large angles, which corresponds to the Bragg-Wolfe diffraction theory, - the ability to control high harmonics of laser radiation in XUV range using an ionized cluster gas.