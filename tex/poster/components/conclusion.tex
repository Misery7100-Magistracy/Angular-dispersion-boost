We found that a periodic structure of dense plasma clusters turned out to be a suitable element for efficient directional scattering of radiation in the XUV range. Since many spherical scatterers require calculations in three-dimensional space, a stationary model was proposed and the range of cluster radii was determined, within which the electron density is quasi-stationary during interaction with an external field. When the ionization is such that the electron concentration is close to resonance for given initial parameters, the scattering efficiency increases significantly and reaches several percent in the case of a single cluster. For many clusters, the efficiency of angular dispersion increases with the number of rows of clusters and can reach several percent in the case of certain directions.

The obtained angular distributions of diffraction maxima for scattering by a set of regularly spaced clusters are well described using the Laue theory, while introducing a small irregularity into the distribution of clusters weakens the most intense diffraction directions, which differ from the direction of the transmitted radiation, by no more than 25\%. In the case of non-monochromatic radiation, the efficiency of amplification of the angular dispersion decreases in accordance with the spectral distribution of the field amplitude, the attenuation can reach about three times.