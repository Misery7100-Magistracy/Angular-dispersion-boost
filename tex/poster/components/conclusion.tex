We found that a periodic structure of dense plasma clusters turned out to be a suitable element for efficient directional scattering of radiation in the XUV range. When the ionization is such that the electron density is near the resonance for given initial parameters, the scattering efficiency increases significantly and reaches several percent in the case of a single cluster. For many clusters, the efficiency of angular dispersion increases with the number of rows of clusters and can reach several percent in the case of certain directions.

The obtained angular distributions of diffraction maxima for scattering by a set of regularly spaced clusters are well described using the Laue theory, while introducing a small irregularity into the distribution of clusters causes the attenuation of the directed energy up to four times for the most intense diffraction orders. In the case of non-monochromatic radiation, the angular dispersion boost decreases in accordance with the spectral distribution of the field amplitude, the directed energy weakens up to three times.