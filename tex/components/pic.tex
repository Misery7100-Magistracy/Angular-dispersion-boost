\section{Particle-in-cell моделирование}

\subsection{Рассеяние волны одиночным кластером}

В моделировании был использован одномерный код LPIC++~\cite{Pfund1998} для исследования динамики электронов и эволюции распределения плотностей в одномерном плазменном слое.

В качестве падающего поля был использован TE-поляризованный лазерный импульс с длиной волны $\lambda_{10} = 83$ нм и длительностью $\tau$. Плазма представлена 200 частицами в каждой ячейке, занятой мишенью. При рассматриваемой длине волны импульса и электронной плотности в критических единицах $n_{el} = 4.4 n_c$ толщина скин-слоя:

    \begin{equation}
        h_{s} = \frac{c}{\omega_{pe}} = \lambda\sqrt{\frac{n_c}{n_{el}}} \approx40 \textrm{ нм} 
        \label{pic_skin_single}
    \end{equation}

В качестве мишени взят одиночный кластер с радиусом $a = 50$ нм. Используется равномерная сетка, в соответствии с толщиной скин-слоя $h_{s}$ мишень занимает 10 ячеек, имея общую толщину 100 нм; бокс моделирования размером 33 ячейки, соответствующий расстоянию примерно четырех длин волн. Период лазерного излучения, соответствующий лазерной гармонике, равен $T = \lambda c^{-1} \approx 2.8$ фс, поэтому длина импульса в моделировании была взята $\tau = 10T$, время моделирования $t = 20T$. 
% Также было проведено дополнительное моделирование с более мелкой сеткой (100 ячеек на мишень, \autoref{lpic_low_high:b}).

    \begin{figure}[htb]
        \subimgtwo[components/img/lpic/lpic_lowres_1]{Ширина ячейки сетки 10 нм, $a = 50$ нм.}{lpic_low_high:a}{0.4\textwidth}
        % \hfil
        % \subimgtwo[components/img/lpic/lpic_hires_1]{Ширина ячейки сетки 1 нм, $a = 50$ нм.}{lpic_low_high:b}{0.4\textwidth}
        % \subimgtwo[components/img/lpic/lpic_highres_small]{Ширина ячейки сетки 1 нм, $a = 9$ нм.}{lpic_low_high:c}{0.4\textwidth}
        \caption{Взаимодействие одномерной мишени с $10$-ой гармоникой, $\lambda_{10} = 83$ нм.}\label{lpic_low_high:image}
    \end{figure}

Видно, что в начале взаимодействия область равномерной плотности резко сужается, а сама мишень расплывается в стороны, после чего пространственное распределение электронной плотности остается практически постоянным (\autoref{lpic_low_high:a}). 
%Также с аналогичными параметрами излучения и более мелкой сеткой было смоделировано взаимодействие с ранее рассмотренным кластер $a \approx 9$ нм (\autoref{lpic_low_high:c}). За счет того, что характерный размер кластера меньше толщины скин-слоя (\autoref{pic_skin_single}), разлет и искажение картины плотности намного сильнее, также такой параметр $a$ соответствует более сильному резонансу рассеяния для выбранной электронной плотности мишени и длины волны импульса (\autoref{m2_resonance}).

\subsection{Рассеяние волны газовым слоем}

\subsection{Рассеяние волнового пакета газовым слоем}