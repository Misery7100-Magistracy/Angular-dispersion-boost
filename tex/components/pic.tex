\subsection{Оправдание стационарной модели}

В общем случае расчет взаимодействия высокоинтенсивного импульса лазерного излучения с группой плотных сферических кластеров, расположенных в трехмерном пространстве, требует длительных и сложных нестационарных вычислений ввиду того, что распределение электронной плотности кластеров в результате взаимодействия с лазерным импульсом изменяется с течением времени.

Для проверки масштабов изменения электронной плотности в рассматриваемом случае было проведено моделирование эволюции распределения электронной плотности в одномерном пространстве отдельного кластера. Для моделирования был взят код LPIC++~\cite{Pfund1998}.

В качестве источника был задан фронтальный линейно поляризованный лазерный импульс с длиной волны $\lambda_{10} = 83$ нм и длительностью $\tau$. Период лазерного излучения, соответствующий лазерной гармонике, равен $T = \lambda_{L} c^{-1} \approx 2.8$ фс, поэтому длина импульса в моделировании была взята $\tau = 10T = 28$ фс, время моделирования $t = 20T = 56$ фс. Плазма представлена 2000 частицами в каждой ячейке, занятой мишенью, расположенной в центре бокса шириной $w_{box} \approx 2\lambda_{10}$; электронная плотность мишени в критических единицах равна $n_{el} = 4.4 n_c$. Относительная амплитуда импульса $a_{0}$ равна:

    \begin{align}
        I_h \lambda_{10}^2 = a_0^2 \times 1.37 \cdot 10^{14} \textrm{Вт}\cdot\textrm{мкм}^2/\textrm{см}^2
    \end{align}
    \begin{equation*}
        a_0 = \frac{\lambda_{10}}{\sqrt{1.37} \cdot 1 \textrm{мкм}} \approx 7 \cdot 10^{-4}
    \end{equation*}


% При рассматриваемой длине волны импульса и электронной плотности в критических единицах $n_{el} = 4.4 n_c$ толщина скин-слоя:

%     \begin{equation}
%         h_{s} = \frac{c}{\omega_{pe}} = \lambda\sqrt{\frac{n_c}{n_{el}}} \approx40 \textrm{ нм} 
%         \label{pic_skin_single}
%     \end{equation}

В качестве мишеней были взяты одиночные кластеры радиуса $a$ от 9 до 50 нм (\autoref{lpic_low_high:image}).

% В качестве мишени взят одиночный кластер с радиусом $a = 50$ нм. Используется равномерная сетка, в соответствии с толщиной скин-слоя $h_{s}$ мишень занимает 10 ячеек, имея общую толщину 100 нм; бокс моделирования размером 33 ячейки, соответствующий расстоянию примерно четырех длин волн. Период лазерного излучения, соответствующий лазерной гармонике, равен $T = \lambda c^{-1} \approx 2.8$ фс, поэтому длина импульса в моделировании была взята $\tau = 10T$, время моделирования $t = 20T$. 
% Также было проведено дополнительное моделирование с более мелкой сеткой (100 ячеек на мишень, \autoref{lpic_low_high:b}).

    \begin{figure}[htbp]
        \subimgtwo[components/img/lpic/9nm_rad_1nm_grid]{$a = 9$ нм.}{lpic_low_high:a}{0.42\textwidth}
        \hfil
        \subimgtwo[components/img/lpic/20nm_rad_1nm_grid]{$a = 20$ нм.}{lpic_low_high:b}{0.42\textwidth}
        \subimgtwo[components/img/lpic/htr_a]{Завимость средней суммарной толщины переходного слоя при $0 \leq t \leq 10T$ в зависимости от радиуса мишени.}{lpic_low_high:c}{0.55\textwidth}
        % \subimgtwo[components/img/lpic/lpic_highres_small]{Ширина ячейки сетки 1 нм, $a = 9$ нм.}{lpic_low_high:c}{0.4\textwidth}
        \caption{Взаимодействие одномерной мишени с $10$-ой гармоникой, $\lambda_{10} = 83$ нм.}\label{lpic_low_high:image}
    \end{figure}

По полученным результатам моделирования была расчитана средняя суммарная толщина переходного слоя в процессе взаимодействия со внешним импульсом $h_{tr}$ в зависимости от радиуса мишени $a$ (\autoref{lpic_low_high:c}). Условие квазистационарности в таком случае принимает вид $h_{tr} \ll 2a$, что соблюдается при $a \geq 20$ нм. В случае волнового пакета ограничение аналогичное, так как для ближайших по порядку гармоник величины $h_{tr}$ при различных радиусах слабо отличаются.
