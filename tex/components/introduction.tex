\section{Введение}

Мишени конечного размера, взаимодействующие с высокоинтенсивным когерентным, излучением представляют собой хорошо изученное явление линейно возбужденных 
поверхностных плазмонных колебаний. Поглощение и рассеяние падающего света в таком случае с хорошей точностью могут быть описаны при помощи теории Ми, которая предсказывает существование резонанса, соответствующего мультипольным колебаниям части свободных электронов мишени относительно положительно заряженных ионов. В режиме резонанса эффективное возбуждение поверхностных плазмонов может привести к значительному усилению внутреннего и внешнего поля на собственной частоте кластера. Что может привести к усилению поля, рассеянного на большие углы относительно исходного направления падающей волны.

В пределах длин волн порядка микрометра могут быть использованы фотонные кристаллы и решетки для направления или дифракции электромагнитных волн \cite{lin_zhang}, в то время как для подобных манипуляций с рентгеновским излучением могут быть использованы кристаллы с атомами, регулярно расположенными на расстоянии нескольких нанометров, в качестве рассеивающих центров \cite{batterman_cole}. При этом большой промежуток между этими диапазонами длин волн, называющийся XUV (extreme-ultraviolet) или жесткий ультрафиолет, оказывается трудно манипулируемым.

В данной работе предлагается использование массивов сферических нанокластеров для направленного рассеяния жесткого ультрафиолетового излучения. Аналогичная задача в случае цилиндрической симметрии -- массивов наноцилиндров в качестве рассеивателей -- была исследована ранее \cite{andreev_lecz}. Полученные результаты показали эффективность подхода, что делает рассмотрение сферической конфигурации многообещающим. Конечно, использование цилиндров более удобно с точки зрения контроля радиусов одиночных рассеивателей и дистанций между ними, но массивы сферических кластеров могут позволить оперировать направлением излучениея в трехмерном пространстве, а также могут быть собраны в более оптимальную пространственную конфигурацию, нежели цилиндры.

Известно, что при помощи короткого интенсивного лазерного импульса можно генерировать лазерные гармоники высокого порядка при взаимодействии с плотными твердыми поверхностями \cite{teubner_gibbon_hoh}. В рассматриваемом диапазоне интенсивностей (порядка $10^{18}$ Вт/см$^2$) коэффициент преобразования в лучшем случае около $10^{-4}$, что дает интенсивности гармоник высокого порядка не выше $10^{14}$ Вт/см$^2$, чего недостаточно для ионизации мишени и генерации плазмы с чисто мнимым коэффициентом преломления. Для решения подобной проблемы предлагается использовать предимпульс.

Обобщенная схема взаимодействия приведена на \autoref{plasma_area1:image}. Гармоники, которые содержит основной импульс, обладают различной интенсивностью под различными углами, что приводит к угловой зависимости формы выходного излучения. Рассеяние одиночным сферическим кластером с хорошей точностью описывается в рамках теории Ми, поэтому такое линейное приближение предлагается использовать и в случае множества кластеров с целью качественной оценки и дальнейшего уточнения при помощи PIC моделирования (метод частиц-в-ячейках).

\img[components/img/plasma_area2]{Схема взаимодействия. Плоскость поляризации параллельна одной из граней кубической области. Размеры сферических кластеров порядка единиц нанометров, расстояния между ними не менее сотни нанометров. Распределение кластеров внутри кубической области в общем случае произвольно, кластеры не пересекают грани области.}{plasma_area1:image}{0.8\textwidth}
