\subsection{Множество кластеров}

В рамках в рамках стационарной теории рассеяния Ми было рассмотрено множество кластеров в виде протяженной цилиндрической газовой струи (в дальнейшем мишень) с регулярной и квазирегулярной пространственной конфигурацией для исследования возможности рассеяния такими структурами на большие углы жёсткого ультрафиолетового излучения, в частности соответствующего гармоникам высокого порядка.

В качестве регулярного распределения была выбрана примитивная кубическая решетка c расстоянием между соседними узлами $d$. Квазирегулярное распределение было построено при помощи внесения случайных сдвигов координат узлов с произвольной нормой сдвига в диапазоне $0 \leq |\Delta d| \leq \eta d$, где $0 \leq \eta < 0.5$ --- степень нерегулярности. Тогда при кратном $d = b\lambda$, $b \in \EuScript{N}$ расстояние между соседними узлами после внесения сдвигов:

    \begin{equation}
        b\left(1 - \eta\right)\lambda \le d_{\textrm{irreg}} \le b\left(1 + \eta\right)\lambda
    \end{equation}

В квазирегулярном случае моделирование было проведено несколько раз с целью усреднения и получения обобщенной картины рассеянного поля. Для вычислений был использован программный код CELES~\cite{celes}.

\subsubsection{Условие дифракции для решетки в пространстве}

%! move to single cluster section
% Наиболее интенсивное излучения ожидается в направлении распространения падающего поля, так как в данном случае моды находятся в фазе и происходит конструктивная интерференция, как и в случае одиночного кластера~\cite{boren_huffman}. Конечно, если плотность такова, чтобы быть достаточно близко к резонансному значению для гармоник высокого порядка, то рассеяние на большие углы также возможно.

Условие дифракции в случае трехмерной регулярной решетки при упругом рассеянии в системе координат, связанной с направлением падающего излучения, принимает вид~\cite{Kittel86}:

%     \begin{equation}
%         \begin{cases}
%         \begin{aligned}
%             \left( \vectbf{D}{x},\: \vectbf{e}{\textrm{out}} - \vectbf{e}{\textrm{in}}\right) &= h \lambda
%             \\
%             \left( \vectbf{D}{y},\: \vectbf{e}{\textrm{out}} - \vectbf{e}{\textrm{in}}\right) &= k \lambda
%             \\
%             \left( \vectbf{D}{z},\: \vectbf{e}{\textrm{out}} - \vectbf{e}{\textrm{in}}\right) &= l \lambda
%         \end{aligned}
%         \end{cases}
%         \label{bragg_wolf_order}
%     \end{equation}

% \noindentгде $h,\:k,\:l$ --- индексы Миллера представленные целыми числами, $\vectbf{D}{i}$ --- вектор, соединяющий соседние узлы решетки вдоль направления $i$, $\vectbf{e}{\textrm{in}}$ --- единичный вектор направления падающего излучения, $\vectbf{e}{\textrm{out}}$ --- единичный вектор направления прошедшего излучения. Переходя к сферическим координатам, связанными с $\vectbf{e}{\textrm{in}}$ так, что в декартовом представлении $\vectbf{e}{\textrm{in}} = \vectbf{e}{z}$, \autoref{bragg_wolf_order} можно преобразовать следующим образом, учитывая, что $|\vectbf{D}{x}| = |\vectbf{D}{y}| = |\vectbf{D}{z}| = d$ для рассматриваемой кубической решетки:

    \begin{equation}
        \begin{cases}
            \cos{\theta_0}\sin{\Delta \theta}\cos{\left( \Delta \varphi - \varphi_0 \right)} - \sin{\theta_0} \left( \cos{\Delta \theta} - 1 \right) = \cfrac{h^{\prime} \lambda}{d}
            \\
            \sin{\Delta \theta} \sin{\left( \Delta \varphi - \varphi_0 \right)} = \cfrac{k^{\prime} \lambda}{d}
            \\
            \sin{\theta_0}\sin{\Delta \theta}\cos{\left( \Delta \varphi - \varphi_0 \right)} + \cos{\theta_0} \left( \cos{\Delta \theta} - 1 \right)= \cfrac{l^{\prime} \lambda}{d}
        \end{cases}
        \label{bragg_wolf_order_spherical}
    \end{equation}

\noindentгде $\Delta \theta,\:\Delta \varphi$ --- углы, характеризующие отклонение направления дифрагировавшего излучения относительно падающего, $\theta_0,\:\varphi_0$ --- углы, характеризующие поворот мишени (решётки) в пространстве, $h^\prime,\:k^\prime,\:l^\prime$ --- индексы Миллера (\autoref{3ddiffr:image}). Используя \autoref{bragg_wolf_order_spherical}, можем получить угловое распределение дифрагировавшего излучения при заданных начальных параметрах $d$, $\lambda$, $\theta_0$, $\varphi_0$. 

    \begin{figure}[H]
        \subimgtwo[../img/3ddiffrxzgas]{Проекция на плоскость $xz$.}{3ddiffr:a}{0.4\textwidth}
        \hfil
        \subimgtwo[../img/3ddiffrxygas]{Проекция на плоскость $xy$.}{3ddiffr:b}{0.4\textwidth}
        \caption{Общая схема взаимодействия падающего излучения с решеткой. $\theta_0$, $\varphi_0$ --- характеризуют углы покорота мишени в пространстве, $\Delta \theta$, $\Delta \varphi$ --- углы отклонения направления дифрагировавшего излучения относительно падающего, $r_{\textrm{gas}}$ --- радиус газовой струи, представляюшей мишень, $w$ --- диаметр гауссова пучка падающего излучения. $\Delta \theta$ отсчитывается вокруг $y$ против часовой стрелки, $\Delta \varphi$ --- вокруг $z$ против часовой стрелки.}\label{3ddiffr:image}
    \end{figure}

% Наиболее интенсивные направления дифракции будут соответствовать минимальным по модулю индексам Миллера, тогда пусть $k^\prime = 0$:

%     \begin{equation}
%         \begin{cases}
%             \Delta \varphi = \varphi_0
%             \\
%             \Delta \theta = \theta_0 + \arcsin{\left( \cfrac{h^{\prime} \lambda}{d} - \sin{\theta_0} \right)},
%             \\
%             l^{\prime} = \cfrac{\lambda}{d}\left(\sin{\theta_0}\sin{\Delta \theta}\cos{\Delta \varphi} + \cos{\theta_0} \left( \cos{\Delta \theta} - 1 \right)\right)
%         \end{cases}
%         \label{bragg_wolf_sol_0}
%     \end{equation}

% Используя \autoref{bragg_wolf_sol_0}, можно построить решения, соответствующие целым значениям $l^\prime$, которые отвечают различным порядкам прошедшего и отраженного излучения (\autoref{phi0_theta0_lprime:image}).

    % \begin{figure}[ht]
    %     \subimgtwo[components/img/celes/phi0_theta0_lprime_d_2l_h_1]{$d = 2\lambda$, $h^\prime = 1$.}{phi0_theta0_lprime:a}{0.4\textwidth}
    %     \hfil
    %     \subimgtwo[components/img/celes/phi0_theta0_lprime_d_3l_h_1]{$d = 3\lambda$, $h^\prime = 1$.}{phi0_theta0_lprime:b}{0.4\textwidth}
    %     \caption{Кривые, отвечающие различным дифракционным порядкам по $l^\prime$ при $k^\prime = 0$, $\Delta \varphi = \varphi_0$.}\label{phi0_theta0_lprime:image}
    % \end{figure}
\subsubsection{Рассеяние монохроматического излучения}

Для того, чтобы проверить достоверность полученной теории, смоделируем стационарное взаимодействие в регулярном случае при различных параметрах решётки и ширине гауссова пучка $w = 800$ nm, радиусе цилиндра, ограничивающего решетку (радиус газовой струи) $r_{\textrm{gas}} = a + 12d \approx 2$ $\upmu\textrm{m}$, где множитель при $d$ --- количество узлов решетки между центральной осью и границей цилиндра. Несмотря на то, что в реальных условиях гауссов пучок $10$-ой гармоники Ti:Sa лазера с шириной 800 nm получить практически невозможно, в силу стационарности вычислений отношение $w\:/\:r_{\textrm{gas}}$ может быть корректно масштабировано при $w \ll 2r_{\textrm{gas}}$. Использованное малое значение $w$ в таком случае ускоряет вычисления, но принципиально не изменяет их результат.

Различие рассеяния в резонансном и нерезонансном случае показано на \autoref{random_ka0.7:image}: квадрат амплитуды рассеянного поля превышает таковой в отсутствии резонанса более, чем в 10 раз, а также в этом случае нет порядков дифракции, кроме нулевого, что напрямую следует из $\lambda\:/\:d = \lambda_L\:/\:2\lambda_{10} = 5$ в \autoref{bragg_wolf_order_spherical}.

Определим наиболее интенсивные направления рассеяния при помощи следующей интегральной характеристики: %\autoref{1st_check_diffrth:b}:

    \begin{equation}
        E_{\textrm{int}} \left( \eta,\:\lambda, \:V, \:E_0 \right) = \int\limits_{V}  |\vectbf{E}{s}\left(\eta,\:\lambda, \:E_0\right)|^2 dV.
        \label{e_int}
    \end{equation}

В данном случае \autoref{e_int} представляет собой интегрирование интенсивности рассеянного поля, вычисленного как квадрат модуля напряженности рассеянного поля (без учета нормирующего множителя) в области пространства $V$ для решётки, обладающей нерегулярностью $\eta$, то есть является энергией, рассеянной решёткой в область $V$, $\lambda$ представляет собой длину волны падающего поля, $E_0$ --- амплитуду. Область $V$ должна быть задана так, чтобы характеризовать некоторое направление в пространстве, и, как правило, для этой цели используется область в виде конуса, образованного некоторым раствором телесного угла $\delta \Omega$ и направлением при помощи углов $\Delta \theta$, $\Delta \varphi$. Для рассматриваемой задачи необходимо исключить из вычисления ближнее поле, ввиду чего накладывается дополнительное ограничение $V$ внутренностью сферического слоя пространства с границами $b_1$ и $b_2$, где $b_2$ --- граница области численного моделирования, $b_1$ --- больше радиуса сферы, описанной вокруг мишени. Хотя газовая струя является протяженным объектом, в моделировании используется только её сегмент, так как падающий пучок ограничен и рассеяное поле слабо зависит от частей струи, удаленных от области падения пучка, что позволяет описать вокруг такого сегмента соответствующую окружность.

    \img[components/img/eint_scheme]{Схематическое изображение области $V$ (\autoref{V_for_e_int}).}{eint_scheme:image}{0.45\textwidth}

Пересечение конической области и сферического слоя вдали от мишени можно приблизить цилиндром, считая $\rho \approx 0.5\,b_2\cdot\delta \Omega$, где $\rho$ --- радиус цилиндра. В таком случае, при помощи вспомогательного вектора $\vectbf{c}{}$ (\autoref{c_for_e_int}) получаем область $V$ (\autoref{eint_scheme:image}).

    \begin{align}
        \vectbf{c}{} = \vectbf{c}{}\left(x,\:y,\:z,\:\Delta \theta,\:\Delta \varphi \right) = \begin{pmatrix}c_{x}\\c_{y}\\c_{z}\end{pmatrix} = M_y(\Delta \theta)\,M_z(\Delta \varphi)\begin{pmatrix}x\\y\\z\end{pmatrix},
        \label{c_for_e_int}
    \end{align}
    \begin{align}
        V\left(\:\rho, \:b_1, \:b_2, \:\vectbf{c}{} \right) = \left\{\:x,\:y,\:z : c_{x}^2 + c_{y}^2 \leq \rho^2, \:\: b_1^2 \leq x^2 + y^2 + z^2 \leq b_2^2 \right\},
        \label{V_for_e_int}
    \end{align}

\noindentгде $M_y(\Delta \theta)$ --- матрица поворота вокруг декартовой оси $y$ на угол $\Delta\theta$ против часовой стрелки, $M_z(\Delta\varphi)$ --- матрица поворота вокруг декартовой оси $z$ на угол $\Delta\varphi$ против часовой стрелки. В дальнейшем взяты значения $\rho = w\:/\:4$, $b_1 = 4r_{\textrm{gas}}$ где $w$ --- ширина гауссова пучка падающего поля, $r_{\textrm{gas}}$ --- радиус газовой струи, формирующей мишень.

\begin{figure}[H]
    \subimgtwo[components/img/celes/Es_20nm_15deg_1harm.pdf]{$\lambda = \lambda_{L} = 830$ nm.}{random_ka0.7:a}{0.42\textwidth}
    \hfil
    \subimgtwo[components/img/celes/Es_20nm_15deg_10harm.pdf]{ $\lambda = \lambda_{10} = 83$ nm.}{random_ka0.7:b}{0.42\textwidth}
    \caption{$|\vectbf{E}{s}|^2$  в плоскости поляризации, сечение $\Delta \varphi = 0$ --- рассеяние гауссового пучка на слое квазирегулярно расположенных кластеров радиуса $a = 20$ nm, $\theta_0 = 15^{\circ}$, $\varphi_0 = 0^{\circ}$. Границы газового слоя обозначены пурпурным цветом. Амплитуда рассеянного поля нормирована на максимальное значение в случае 10 гармоники.}\label{random_ka0.7:image}
\end{figure}

\begin{figure}[H]
    \subimgtwo[components/img/celes/dphi_dtheta_kprime_d_2l_phi0_0_theta0_15.pdf]{Решение \autoref{bragg_wolf_order_spherical} в целых индексах Миллера.}{1st_check_diffrth:a}{0.335\textwidth}
    \hfil
    \subimgtwo[components/img/celes/e_int_20nm_15deg]{ $E_{\textrm{int}}$ по \autoref{e_int}.}{1st_check_diffrth:b}{0.42\textwidth}
    %\\
    %\subimgtwo[components/img/celes/dphi_dtheta_kprime_d_3l_phi0_0_theta0_15_refr]{$\Delta \theta \in \left[ \cfrac{\pi}{2},\:\pi \right]$.}{1st_check_diffrth:c}{0.27\textwidth}
    %\hfil
    %\subimgtwo[components/img/celes/e_int_cylinder_15edge_theta0_15_phi0_0_249gap_rad50nm_refr]{$\Delta \theta \in \left[ \cfrac{\pi}{2},\:\pi \right]$.}{1st_check_diffrth:d}{0.35\textwidth}
    \caption{Рассеяние 10-ой гармоники при параметрах решетки $a = 20$ nm и $d = 2\lambda_{10}$, $\varphi_0 = 0^{\circ}$, $\theta_0 = 15^{\circ}$, $\lambda = \lambda_{10} = 83$ nm, диапазон построения $\Delta \theta \in \left[ 0,\:\pi\,/\,2 \right]$.}\label{1st_check_diffrth:image}
\end{figure}

Также построим целочисленные решения для $h^\prime,\:k^\prime,\:l^\prime$ с заданными $\theta_0$, $\varphi_0$ в осях $\Delta \varphi$, $\Delta \theta$ при помощи \autoref{bragg_wolf_order_spherical} (\Autoref{1st_check_diffrth:a}). Графики на \autoref{1st_check_diffrth:image} представляют собой диаграммы в полярных координатах $(\Delta \theta, \: \Delta \varphi)$, то есть проекции поверхности сферы, ограниченной некоторым диапазоном угла $\Delta \theta$, на плоскость. Такой метод построения более удобный для изображения пространственного распределения рассеянного излучения, а также более естественный для отображения целочисленных решений на индексы Миллера (\autoref{bragg_wolf_order_spherical}), так как они в таком случае представляют собой наборы колец на сфере. По полученным результатам можно заметить, что наиболее интенсивные направления дифракции по $E_{\textrm{int}}$ отвечают наиболее близкому раположению кривых, соответствующих целочисленным значениям индексов Миллера.

Рассмотрим влияние нерегулярности решетки на наиболее интенсивные направления рассеянного поля, то есть зависимость \autoref{e_int} от нерегулярности решётки $\eta$. Для этого смоделировано рассеяние в случае квазирегулярной решётки при различных параметрах и нерегулярностью $\eta$ от 0 до 0.5 (\Autoref{nonreg_ka0.7:image}).

    % \begin{figure}[H]
    %     \subimgtwo[components/img/celes/e_int_rad_20nm_20deg_nonreg0.0]{$\eta = 0.0$.}{nonreg_mono:a}{0.42\textwidth}
    %     \hfil
    %     \subimgtwo[components/img/celes/e_int_rad_20nm_20deg_nonreg0.1]{$\eta = 0.1$.}{nonreg_mono:b}{0.42\textwidth}
    %     \caption{Характеристика \autoref{e_int} при различной нерегулярности решетки $\eta$, \\$a = 50$ nm, $d = 2\lambda_{10}$, $\varphi_0 = 0^{\circ}$, $\theta_0 = 20^{\circ}$, $\lambda = \lambda_{10} = 83$ nm, диапазон построения $\Delta \theta \in \left[ 0,\:\pi\,/\,2 \right]$.}\label{nonreg_mono:image}
    % \end{figure}

    \begin{figure}[H]

        \subimgtwo[components/img/celes/e_int_20nm_15deg_0.1nonreg.pdf]{$\eta = 0.1$.}{nonreg_ka0.7:b}{0.42\textwidth}
        \hfil
        \subimgtwo[components/img/celes/e_int_20nm_15deg_0.3nonreg]{$\eta = 0.3$.}{nonreg_ka0.7:c}{0.42\textwidth}
        \\
        \subimgtwo[components/img/celes/Es_20nm_15deg_10harm_0.1nonreg.pdf]{$|\vectbf{E}{s}|^2$  в плоскости поляризации, сечение $\Delta \varphi = 0$, $\eta = 0.1$.}{nonreg_ka0.7:a}{0.42\textwidth}
        \caption{Рассеяние гауссового пучка на слое квазирегулярно расположенных кластеров радиуса $a = 20$ nm, $\theta_0 = 15^{\circ}$, $\varphi_0 = 0^{\circ}$.}\label{nonreg_ka0.7:image}
    \end{figure}

    % \begin{figure}[ht]
    %     \subimgtwo[components/img/celes/mean_field_0_42]{Рассеяние $10$-ой гармоники, $\lambda_{10} = 83$ нм, $\eta = 0.43$.}{random_ka0.7:a}{0.4\textwidth}
    %     \hfil
    %     \subimgtwo[components/img/celes/mean_field_0_42_1harm]{Рассеяние лазерной гармоники, $\lambda_{L} = 830$ нм, $\eta = 0.43$.}{random_ka0.7:b}{0.4\textwidth}
    %     \subimgtwo[components/img/celes/reference_regular_14.324]{Рассеяние $10$-ой гармоники, $\lambda_{10} = 83$ нм, $\eta = 0$.}{random_ka0.7:c}{0.4\textwidth}
    %     % \hfil
    %     % \subimgtwo[components/img/celes/check20_rad]{Рассеяние $10$-ой гармоники, $\lambda_{10} = 83$ нм, $|\Delta d|_{\max} = 0$.}{random_ka0.7:c}{0.4\textwidth}
    %     \caption{Рассеяние гауссового пучка ширины $w = 1700$ нм на слое квазирегулярно расположенных кластеров размера $ka = 0.7$ ($a \approx 8.9$ нм), $\theta_0 = 14.32^{\circ}$. Границы газового слоя обозначены пурпурным цветом. Амплитуда $|\vectbf{E}{s}|^2$ построена в плоскости поляризации падающей волны, нормирована на максимальную амплитуду в случае рассеяния 10 гармоники.}\label{random_ka0.7:image}
    % \end{figure}

При помощи нормированного варианта $E_{\textrm{int}}$ (\autoref{e_int_norm}) была построена зависимость изменения рассеяния от нерегулярности решетки (\autoref{energy_vs_nonreg:image}). C ростом нерегулярности решётки наиболее интенсивные направления, кроме нулевого, значительно ослабляются, при этом общая картина рассеяния в различных направлениях выравнивается вплоть до практически однородной при $\eta \to 0.5$ (\autoref{nonreg_ka0.7:image}).

    \begin{equation}
        E_{\textrm{int}}^{\textrm{norm}} \left( \eta,\:\lambda, \:V, \:E_0 \right) = \frac{E_{\textrm{int}} \left( \eta,\:\lambda, \:V, \:E_0 \right)}{E_{\textrm{int}} \left( 0,\:\lambda, \:V, \:E_0 \right)}\label{e_int_norm}
    \end{equation}

    \img[components/img/celes/energy_vs_nonreg]{Ослабление рассеяния в зависимости от нерегулярности решётки.}{energy_vs_nonreg:image}{0.5\textwidth}

Также выявлено ослабление интенсивных направлений рассеяния в зависимости от радиуса кластеров в решётке (\Autoref{radius_wawa:image, energy_vs_radius:image}). При увеличении радиуса кластеров происходит распредление энергии между большим числом направлений (\Autoref{radius_wawa:image}) в соответствии с расположением пересечений кривых, соответствующих целочисленным индексам Миллера (\autoref{1st_check_diffrth:a}), то есть перераспределение в более высокие порядки дифракции по $h^\prime,\:k^\prime,\:l^\prime$. Это связано с уменьшением свободного пространства между узлами решётки при неизменном периоде $d$ --- количество прошедшего излучения уменьшается, расходимость прошедшего через решётку излучения увеличивается, ввиду чего дифракционная картина расплывается, а дифракционные максимумы ослабляются~\cite{born_wolf}.

    \begin{figure}[H]
        \subimgtwo[components/img/celes/e_int_30nm_15deg.pdf]{$a = 30$ nm.}{radius_wawa:a}{0.42\textwidth}
        \hfil
        \subimgtwo[components/img/celes/e_int_60nm_15deg.pdf]{$a = 60$ nm.}{radius_wawa:b}{0.42\textwidth}
        \caption{Характеристика \autoref{e_int} при радиусе кластеров $a = 30$ и $60$ nm, $d = 2\lambda_{10}$, $\varphi_0 = 0^{\circ}$, $\theta_0 = 15^{\circ}$, $\lambda = \lambda_{10} = 83$ nm, диапазон построения $\Delta \theta \in \left[ 0,\:\pi\,/\,2 \right]$.}\label{radius_wawa:image}
    \end{figure}

    \img[components/img/celes/energy_vs_radius]{Ослабление рассеяния в зависимости от радиуса кластеров, значения нормированы на $E_{\textrm{int}}$ при $a = 20$ nm.}{energy_vs_radius:image}{0.5\textwidth}

Была рассмотрена завивимость интенсивных направлений рассеяния по $\Delta \theta$ в зависимости от $\theta_0$ в сечении $\Delta \varphi = 0$. При этом было взято значение $\varphi_0 = 0$, так как любое ненулевое значение этого угла в сути поворачивает угловое распределение на тот же угол, что следует из \autoref{bragg_wolf_order_spherical}.

\begin{comment}
    

% components/img/celes/theta0_dtheta_dphi_phi0_0_miller
    % \img[components/img/celes/theta0_dtheta_dphi_phi0_0.pdf]{Характеристика \autoref{e_int} при различном угле $\theta_0$, $\varphi_0 = 0$, $\Delta \varphi = 0$, $a = 20$ nm, $d = 2\lambda_{10}$.}{dtheta_vs_theta0_eint:image}{0.54\textwidth}

    \begin{figure}[H]
        \subimgtwo[components/img/celes/theta0_dtheta_dphi_phi0_0_miller]{Решение \autoref{bragg_wolf_order_spherical} в целых индексах Миллера.}{theta0_dphi:a}{0.397\textwidth}
        \hfil
        \subimgtwo[components/img/celes/theta0_dtheta_dphi_phi0_0]{$E_{\textrm{int}}$ по \autoref{e_int}.}{theta0_dphi:b}{0.45\textwidth}
        \caption{Рассеяние 10-ой гармоники при различном угле $\theta_0$, $\varphi_0 = 0$, $\Delta \varphi = 0$, $a = 20$ nm, $d = 2\lambda_{10}$. На (а) $k^\prime = 0$ для любых $\theta_0$ и $\Delta \theta$.}\label{theta0_dphi:image}
    \end{figure}

    %\img[components/img/celes/20_rad_1st_check]{Рассеяние гауссового пучка ширины $w = 1700$ нм на слое регулярной решетке кластеров размера $a = 20$ нм, $\theta_0 = 20^{\circ}$. Границы газового слоя обозначены пурпурным цветом. Амплитуда $|\vectbf{E}{s}|^2$ построена в плоскости поляризации падающей волны, нормированая на собственный максимум.}{20rad_1st_check:image}{0.4\textwidth}
\end{comment}
\subsubsection{Рассеяние волнового пакета}

Рассмотрим рассеяние волнового пакета, амплитуда которого описывается гауссовой функцией во времени (\autoref{gaussian_E0}). Рассматривая периодическое продолжение этого импульса на промежутке $[-\tau, \tau]$, где $\tau$ --- ширина импульса, можем построить ряд Фурье (\autoref{gaussian_E0_fourier}). Таким образом, имеем коэффициенты Фурье (\autoref{gaussian_E0_aj}), представляющие собой вклад каждой из гармоник в общий импульс.
%Аппроксимируем такой волновой пакет 10-ой гармоникой, рассмотренной ранее и четырьмя ближайшими к ней, то есть с 8-ой по 12-ую (\autoref{gauss_wavepacket_lambda:image}). 

Для того, чтобы построить диаграмму рассеяния волнового пакета, была использована новая интегральная характеристика, определенная с учетом коэффициентов разложения в ряд Фурье волнового пакета (\autoref{wavepacket_eint}). Такая характеристика разумна для описания направлений рассеяния в силу аддитивности энергии как количественной характеристики. Область $V$ в данном случае представляет собой аналогичную той, что была использована для предыдущей интегральной характеристики (\autoref{V_for_e_int}, \autoref{eint_scheme:image}).

    \begin{equation}
        E_0\left( t \right) = \exp{\left( - \frac{t^2}{\tau^2}\right)}
        \label{gaussian_E0}
    \end{equation}

    \begin{equation}
        E_0\left( t \right) = \frac{\sqrt{\pi}}{2} + \sum_{j = 1}^{\infty}{ a_j \, \cos{\left(\omega_j t \right)}}, \quad \omega_j = \frac{2 \pi j}{\tau} = \frac{c}{\lambda_j}, \quad \lambda_{j} = \frac{\lambda_{L}}{j},
        \label{gaussian_E0_fourier}
    \end{equation}

    \begin{equation}
        a_j = \frac{1}{\tau} \int\limits_{-\tau}^{\tau}  \exp{\left( - \frac{t^2}{\tau^2}\right)} \cos{\left(\omega_j t \right)} dt,
        \label{gaussian_E0_aj}
    \end{equation}

    % \begin{equation}
    %     \vectbf{E}{i} = \sum_{n = 1}^{\infty} E_{0, \: n} \:e^{i n \omega_0 t-ikz}\:\vectbf{e}{x}\label{gaussian_E}
    % \end{equation}

    % \begin{equation}
    %     E_0 \left( \lambda \right) = \frac{1}{\sigma\sqrt{2\pi}}\,e^{-\frac{1}{2}{\left(\frac{\lambda - \mu}{\sigma}\right)}^2}\label{gaussian_E0}
    % \end{equation}

    \begin{equation}
        \EuScript{E}_{\textrm{int}} \left(V, \:\eta\right) = \sum\limits_{j\:=\:N_1\:>\:0}^{N_2}{E_{\textrm{int}} \left( \eta,\:\lambda_{j}, \:V, \:a_j) \right)}
        \label{wavepacket_eint}
    \end{equation}

Определим наиболее интенсивные направления рассеянного поля для решётки с $d = 2\lambda_{10}$, радиусом кластеров $a = 20$~nm, $\theta_0 = 15^\circ$, $\varphi_0 = 0^\circ$, гармоники в волновом пакете с 8-ой по 12-ую, то есть $N_1 = 8$, $N_2 = 12$ в \autoref{wavepacket_eint}, гауссова импульс имеет длину $\tau \approx 17$ fs. 

Сравнивая полученный результат с аналогичной диаграммой, вычисленной при помощи \autoref{e_int} только для 10-ой гармоники, можно заметить значительное усиление нулевого дифракционного максимума ($h^\prime = k^\prime = l^\prime = 0$), отвечающего прошедшему излучению, и небольшое усиление остальных, что полностью соответствует \autoref{bragg_wolf_order_spherical} в силу того, что индексы Миллера, отвечающие дифракционным уравнениям для разных длин волн будут связаны между собой коэффициентами пропорциональности. Таким образом имеем масштабирование кривых, отвечающих целочисленным значениям индексов Миллера.

На \Autoref{wavepacket1:a, wavepacket1:b} видно, что немонохроматичность излучения вносит погрешность для отклонения в заданном направлении --- наиболее интенсивные направления дифракции расплываются (за исключением нулевого).

    % \img[components/img/celes/wavepacket8..12_rad]{Амплитуда элементов гауссова волнового пакета напряженности поля в зависимости от длины волны. Синяя кривая соответствует \autoref{gaussian_E0} с $\mu = 83$~nm, $\sigma = 11$~nm, черные точки --- дискретная аппроксимация волнового пакета, гармоники с 12-ой по 8-ую слева направо.}{gauss_wavepacket_lambda:image}{0.45\textwidth}

    \begin{figure}[ht]
        \subimgtwo[components/img/celes/E_squared/eint_10harm_15deg_0.0nonreg.pdf]{Рассеяние 10-ой гармоники.}{wavepacket1:a}{0.45\textwidth}
        \hfil
        \subimgtwo[components/img/celes/E_squared/eint_wavepacket_15deg_0.0nonreg.pdf]{Рассеяние волнового пакета из гармоник с 8-ой по 12-ую.}{wavepacket1:b}{0.45\textwidth}
        \caption{Полярная диаграмма рассеяния гауссового волнового пакета в сравнении с рассеянием 10-ой гармоники. $\theta_0 = 15^\circ$, $\varphi_0 = 0^\circ$, $d = 2\lambda_{10}$, радиус кластеров $a = 20$ nm.}\label{wavepacket1:image}
    \end{figure}

\begin{comment}
\subsubsection{Резонансное рассеяние лазерной гармоники}

%Для того, чтобы найти оптимальный угол рассеяния при помощи численного моделирования, введена следующая интегральная характеристика:

Варьируя $\theta_0$, был обнаружен оптимальный для резонансного рассения угол $\theta_0 = 14.32^{\circ}$, соответствующий наиболее интенсивному рассеянию в направлении дифракционного максимума $100$ по $h^\prime k^\prime l^\prime$ при $d = 2\lambda_{10}$, $w = 1700$ нм (\autoref{energy_vs_theta:image}).

    \img[components/img/celes/energy_vs_theta]{Зависимость относительной характеристики \autoref{integrate_sc_E} от угла падения $\theta_0$ при $\eta = 0$, $w = 1700$ нм.}{energy_vs_theta:image}{0.55\textwidth}

Рассеянные поля, полученные при моделировании, представлены на \autoref{random_ka0.7:image}. В этом случае мишень более реалистична, так как состоит из материала с реалистичной электронной плотностью $n_{el} = 5.7 \cdot 10^{23}\:\,\textrm{см}^{-3} \approx 4.4 n_{c}$ для $\lambda_{10} = 83$ нм. В качестве падающего поля был использован гауссов пучок с той же интенсивностью, что и в случае с одиночным кластером $I_{L} \approx 10^{18}\:\,\textrm{Вт/см}^2$, $I_h = I_{10} \approx 10^{14}\:\,\textrm{Вт/см}^2$, параметром ширины $w = 1700$ нм, направленный вдоль оси $z$ и поляризованный вдоль оси $x$.

На \Autoref{random_ka0.7:a, random_ka0.7:b} видна значительная разница между резонансным и нерезонансным случаем --- рассеяное поле $10$-ой гармоники четко ограничено, хорошо видно рассеяние в двух направлениях, соответствующих порядкам дифракции $000$ и $100$ по $h^\prime k^\prime l^\prime$ (\autoref{bragg_wolf_sol_0}), амплитуда поля превышает таковую в отсутствии резонанса более, чем в 10 раз при найденном угле $\Delta\theta$, что соответствует описанной ранее теории дифракции.

    \begin{figure}[ht]
        \subimgtwo[components/img/celes/mean_field_0_42]{Рассеяние $10$-ой гармоники, $\lambda_{10} = 83$ нм, $\eta = 0.43$.}{random_ka0.7:a}{0.4\textwidth}
        \hfil
        \subimgtwo[components/img/celes/mean_field_0_42_1harm]{Рассеяние лазерной гармоники, $\lambda_{L} = 830$ нм, $\eta = 0.43$.}{random_ka0.7:b}{0.4\textwidth}
        \subimgtwo[components/img/celes/reference_regular_14.324]{Рассеяние $10$-ой гармоники, $\lambda_{10} = 83$ нм, $\eta = 0$.}{random_ka0.7:c}{0.4\textwidth}
        % \hfil
        % \subimgtwo[components/img/celes/check20_rad]{Рассеяние $10$-ой гармоники, $\lambda_{10} = 83$ нм, $|\Delta d|_{\max} = 0$.}{random_ka0.7:c}{0.4\textwidth}
        \caption{Рассеяние гауссового пучка ширины $w = 1700$ нм на слое квазирегулярно расположенных кластеров размера $ka = 0.7$ ($a \approx 8.9$ нм), $\theta_0 = 14.32^{\circ}$. Границы газового слоя обозначены пурпурным цветом. Амплитуда $|\vectbf{E}{s}|^2$ построена в плоскости поляризации падающей волны, нормирована на максимальную амплитуду в случае рассеяния 10 гармоники.}\label{random_ka0.7:image}
    \end{figure}

    \img[components/img/celes/energy_vs_nonreg]{Зависимость относительной характеристики \autoref{integrate_sc_E} от нерегулярности $\eta$ при $w = 1700$ нм, $\theta_0 = 14.32^{\circ}$.}{energy_vs_nonreg:image}{0.55\textwidth}

% \subsubsection{Учет квазимонохроматичности падающего поля}

% Гармоническое излучение состоит из множества частот с хорошо определенными фазами, зависящими от природы излучающей среды. Для каждой гармоники условия рассеяния разные, так как нормированные величины определяют картину рассеянного поля. Была получена обобщенная картина рассеяного поля в случае волнового пакета, включающего в себя гармоники с 8 по 12.

\subsubsection{Направленная энергия в зависимости от нерегулярности расположения кластеров}

Для того, чтобы определить, как нерегулярность расположения кластеров в слое влияет на количество излучения, отклоненного от направления падения, было смоделировано рассеяние на множествах кластеров с различным показателем нерегулярности $\eta$ в соответствии с \autoref{random_shifts} и посчитана нормированная характеристика \autoref{e_int} на прямоугольной области с шириной $w$, соответствующей ширине падающего пучка, вне газового слоя в направлении дифракционного максимума $\Delta\theta = 2\theta_0$ (\autoref{energy_vs_nonreg:image}):

    \begin{equation}
        E_{\textrm{int}}^{\textrm{norm}} \left( 2\theta_0,\: 0, \:w, \:\eta\right) = \cfrac{E_{\textrm{int}} \left( 2\theta_0,\: 0, \:w, \:\eta\right)}{E_{\textrm{int}} \left( 2\theta_0,\: 0, \:w, \:0\right)}
        \label{integrate_sc_E}
    \end{equation}
\end{comment}