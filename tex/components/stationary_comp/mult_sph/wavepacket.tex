\subsubsection{Рассеяние волнового пакета}

Рассмотрим рассеяние волнового пакета, амплитуда которого описывается гауссовой функцией в пространстве длин волн (\autoref{gaussian_E0}). Аппроксимируем такой волновой пакет кратными гармониками (\autoref{gauss_wavepacket_lambda:image}).

%Аппроксимируем такой волновой пакет 10-ой гармоникой, рассмотренной ранее и четырьмя ближайшими к ней, то есть с 8-ой по 12-ую (\autoref{gauss_wavepacket_lambda:image}). 

Для того, чтобы построить диаграмму рассеяния, была использована новая интегральная характеристика, определенная с учетом формы амплитуды волнового пакета и его дискретного состава (\autoref{wavepacket_eint}). Такая характеристика разумна для описания направлений рассеяния в силу аддитивности энергии как количественной характеристики. Область $V$ в данном случае представляет собой аналогичную той, что была использована для предыдущей интегральной характеристики (\autoref{V_for_e_int}, \autoref{eint_scheme:image}).

    % \begin{equation}
    %     \vectbf{E}{i} = \sum_{n = 1}^{\infty} E_{0, \: n} \:e^{i n \omega_0 t-ikz}\:\vectbf{e}{x}\label{gaussian_E}
    % \end{equation}

    \begin{equation}
        E_0 \left( \lambda \right) = \frac{1}{\sigma\sqrt{2\pi}}\,e^{-\frac{1}{2}{\left(\frac{\lambda - \mu}{\sigma}\right)}^2}\label{gaussian_E0}
    \end{equation}

    \begin{equation}
        \EuScript{E}_{\textrm{int}} \left(V, \:\eta\right) = \sum\limits_{i\:=\:N_1\:>\:0}^{N_2}{E_{\textrm{int}} \left( \eta,\:\lambda_{i}, \:V, \:E_0 \left( \lambda_{i} \right) \right)}
        \label{wavepacket_eint}
    \end{equation}

    \begin{equation*}
        \lambda_{i} = \frac{\lambda_{L}}{i}, \quad i \in \EuScript{N}\label{wavepacket_lambdas}
    \end{equation*}

Определим наиболее интенсивные направления рассеянного поля для решётки с $d = 2\lambda_{10}$, радиусом кластеров $a = 20$~nm, $\theta_0 = 15^\circ$, $\varphi_0 = 0^\circ$, гармоники в волновом пакете с 8-ой по 12-ую, то есть $N_1 = 8$, $N_2 = 12$ в \autoref{wavepacket_eint}, гауссова амплитуда охарактеризована $\mu = 83$~nm, $\sigma = 11$~nm в \autoref{gaussian_E0} (\autoref{wavepacket1:b}). 

Сравнивая полученный результат с аналогичной диаграммой, вычисленной при помощи \autoref{e_int} только для 10-ой гармоники, можно заметить значительное усиление нулевого дифракционного максимума ($h^\prime = k^\prime = l^\prime = 0$), отвечающего прошедшему излучению, и небольшое усиление остальных, что полностью соответствует \autoref{bragg_wolf_order_spherical} в силу того, что индексы Миллера, отвечающие дифракционным уравнениям для разных длин волн будут связаны между собой коэффициентами пропорциональности. Таким образом имеем масштабирование кривых, отвечающих целочисленным значениям индексов Миллера.

На \Autoref{wavepacket1:a, wavepacket1:b} видно, что немонохроматичность излучения вносит погрешность для отклонения в заданном направлении --- наиболее интенсивные направления дифракции расплываются (за исключением нулевого).

    \img[components/img/celes/wavepacket8..12_rad]{Амплитуда элементов гауссова волнового пакета напряженности поля в зависимости от длины волны. Синяя кривая соответствует \autoref{gaussian_E0} с $\mu = 83$~nm, $\sigma = 11$~nm, черные точки --- дискретная аппроксимация волнового пакета, гармоники с 12-ой по 8-ую слева направо.}{gauss_wavepacket_lambda:image}{0.45\textwidth}

    \begin{figure}[ht]
        \subimgtwo[components/img/celes/e_int_wavepacket10_rad_20nm_15deg]{Рассеяние 10-ой гармоники.}{wavepacket1:a}{0.35\textwidth}
        \hfil
        \subimgtwo[components/img/celes/e_int_wavepacket8..12_rad_20nm_15deg]{Рассеяние волнового пакета из гармоник с 8-ой по 12-ую.}{wavepacket1:b}{0.35\textwidth}
        \caption{Полярная диаграмма рассеяния гауссового волнового пакета в сравнении с рассеянием 10-ой гармоники. $\theta_0 = 15^\circ$, $\varphi_0 = 0^\circ$, $d = 2\lambda_{10}$, радиус кластеров $a = 20$ nm.}\label{wavepacket1:image}
    \end{figure}