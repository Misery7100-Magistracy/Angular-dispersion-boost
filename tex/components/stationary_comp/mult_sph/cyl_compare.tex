\subsubsection{Сравнение с цилиндрической симметрией}

Было проведено сравнение результатов моделирования рассеяния на массиве цилиндров и массиве сферических кластеров с аналогичными начальными параметрами: радиус рассеивателей, пространственное расстояние между ними, параметры падающего поля. Результаты значительно отличаются, в частности в сферическом случае заметны отдельные дифракционные максимумы с малой эффективностью, при этом большая часть поля локализована около решетки кластеров. Направления дифракции соответствуют ранее описанной теории (\autoref{bragg_wolf_order_spherical}) с учетом $\vectbf{D}{z} = 0$.

\begin{figure}[ht]
    \subimgtwo[components/img/celes/plane_flat_to_compare.pdf]{Рассеяние массивом сфер, падающее поле направлено вдоль оси $z$.}{cyl_compare:a}{0.42\textwidth}
    \hfil
    \subimgtwo[components/img/external/oe-28_screen_mult2.png]{Рассеяние массивом цилиндров~\cite{andreev_lecz}, падающее поле направлено из левого нижнего угла в правый верхний под углом $\theta_0$.}{cyl_compare:b}{0.375\textwidth}
    \caption{Рассеяние гармоники с длиной волны $\lambda \approx 89$ nm массивом сфер (а) и массивом цилиндров (б), $\varphi_0 = 0^\circ$, $\theta_0 = 30^\circ$, радиус сферических кластеров и цилиндров $a = 30$ nm, расстояние между кластерами и цилиндрами $d = 3\lambda$.}\label{cyl_compare:image}
\end{figure}

% plane_flat_to_compare