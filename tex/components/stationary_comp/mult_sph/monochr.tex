\subsubsection{Рассеяние монохроматического излучения}

Для того, чтобы проверить достоверность полученной теории, смоделируем стационарное взаимодействие в регулярном случае при различных параметрах решётки и ширине гауссова пучка $w = 800$ nm, радиусе цилиндра, ограничивающего решетку (радиус газовой струи) $r_{\textrm{gas}} = a + 12d \approx 2$ $\upmu\textrm{m}$, где множитель при $d$ --- количество узлов решетки между центральной осью и границей цилиндра. Несмотря на то, что в реальных условиях гауссов пучок $10$-ой гармоники Ti:Sa лазера с шириной 800 nm получить практически невозможно, в силу стационарности вычислений отношение $w\:/\:r_{\textrm{gas}}$ может быть корректно масштабировано при $w \ll 2r_{\textrm{gas}}$. Использованное малое значение $w$ в таком случае ускоряет вычисления, но принципиально не изменяет их результат.

Различие рассеяния в резонансном и нерезонансном случае показано на \autoref{random_ka0.7:image} --- квадрат амплитуды рассеянного поля превышает таковой в отсутствии резонанса более, чем в 10 раз, а также в этом случае нет порядков дифракции, кроме нулевого, что напрямую следует из $\lambda\:/\:d = \lambda_L\:/\:2\lambda_{10} = 5$ в \autoref{bragg_wolf_order_spherical}.

Определим наиболее интенсивные направления рассеяния при помощи следующей интегральной характеристики: %\autoref{1st_check_diffrth:b}:

    \begin{equation}
        E_{\textrm{int}} \left( \eta,\:\lambda, \:V\left(\:\Delta \theta,\:\Delta \varphi \right), \:E_0,\:\varphi_0,\:\theta_0 \right) = \int\limits_{V\left(\:\Delta \theta,\:\Delta \varphi \right)}  |\vectbf{E}{s}\left(\eta,\:\lambda,\:E_0,\:\varphi_0,\:\theta_0\right)|^2 dV.
        \label{e_int}
    \end{equation}

В данном случае \autoref{e_int} представляет собой интегрирование интенсивности рассеянного поля, вычисленного как квадрат модуля напряженности рассеянного поля (без учета нормирующего множителя) в области пространства $V$ для решётки, обладающей нерегулярностью $\eta$, то есть является энергией, рассеянной решёткой в область $V$, $\lambda$ представляет собой длину волны падающего поля, $E_0$ --- амплитуду, углы $\varphi_0$, $\theta_0$ --- задают положение мишени в пространстве в соответствии с \autoref{3ddiffr:image}. Область $V$ должна быть задана так, чтобы характеризовать некоторое направление в пространстве, и, как правило, для этой цели используется область в виде конуса, образованного некоторым раствором телесного угла $\delta \Omega$ и направлением при помощи углов $\Delta \theta$, $\Delta \varphi$. Для рассматриваемой задачи необходимо исключить из вычисления ближнее поле, ввиду чего накладывается дополнительное ограничение $V$ внутренностью сферического слоя пространства с границами $b_1$ и $b_2$, где $b_2$ --- граница области численного моделирования, $b_1$ --- больше радиуса сферы, описанной вокруг мишени. Хотя газовая струя является протяженным объектом, в моделировании используется только её сегмент, так как падающий пучок ограничен и рассеяное поле слабо зависит от частей струи, удаленных от области падения пучка, что позволяет описать вокруг такого сегмента соответствующую окружность.

    \img[components/img/eint_scheme]{Схематическое изображение области $V$ (\autoref{V_for_e_int}).}{eint_scheme:image}{0.45\textwidth}

Пересечение конической области и сферического слоя вдали от мишени можно приблизить цилиндром, считая $\rho \approx 0.5\,b_2\cdot\delta \Omega$, где $\rho$ --- радиус цилиндра. В таком случае, при помощи вспомогательного вектора $\vectbf{c}{}$ (\autoref{c_for_e_int}) получаем область $V$ (\autoref{eint_scheme:image}).

    \begin{align}
        \vectbf{c}{} = \vectbf{c}{}\left(\vectbf{x}{},\:\Delta \theta,\:\Delta \varphi \right) = \begin{pmatrix}c_{x}\\c_{y}\\c_{z}\end{pmatrix} = M_y(\Delta \theta)\,M_z(\Delta \varphi)\:\vectbf{x}{}, \quad \vectbf{x}{} = \begin{pmatrix}x\\y\\z\end{pmatrix},
        \label{c_for_e_int}
    \end{align}
    \begin{align}
        V\left(\:\Delta \theta,\:\Delta \varphi \right) = \left\{\vectbf{x}{} : c_{x}^2 + c_{y}^2 \leq \rho^2, \:\: b_1^2 \leq |\vectbf{x}{}|^2 \leq b_2^2 \right\},
        \label{V_for_e_int}
    \end{align}

\noindentгде $M_y(\Delta \theta)$ --- матрица поворота вокруг декартовой оси $y$ на угол $\Delta\theta$ против часовой стрелки, $M_z(\Delta\varphi)$ --- матрица поворота вокруг декартовой оси $z$ на угол $\Delta\varphi$ против часовой стрелки. В дальнейшем взяты значения $\rho = w\:/\:4$, $b_1 = 4r_{\textrm{gas}}$ где $w$ --- ширина гауссова пучка падающего поля, $r_{\textrm{gas}}$ --- радиус газовой струи, формирующей мишень.

\begin{figure}[H]
    \subimgtwo[components/img/celes/Es_20nm_15deg_1harm.pdf]{$\lambda = \lambda_{L} = 830$ nm.}{random_ka0.7:a}{0.42\textwidth}
    \hfil
    \subimgtwo[components/img/celes/Es_20nm_15deg_10harm.pdf]{ $\lambda = \lambda_{10} = 83$ nm.}{random_ka0.7:b}{0.42\textwidth}
    \caption{$|\vectbf{E}{s}|^2$  в плоскости поляризации, сечение $\Delta \varphi = 0$ --- рассеяние гауссового пучка на слое квазирегулярно расположенных кластеров радиуса $a = 20$ nm, $\theta_0 = 15^{\circ}$, $\varphi_0 = 0^{\circ}$. Границы газового слоя обозначены пурпурным цветом. Амплитуда рассеянного поля нормирована на максимальное значение в случае 10 гармоники.}\label{random_ka0.7:image}
\end{figure}

\begin{figure}[H]
    \subimgtwo[components/img/celes/dphi_dtheta_kprime_d_2l_phi0_0_theta0_15.pdf]{Решение \autoref{bragg_wolf_order_spherical} в целых индексах Миллера.}{1st_check_diffrth:a}{0.335\textwidth}
    \hfil
    \subimgtwo[components/img/celes/E_squared/eint_10harm_15deg_0.0nonreg.pdf]{ $E_{\textrm{int}}$ по \autoref{e_int}.}{1st_check_diffrth:b}{0.42\textwidth}
    \caption{Рассеяние 10-ой гармоники при параметрах решетки $a = 20$ nm и $d = 2\lambda_{10}$, $\varphi_0 = 0^{\circ}$, $\theta_0 = 15^{\circ}$, $\lambda = \lambda_{10} = 83$ nm, диапазон построения $\Delta \theta \in \left[ 0,\:\pi\,/\,2 \right]$.}\label{1st_check_diffrth:image}
\end{figure}

Также построим целочисленные решения для $h^\prime,\:k^\prime,\:l^\prime$ с заданными $\theta_0$, $\varphi_0$ в осях $\Delta \varphi$, $\Delta \theta$ при помощи \autoref{bragg_wolf_order_spherical} (\Autoref{1st_check_diffrth:a}). Графики на \autoref{1st_check_diffrth:image} представляют собой диаграммы в полярных координатах $(\Delta \theta, \: \Delta \varphi)$, то есть проекции поверхности сферы, ограниченной некоторым диапазоном угла $\Delta \theta$, на плоскость. Такой метод построения более удобный для изображения пространственного распределения рассеянного излучения, а также более естественный для отображения целочисленных решений на индексы Миллера (\autoref{bragg_wolf_order_spherical}), так как они в таком случае представляют собой наборы колец на сфере. По полученным результатам можно заметить, что наиболее интенсивные направления дифракции по $E_{\textrm{int}}$ отвечают наиболее близкому раположению кривых, соответствующих целочисленным значениям индексов Миллера.

Рассмотрим влияние нерегулярности решетки на наиболее интенсивные направления рассеянного поля, то есть зависимость \autoref{e_int} от нерегулярности решётки $\eta$. Для этого смоделировано рассеяние в случае квазирегулярной решётки при различных параметрах и нерегулярностью $\eta$ от 0 до 0.5 (\Autoref{nonreg_ka0.7:image}).

    % \begin{figure}[H]
    %     \subimgtwo[components/img/celes/e_int_rad_20nm_20deg_nonreg0.0]{$\eta = 0.0$.}{nonreg_mono:a}{0.42\textwidth}
    %     \hfil
    %     \subimgtwo[components/img/celes/e_int_rad_20nm_20deg_nonreg0.1]{$\eta = 0.1$.}{nonreg_mono:b}{0.42\textwidth}
    %     \caption{Характеристика \autoref{e_int} при различной нерегулярности решетки $\eta$, \\$a = 50$ nm, $d = 2\lambda_{10}$, $\varphi_0 = 0^{\circ}$, $\theta_0 = 20^{\circ}$, $\lambda = \lambda_{10} = 83$ nm, диапазон построения $\Delta \theta \in \left[ 0,\:\pi\,/\,2 \right]$.}\label{nonreg_mono:image}
    % \end{figure}

    \begin{figure}[H]

        \subimgtwo[components/img/celes/E_squared/e_int_20nm_15deg_0.1nonreg.pdf]{$\eta = 0.1$.}{nonreg_ka0.7:b}{0.42\textwidth}
        \hfil
        \subimgtwo[components/img/celes/E_squared/e_int_20nm_15deg_0.3nonreg]{$\eta = 0.3$.}{nonreg_ka0.7:c}{0.42\textwidth}
        \\
        \subimgtwo[components/img/celes/Es_20nm_15deg_10harm_0.1nonreg.pdf]{$|\vectbf{E}{s}|^2$  в плоскости поляризации, сечение $\Delta \varphi = 0$, $\eta = 0.1$.}{nonreg_ka0.7:a}{0.42\textwidth}
        \caption{Рассеяние гауссового пучка на слое квазирегулярно расположенных кластеров радиуса $a = 20$ nm, $\theta_0 = 15^{\circ}$, $\varphi_0 = 0^{\circ}$.}\label{nonreg_ka0.7:image}
    \end{figure}

    % \begin{figure}[ht]
    %     \subimgtwo[components/img/celes/mean_field_0_42]{Рассеяние $10$-ой гармоники, $\lambda_{10} = 83$ нм, $\eta = 0.43$.}{random_ka0.7:a}{0.4\textwidth}
    %     \hfil
    %     \subimgtwo[components/img/celes/mean_field_0_42_1harm]{Рассеяние лазерной гармоники, $\lambda_{L} = 830$ нм, $\eta = 0.43$.}{random_ka0.7:b}{0.4\textwidth}
    %     \subimgtwo[components/img/celes/reference_regular_14.324]{Рассеяние $10$-ой гармоники, $\lambda_{10} = 83$ нм, $\eta = 0$.}{random_ka0.7:c}{0.4\textwidth}
    %     % \hfil
    %     % \subimgtwo[components/img/celes/check20_rad]{Рассеяние $10$-ой гармоники, $\lambda_{10} = 83$ нм, $|\Delta d|_{\max} = 0$.}{random_ka0.7:c}{0.4\textwidth}
    %     \caption{Рассеяние гауссового пучка ширины $w = 1700$ нм на слое квазирегулярно расположенных кластеров размера $ka = 0.7$ ($a \approx 8.9$ нм), $\theta_0 = 14.32^{\circ}$. Границы газового слоя обозначены пурпурным цветом. Амплитуда $|\vectbf{E}{s}|^2$ построена в плоскости поляризации падающей волны, нормирована на максимальную амплитуду в случае рассеяния 10 гармоники.}\label{random_ka0.7:image}
    % \end{figure}

При помощи нормированного варианта $E_{\textrm{int}}$ (\autoref{e_int_norm}) была построена зависимость изменения рассеяния от нерегулярности решетки (\autoref{energy_vs_nonreg:image}). C ростом нерегулярности решётки наиболее интенсивные направления, кроме нулевого, значительно ослабляются, при этом общая картина рассеяния в различных направлениях выравнивается вплоть до практически однородной при $\eta \to 0.5$ (\autoref{nonreg_ka0.7:image}).

    \begin{equation}
        E_{\textrm{int}}^{\textrm{norm}} \left( \eta,\:\lambda, \:V, \:E_0 \right) = \frac{E_{\textrm{int}} \left( \eta,\:\lambda, \:V, \:E_0 \right)}{E_{\textrm{int}} \left( 0,\:\lambda, \:V, \:E_0 \right)}\label{e_int_norm}
    \end{equation}

    \img[components/img/celes/energy_vs_nonreg]{Ослабление рассеяния в зависимости от нерегулярности решётки.}{energy_vs_nonreg:image}{0.5\textwidth}

Была рассмотрена завивимость интенсивных направлений рассеяния по $\Delta \theta$ в зависимости от $\theta_0$ в сечении $\Delta \varphi = 0$. При этом было взято значение $\varphi_0 = 0$, так как любое ненулевое значение этого угла в сути поворачивает угловое распределение на тот же угол, что следует из \autoref{bragg_wolf_order_spherical}. Полученный результат полностью соотносится с описанной ранее теорией дифракции (\autoref{bragg_wolf_order_spherical}) --- положение пятен, отвечающих наиболее интенсивным направлениям рассеянного поля, отличным от нулевого, на \autoref{theta0_dphi:b} соответствуют пересечениям линий целочисленных значений индексов Миллера на \autoref{theta0_dphi:a}.

% components/img/celes/theta0_dtheta_dphi_phi0_0_miller
    % \img[components/img/celes/theta0_dtheta_dphi_phi0_0.pdf]{Характеристика \autoref{e_int} при различном угле $\theta_0$, $\varphi_0 = 0$, $\Delta \varphi = 0$, $a = 20$ nm, $d = 2\lambda_{10}$.}{dtheta_vs_theta0_eint:image}{0.54\textwidth}

    \begin{figure}[H]
        \subimgtwo[components/img/celes/theta0_dtheta_dphi_phi0_0_miller]{Решение \autoref{bragg_wolf_order_spherical} в целых индексах Миллера.}{theta0_dphi:a}{0.397\textwidth}
        \hfil
        \subimgtwo[components/img/celes/theta0_dtheta_dphi_phi0_0]{$E_{\textrm{int}}$ по \autoref{e_int}.}{theta0_dphi:b}{0.45\textwidth}
        \caption{Рассеяние 10-ой гармоники при различном угле $\theta_0$, $\varphi_0 = 0$, $\Delta \varphi = 0$, $a = 20$ nm, $d = 2\lambda_{10}$. На (а) $k^\prime = 0$ для любых $\theta_0$ и $\Delta \theta$.}\label{theta0_dphi:image}
    \end{figure}

    %\img[components/img/celes/20_rad_1st_check]{Рассеяние гауссового пучка ширины $w = 1700$ нм на слое регулярной решетке кластеров размера $a = 20$ нм, $\theta_0 = 20^{\circ}$. Границы газового слоя обозначены пурпурным цветом. Амплитуда $|\vectbf{E}{s}|^2$ построена в плоскости поляризации падающей волны, нормированая на собственный максимум.}{20rad_1st_check:image}{0.4\textwidth}

Было выбрано наиболее интенсивное направление рассеяния, отвечающее $\theta_0 = 15^\circ$, $\Delta \theta = 30^\circ$, для него рассмотрена зависимость величины ранее построенной интегральной характеристики (\autoref{e_int}) от радиуса кластеров (\autoref{energy_vs_radius:image}). Видно наличие глобального максимума у построенной зависимости, что позволяет говорить о существовании наиболее оптимального значения радиуса для соответствующего направления. 

\img[components/img/celes/energy_vs_radius]{Ослабление рассеяния в зависимости от радиуса кластеров, значения нормированы на $E_{\textrm{int}}$ при $a = 20$ nm.}{energy_vs_radius:image}{0.5\textwidth}

Таким образом, мы имеем алгоритм нахождения оптимальных параметров для рассеяния в заданном направлении. Задавая большое $d$ для решетки, мы увеличиваем количество реализуемых дифракционных максимумов, снижая эффективность (отношение диаметра кластера к расстоянию между кластерами уменьшается), но при этом увеличивая диапазон углов, куда можно потенциально отклонить излучение, при этом малое $d$ обеспечивает наиболее интенсивное рассеяние в ближние порядки дифракции, но имеет ограниченный набор углов, расположенный вблизи пересечений кривых целочисленных решений для \autoref{bragg_wolf_order_spherical}, как это хорошо видно на \autoref{theta0_dphi:b}. Увеличение нерегулярности решётки $\eta$ заставляет дифракционные максимумы расплываться и ослабляться, соответственно, чем более регулярная структура газовой струи, тем качественнее усиление рассеяния в заданном направлении. 

% Также выявлено ослабление интенсивных направлений рассеяния в зависимости от радиуса кластеров в решётке (\Autoref{radius_wawa:image, energy_vs_radius:image}). При увеличении радиуса кластеров происходит распредление энергии между большим числом направлений (\Autoref{radius_wawa:image}) в соответствии с расположением пересечений кривых, соответствующих целочисленным индексам Миллера (\autoref{1st_check_diffrth:a}), то есть перераспределение в более высокие порядки дифракции по $h^\prime,\:k^\prime,\:l^\prime$. Это связано с уменьшением свободного пространства между узлами решётки при неизменном периоде $d$ --- количество прошедшего излучения уменьшается, расходимость прошедшего через решётку излучения увеличивается, ввиду чего дифракционная картина расплывается, а дифракционные максимумы ослабляются~\cite{born_wolf}.

    % \begin{figure}[H]
    %     \subimgtwo[components/img/celes/e_int_30nm_15deg.pdf]{$a = 30$ nm.}{radius_wawa:a}{0.42\textwidth}
    %     \hfil
    %     \subimgtwo[components/img/celes/e_int_60nm_15deg.pdf]{$a = 60$ nm.}{radius_wawa:b}{0.42\textwidth}
    %     \caption{Характеристика \autoref{e_int} при радиусе кластеров $a = 30$ и $60$ nm, $d = 2\lambda_{10}$, $\varphi_0 = 0^{\circ}$, $\theta_0 = 15^{\circ}$, $\lambda = \lambda_{10} = 83$ nm, диапазон построения $\Delta \theta \in \left[ 0,\:\pi\,/\,2 \right]$.}\label{radius_wawa:image}
    % \end{figure}
