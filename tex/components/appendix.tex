\newpage
\section*{Приложение}
\subsection*{Случайный сдвиг кластера в пространственной решетке}

Процесс вычисления сдвига для отдельного кластера описывается следующим образом:

    \begin{equation}
        P_0 = (x_0,\:y_0,\:z_0)
        \label{random_shifts}
    \end{equation}
    \begin{equation*}
        \Delta_{xyz} = (\Delta_x,\: \Delta_y\:, \Delta_z) = \textrm{rand.uniform}\left( -1,\:\,1 \right)|_{size=3}
    \end{equation*}
    \begin{equation*}
        \Delta_{xyz} = \textrm{rand.uniform}\left(0,\:\,|\Delta d|_{\max} \right)\:\frac{\Delta_{xyz}}{|\Delta_{xyz}|}
    \end{equation*}
    \begin{equation*}
        P_1 = P_0 + \Delta_{xyz}
    \end{equation*}

\subsection*{Резонансная электронная плотность в первом приближении}

В зависимости от нормированного радиуса сферического кластера $x = ka$ и порядка сферической гармоники $n$:

\begin{equation}
    m^2 \left(x,\:n \right) = \frac{8n^2 (n + 1) - (6n + 3)x^2 + 6n}{2n x^2 (2n-1)} \left[ 1 + \sqrt{ 1 - \frac{4n (n-3 + 4n^2 (n + 2)) (x^2 + 4n-2) x^2}{{\left(8n^2 (n + 1) - (6n + 3)x^2 + 6n \right)}^{2}} } \right]
    \label{m2_resonance}
\end{equation}

\begin{equation}
    \frac{n_e}{n_c} = (1 - m^2) (1 + i \beta_e)
    \label{nenc_resonance}
\end{equation}