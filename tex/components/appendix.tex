\newpage
\section*{Приложение}
\subsection*{Случайный сдвиг кластера в пространственной решетке}

Процесс вычисления сдвига для отдельного кластера описывается следующим образом:

    \eq
        P_0 = (x_0,\:y_0,\:z_0)
        \label{random_shifts}
    \qe
    \eqc
        \Delta_{xyz} = (\Delta_x,\: \Delta_y\:, \Delta_z) = \textrm{rand.uniform}\left( -1,\:\,1 \right)|_{size=3}
    \cqe
    \eqc
        \Delta_{xyz} = \textrm{rand.uniform}\left(0,\:\,|\Delta d|_{\max} \right)\:\frac{\Delta_{xyz}}{|\Delta_{xyz}|}
    \cqe
    \eqc
        P_1 = P_0 + \Delta_{xyz}
    \cqe

\subsection*{Резонансная электронная плотность в первом приближении}

В зависимости от нормированного радиуса сферического кластера $x = ka$ и порядка сферической гармоники $n$:

\eq
    m^2 \left( x,\:n \right) = - \frac{8n^2 (n + 1) - 6n (x^2 + 1)}{2n x^2 (2n - 1)} \left[ 1 + \sqrt{ 1 - \frac{4n ( n - 3 + 4n^2 (n + 2)) (x^2 + 4n - 2)x^2}{8n^2 (n + 1) - 6n (x^2 + 1)} } \right]
    \label{m2_resonance}
\qe

\eq
    \frac{n_e}{n_c} = (1 - m^2) (1 + i \beta_e)
    \label{nenc_resonance}
\qe