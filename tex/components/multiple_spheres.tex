\subsection{Множество кластеров}

В рамках в рамках стационарной теории рассеяния Ми было рассмотрено множество кластеров в виде протяженного газового слоя с регулярной и квазирегулярной пространственной конфигурацией для исследования возможности рассеяния такими структурами на большие углы жёсткого ультрафиолетового излучения, в частности соответствующего гармоникам высокого порядка.

В качестве регулярного распределения была выбрана примитивная кубическая решетка c расстоянием между соседними кластерами $d$. Квазирегулярное распределение было построено при помощи внесения случайных сдвигов координат центров кластеров с произвольной нормой сдвига в диапазоне $0 \leq |\Delta d| \leq \eta d$, где $0 \leq \eta < 0.5$ --- степень нерегулярности. При $d = 2\lambda_{10}$ расстояние между соседними кластерами:

    \begin{equation}
        2\left(1 - \eta\right)\lambda_{10} \le d_{\textrm{irreg}} \le 2\left(1 + \eta\right)\lambda_{10}
    \end{equation}

В квазирегулярном случае моделирование было проведено несколько раз с целью усреднения и получения обобщенной картины рассеянного поля. Для вычислений был использован программный код CELES~\cite{celes}.

\subsubsection{Условие дифракции Брэгга-Вульфа}

%! move to single cluster section
% Наиболее интенсивное излучения ожидается в направлении распространения падающего поля, так как в данном случае моды находятся в фазе и происходит конструктивная интерференция, как и в случае одиночного кластера~\cite{boren_huffman}. Конечно, если плотность такова, чтобы быть достаточно близко к резонансному значению для гармоник высокого порядка, то рассеяние на большие углы также возможно.

Условие дифракции в случае регулярной решетки:

    \begin{equation}
        2 d \sin(\theta+\varphi) = 4 \lambda_{10} \sin(\theta+\varphi) = n \lambda, \qquad n = \frac{4 \lambda_{10}}{\lambda} \sin(\theta+\varphi),
        \label{bragg_wolf_order}
    \end{equation}

    % \begin{equation}
    %     2 d_{hkl} \sin(\theta) = \lambda, \qquad d_{hkl} = {\left( \frac{h^2}{d_x} + \frac{h^2}{d_y} + \frac{h^2}{d_z} \right)}^{-\frac{1}{2}}, \qquad n = \frac{4 \lambda_{10}}{\lambda} \sin(\theta+\varphi),
    %     \label{bragg_wolf_order}
    % \end{equation}

\noindentгде $\theta$ --- угол между направлением падающего излучения и нормалью к поверхности структуры $\vec{n}$, $\varphi$ --- угол между нормалью к поверхности $\vec{n}$ и вектором решетки структуры $\vec{K}$. Для квазирегулярного распределения в слое вместо точного $d$ использовано усредненное расстояние между кластерами, которое за счет использования равномерного распределения (\autoref{random_shifts}) в построении сдвигов будет примерно равно $d$.

    % \img[components/img/thetaphibrwf]{Схема падения излучения на поверхность структуры в трёхмерном пространстве.}{thetaphibrwf:image}{0.4\textwidth}

    \img[components/img/celes/20_rad_1st_check.pdf]{.}{20rad_1st_check:image}{0.4\textwidth}

\subsubsection{Резонансное рассеяние лазерной гармоники}

Для того, чтобы найти оптимальный угол рассеяния при помощи численного моделирования, введена следующая интегральная характеристика:

    \begin{align}
        E_{\textrm{int}} \left(\eta,\:\theta,\:w \right) = \int \limits_{S_{\theta, \:w}} dS\:|\vectbf{E}{s}|^2_{\eta \:= \eta}
        \label{e_int}
    \end{align}
    \begin{equation}
        S_{\theta, \:w} = \left\{\:x,\,y : x\tan{\theta} - \frac{w}{2\cos{\theta}} \leq y \leq x\tan{\theta} + \frac{w}{2\cos{\theta}},\:\: y > x\tan(\theta + \frac{\pi}{2}) - \frac{F\left(a, d, e\right)}{2\sin{\theta}}\:\right\},
        \label{S_for_e_int}
    \end{equation}
    \begin{align*}
        F\left(a, d, e\right) = 2a + d(e - 1),
    \end{align*}

\noindentгде интегрирование по сути представляет собой суммирование значений в указанной области $S_{\theta, \:w}$, $F\left(a, d, e\right)$ --- полная толщина газового слоя, $\theta$ --- угол падения излучения, $w$ --- параметр ширины гауссова пучка. Варьируя $\theta$, был обнаружен оптимальный для резонансного рассения угол $\theta = 14.324^{\circ}$, соответствующий наиболее интенсивному рассеянию в направлении минус первого дифракционного максимума при $d = 2\lambda_{10}$. При этом угол $\varphi = 0^{\circ}$ для простоты, а $w = 1700$ нм (\autoref{energy_vs_theta:image}).

    \img[components/img/celes/energy_vs_theta]{Зависимость относительной характеристики \autoref{integrate_sc_E} от угла падения $\theta$ при $\eta = 0$.}{energy_vs_theta:image}{0.6\textwidth}

Рассеянные поля, полученные при моделировании, представлены на \autoref{random_ka0.7:image}. В этом случае мишень более реалистична, так как состоит из материала с реалистичной электронной плотностью $n_{el} = 5.7 \cdot 10^{23}\:\,\textrm{см}^{-3} \approx 4.4 n_{c}$ для $\lambda_{10} = 83$ нм. В качестве падающего поля был использован гауссов пучок с той же интенсивностью, что и в случае с одиночным кластером $I_{L} \approx 10^{18}\:\,\textrm{Вт/см}^2$, $I_h = I_{10} \approx 10^{14}\:\,\textrm{Вт/см}^2$, параметром ширины $w = 1700$ нм, направленный вдоль оси $z$ и поляризованный вдоль оси $x$.

На \Autoref{random_ka0.7:a, random_ka0.7:b} видна значительная разница между резонансным и нерезонансным случаем --- рассеяное поле $10$-ой гармоники четко ограничено, хорошо видно рассеяние в двух направлениях, соответствующих минус первому и первому порядкам дифракции (\autoref{bragg_wolf_order}), амплитуда поля превышает таковую в отсутствии резонанса более, чем в 10 раз, что соответствует условию Брэгга-Вульфа при найденном угле.

    \begin{figure}[ht]
        \subimgtwo[components/img/celes/mean_field_0_42]{Рассеяние $10$-ой гармоники, $\lambda_{10} = 83$ нм, $\eta = 0.43$.}{random_ka0.7:a}{0.4\textwidth}
        \hfil
        \subimgtwo[components/img/celes/mean_field_0_42_1harm]{Рассеяние лазерной гармоники, $\lambda_{L} = 830$ нм, $\eta = 0.43$.}{random_ka0.7:b}{0.4\textwidth}
        \subimgtwo[components/img/celes/reference_regular_14.324]{Рассеяние $10$-ой гармоники, $\lambda_{10} = 83$ нм, $\eta = 0$.}{random_ka0.7:c}{0.4\textwidth}
        % \hfil
        % \subimgtwo[components/img/celes/check20_rad]{Рассеяние $10$-ой гармоники, $\lambda_{10} = 83$ нм, $|\Delta d|_{\max} = 0$.}{random_ka0.7:c}{0.4\textwidth}
        \caption{Рассеяние гауссового пучка ширины $w = 1700$ нм на слое квазирегулярно расположенных кластеров размера $ka = 0.7$ ($a \approx 8.9$ нм). Угол падения $\theta = 14.324^{\circ}$. Границы газового слоя обозначены пурпурным цветом. Амплитуда $|\vectbf{E}{s}|^2$ построена в плоскости поляризации падающей волны, нормирована на максимальную амплитуду в случае рассеяния 10 гармоники.}\label{random_ka0.7:image}
    \end{figure}

% \subsubsection{Учет квазимонохроматичности падающего поля}

% Гармоническое излучение состоит из множества частот с хорошо определенными фазами, зависящими от природы излучающей среды. Для каждой гармоники условия рассеяния разные, так как нормированные величины определяют картину рассеянного поля. Была получена обобщенная картина рассеяного поля в случае волнового пакета, включающего в себя гармоники с 8 по 12.

\subsubsection{Направленная энергия в зависимости от нерегулярности расположения кластеров}

Для того, чтобы определить, как нерегулярность расположения кластеров в слое влияет на количество излучения, отклоненного от направления падения, было смоделировано рассеяние на множествах кластеров с различными диапазонами нормы сдвига $\Delta d$ в соответствии с \autoref{random_shifts}. Для получения энергетической характеристики, квадрат рассеянного поля был проинтегрирован на прямоугольной области с шириной $w$, соответствующей ширине падающего пучка, вне газового слоя в направлении минус первого дифракционного максимума $\theta$ (\autoref{integrate_sc_E}).

    \begin{equation}
        E_{\textrm{int}}^{\textrm{norm}} \left(\eta,\:\theta,\:w \right) = \cfrac{E_{\textrm{int}} \left(\eta,\:\theta,\:w \right)}{E_{\textrm{int}} \left(0,\:\theta,\:w \right)}
        \label{integrate_sc_E}
    \end{equation}

\noindentгде $a$ --- радиус одиночного кластера, $d$ --- среднее расстояние между центрами соседних кластеров, $e$ --- ширина газового слоя (в количестве кластеров). Интегрирование было проведено при помощи подсчета интегральных сумм с единичным шагом, то есть суммированием значений в области интегрирования. Полученный результат был нормирован на соответствующую интегральную характеристику в случае регулярной структуры расположения кластеров (\autoref{random_ka0.7:c}).

    \img[components/img/celes/energy_vs_nonreg]{Зависимость относительной характеристики \autoref{integrate_sc_E} от нерегулярности $\eta$.}{energy_vs_nonreg:image}{0.6\textwidth}