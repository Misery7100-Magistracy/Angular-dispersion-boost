\section{Множество кластеров}

В рамках теории рассеяния Ми было рассмотрено множество кластеров в виде протяженного газового слоя с регулярной и квазирегулярной пространственной конфигурацией для исследования возможности рассеяния такими структурами на большие углы жёсткого ультрафиолетового излучения, в частности соответствующего гармоникам высокого порядка.

В качестве регулярного распределения была выбрана примитивная кубическая решетка c расстоянием между соседними кластерами $d = 2\lambda_{10}$, одинаковым во всех направлениях декартовой системы координат. Квазирегулярное распределение было построено при помощи внесения случайных сдвигов координат центров кластеров относительно регулярного расположения со случайной нормой сдвига в диапазоне $0 \leq |\Delta d| \leq 0.43d$. Таким образом расстояние между соседними кластерами имеет диапазон $0.28\lambda_{10} \le d \le 3.72\lambda_{10}$.

В квазирегулярном случае моделирование было проведено несколько раз с целью усреднения и получения обобщенной картины рассеянного поля. Для вычислений был использован программный код CELES~\cite{celes}.

\subsection{Резонансное рассеяние лазерной гармоники}

%! move to single cluster section
% Наиболее интенсивное излучения ожидается в направлении распространения падающего поля, так как в данном случае моды находятся в фазе и происходит конструктивная интерференция, как и в случае одиночного кластера~\cite{boren_huffman}. Конечно, если плотность такова, чтобы быть достаточно близко к резонансному значению для гармоник высокого порядка, то рассеяние на большие углы также возможно.

Условие Брэгга-Вульфа~\cite{boren_huffman} в случае регулярной решетки:

    \begin{equation}
        2 d \sin(\theta+\varphi) = 4 \lambda_{10} \sin(\theta+\varphi) = n \lambda, \qquad n = \frac{4 \lambda_{10}}{\lambda} \sin(\theta+\varphi),
        \label{bragg_wolf_order}
    \end{equation}

\noindentгде $\theta$ --- угол между направлением падающего излучения и нормалью к поверхности структуры, $\varphi$ --- угол между нормалью к поверхности и вектором решетки структуры. Для квазирегулярного распределения в слое вместо точного $d$ использовано усредненное расстояние между кластерами, которое за счет использования равномерного распределения (\autoref{random_shifts}) в построении сдвигов будет примерно равно $d$.

Поиском по сетке был найден угол $\theta = 14.324^{\circ}$, соответствующий наиболее интенсивному рассеянию в направлении минус первого дифракционного максимума при $d = 2\lambda_{10}$. При этом угол $\varphi$ был взят нулевым для простоты.

Рассеянные поля, полученные при моделировании, представлены на \autoref{random_ka0.7:image}. В этом случае мишень более реалистична, так как состоит из материала с реалистичной электронной плотностью $n_{el} = 5.7 \cdot 10^{23}\:\,\textrm{см}^{-3} \approx 4.4 n_{c}$ для $\lambda_{10} = 83$ нм. В качестве падающего поля был использован гауссов пучок с той же интенсивностью, что и в случае с одиночным кластером $I_{L} \approx 10^{18}\:\,\textrm{Вт/см}^2$, $I_{10} \approx 10^{14}\:\,\textrm{Вт/см}^2$, параметром ширины $w = 1700$ нм, направленный вдоль оси $z$ и поляризованный вдоль оси $x$.

На \Autoref{random_ka0.7:a, random_ka0.7:b} видна значительная разница между резонансным и нерезонансным случаем --- рассеяное поле $10$-ой гармоники четко ограничено, хорошо видно рассеяние в двух направлениях, соответствующих минус первому и первому порядкам дифракции (\autoref{bragg_wolf_order}), амплитуда поля превышает таковую в отсутствии резонанса более, чем в 10 раз, что соответствует условию Брэгга-Вульфа при найденном угле.

    \begin{figure}[H]
        \subimgtwo[components/img/celes/mean_field_0_42]{Рассеяние $10$-ой гармоники, $\lambda_{10} = 83$ нм, $|\Delta d|_{\max} = 0.43\lambda_{10}$.}{random_ka0.7:a}{0.4\textwidth}
        \hfil
        \subimgtwo[components/img/celes/mean_field_0_42_1harm]{Рассеяние лазерной гармоники, $\lambda_{L} = 830$ нм, $|\Delta d|_{\max} = 0.43\lambda_{10}$.}{random_ka0.7:b}{0.4\textwidth}
        \subimgtwo[components/img/celes/reference_regular_14.324]{Рассеяние $10$-ой гармоники, $\lambda_{10} = 83$ нм, $|\Delta d|_{\max} = 0$.}{random_ka0.7:c}{0.4\textwidth}
        \caption{Рассеяние гауссового пучка ширины $w = 1700$ нм на слое квазирегулярно расположенных кластеров размера $ka = 0.7$ ($a \approx 8.9$ нм). Угол падения $\theta = 14.324^{\circ}$. Границы газового слоя обозначены пурпурным цветом. Амплитуда $|\vectbf{E}{s}|$ построена в плоскости поляризации падающей волны, нормированная на максимальную амплитуду в случае рассеяния 10 гармоники.}\label{random_ka0.7:image}
    \end{figure}

\subsection{Учет квазимонохроматичности падающего поля}

Гармоническое излучение состоит из множества частот с хорошо определенными фазами, зависящими от природы излучающей среды. Для каждой гармоники условия рассеяния разные, так как нормированные величины определяют картину рассеянного поля. Была получена обобщенная картина рассеяного поля в случае волнового пакета, включающего в себя гармоники с 8 по 12.

\textit{рисунок}

\subsection{Направленная энергия в зависимости от нерегулярности расположения кластеров}

Для того, чтобы определить, как нерегулярность расположения кластеров в слое влияет на количество излучения, отклоненного от направления падения, было смоделировано рассеяние на множествах кластеров с различными диапазонами нормы сдвига $|\Delta d|$ в соответствии с \autoref{random_shifts}. Для получения энергетической характеристики, квадрат рассеянного поля был проинтегрирован на прямоугольной области с шириной, соответствующей ширине падающего пучка, вне газового слоя в направлении минус первого дифракционного максимума, отличающегося направления падающего пучка. 

Интегрирование было проведено при помощи подсчета интегральных сумм с единичным шагом, то есть суммированием значений в области интегрирования. Полученный результат был нормирован на соответствующую интегральную характеристику в случае регулярной структуры расположения кластеров (\autoref{random_ka0.7:c}).

\img[components/img/celes/energy_vs_nonreg]{Зависимость нормированного интеграла квадрата рассеянного поля в плоскости поляризации от нерегулярности. $|\Delta d|_{\max}$ в единицах длины волны падающего излучения $\lambda$.}{energy_vs_nonreg:image}{0.6\textwidth}