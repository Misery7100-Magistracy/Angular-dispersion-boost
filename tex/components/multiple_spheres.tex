\section{Множество кластеров}

В рамках теории рассеяния Ми было рассмотрено квазирегулярное распределение кластеров в газовом слое, построенное при помощи случайных отклонений от регулярного расположения примитивной кубической решетки. Вычисления были проведены несколько раз, результаты были усреднены для получения обобщенной картины рассеянного поля в квазирегулярном случае. Для моделирования был использован программный код CELES~\cite{celes}.

В качестве падающего поля был использован гауссов пучок с параметром ширины $w = 850$ нм, направленный вдоль оси $z$ и поляризованный вдоль $x$, как и ранее. На \autoref{random_ka0.7:image} видна значительная разница между резонансным и нерезонансным случаем - рассеяное поле $10$-ой гармоники имеет более четкие очертания, хорошо видно рассеяние в двух направлениях, соответствующих нулевому и первому порядкам дифракции по теории Брэгга-Вульфа~\cite{boren_huffman}, амплитуда поля превышает таковую в отсутствии резонанса более, чем в 10 раз.

\begin{figure}[H]
    \subimgtwo[components/img/celes/mean_field_0_42]{Рассеяние $10$-ой гармоники, $\lambda_{10} = 83$ нм.}{random_ka0.7:a}{0.48\textwidth}
    \hfil
    \subimgtwo[components/img/celes/mean_field_0_42_1harm]{Рассеяние лазерной гармоники, $\lambda_{L} = 830$ нм.}{random_ka0.7:b}{0.48\textwidth}
    \subimgtwo[components/img/celes/reference_regular_14.324]{Рассеяние $10$-ой гармоники, $\lambda_{10} = 83$ нм, регулярная структура с $d = 2\lambda_{10}$.}{random_ka0.7:c}{0.48\textwidth}
    \caption{Рассеяние гауссового пучка ширины $w = 850$ нм на слое квазирегулярно расположенных кластеров размера $ka = 0.7$ ($a \approx 8.9$ нм). Угол падения $\theta = 14.324^{\circ}$, расстояние между кластерами находится в пределах $0.28\lambda_{10} \le d \le 3.72\lambda_{10}$. Границы газового слоя обозначены пурпурным цветом. Амплитуда $|\vectbf{E}{s}|$ построена в плоскости поляризации падающей волны, нормированная на максимальную амплитуду в случае 10 гармоники.}
    \label{random_ka0.7:image}
\end{figure}

Также приведено сравнение с регулярным случаем (\autoref{random_ka0.7:c}), для которого амплитуда рассеянного в первый порядок излучения выше, а объем поля, локализованного в области газового слоя, ниже.



% Within the Mie theory following multiple clusters spatial configuration considered: simple cubic lattice with 4 edge nodes and different length of the edge of unit cell $b : \{\lambda_{10},\:2\lambda_{10},\:3\lambda_{10}\}$. Cluster radius $a = 6.37$ nm. The CELES software package was used for modeling \cite{celes}.

% For the case of $b = \lambda_{10}$ we can see efficient scattering by facets of the spatial lattice at angles $\approx 150^{\circ}$, $-150^{\circ}$ relative to the wave vector direction (\autoref{multi_sph_b1:image}d). Most of the field is localized in the area of clusters.

% For $b = 2\lambda_{10}$ and angle of incidence $\approx 30^{\circ}$ there is a slight increase in scattering at small angles (\autoref{multi_sph_b2:image}d).

% For $b = 3\lambda_{10}$ due to the increased rarefaction between clusters allows us to get rid of strong reflection, which can be seen from the far-field (\autoref{multi_sph_b3:image}d).

% \begin{figure}[H]
%     (a)\:\subimg[components/img/celes/64sph_b83nm_l83nm_0deg_near]{0.4\textwidth}
%     (b)\:\subimg[components/img/celes/64sph_b83nm_l83nm_0deg_far]{0.4\textwidth}
%     \\(c)\:\subimg[components/img/celes/64sph_b83nm_l83nm_45deg_near]{0.4\textwidth}
%     (d)\:\subimg[components/img/celes/64sph_b83nm_l83nm_45deg_far]{0.4\textwidth}
%     \caption{$10$-th harmonic scattering by multiple clusters. $b = \lambda_{10}$; a, b --- normal incidence; c, d --- incidence at $45^{\circ}$ angle; a, c --- near-field; b, d --- far-field.}
%     \label{multi_sph_b1:image}
% \end{figure}

% \begin{figure}[H]
%     (a)\:\subimg[components/img/celes/64sph_b166nm_l83nm_0deg_near]{0.4\textwidth}
%     (b)\:\subimg[components/img/celes/64sph_b166nm_l83nm_0deg_far]{0.4\textwidth}
%     \\(c)\:\subimg[components/img/celes/64sph_b166nm_l83nm_30deg_near]{0.4\textwidth}
%     (d)\:\subimg[components/img/celes/64sph_b166nm_l83nm_30deg_far]{0.4\textwidth}
%     \caption{$10$-th harmonic scattering by multiple clusters. $b = 2\lambda_{10}$; a, b --- normal incidence; c, d --- incidence at $45^{\circ}$ angle; a, c --- near-field; b, d --- far-field.}
%     \label{multi_sph_b2:image}
% \end{figure}

% \begin{figure}[H]
%     (a)\:\subimg[components/img/celes/64sph_b249nm_l83nm_0deg_near]{0.4\textwidth}
%     (b)\:\subimg[components/img/celes/64sph_b249nm_l83nm_0deg_far]{0.4\textwidth}
%     % \subimg[../checkit2]{0.45\textwidth}
%     % \subimg[../checkit]{0.45\textwidth}
%     \caption{$10$-th harmonic scattering by multiple clusters. $b = 3\lambda_{10}$; a, b --- normal incidence; c, d --- incidence at $45^{\circ}$ angle; a, c --- near-field; b, d --- far-field.}
%     \label{multi_sph_b3:image}
% \end{figure}