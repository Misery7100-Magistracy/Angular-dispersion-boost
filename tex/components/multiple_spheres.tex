\subsection{Множество кластеров}

В рамках в рамках стационарной теории рассеяния Ми было рассмотрено множество кластеров в виде протяженного газового слоя с регулярной и квазирегулярной пространственной конфигурацией для исследования возможности рассеяния такими структурами на большие углы жёсткого ультрафиолетового излучения, в частности соответствующего гармоникам высокого порядка.

В качестве регулярного распределения была выбрана примитивная кубическая решетка c расстоянием между соседними узлами $d$. Квазирегулярное распределение было построено при помощи внесения случайных сдвигов координат узлов с произвольной нормой сдвига в диапазоне $0 \leq |\Delta d| \leq \eta d$, где $0 \leq \eta < 0.5$ --- степень нерегулярности. При $d = 2\lambda$:

    \begin{equation}
        2\left(1 - \eta\right)\lambda \le d_{\textrm{irreg}} \le 2\left(1 + \eta\right)\lambda
    \end{equation}

В квазирегулярном случае моделирование было проведено несколько раз с целью усреднения и получения обобщенной картины рассеянного поля. Для вычислений был использован программный код CELES~\cite{celes}.

\subsubsection{Условие дифракции для решетки в пространстве}

%! move to single cluster section
% Наиболее интенсивное излучения ожидается в направлении распространения падающего поля, так как в данном случае моды находятся в фазе и происходит конструктивная интерференция, как и в случае одиночного кластера~\cite{boren_huffman}. Конечно, если плотность такова, чтобы быть достаточно близко к резонансному значению для гармоник высокого порядка, то рассеяние на большие углы также возможно.

Условие дифракции в случае трехмерной регулярной решетки при упругом рассеянии принимает вид~\cite{Kittel86}:

    \begin{equation}
        \begin{cases}
        \begin{aligned}
            \left( \vectbf{D}{x},\: \vectbf{e}{\textrm{out}} - \vectbf{e}{\textrm{in}}\right) &= h \lambda
            \\
            \left( \vectbf{D}{y},\: \vectbf{e}{\textrm{out}} - \vectbf{e}{\textrm{in}}\right) &= k \lambda
            \\
            \left( \vectbf{D}{z},\: \vectbf{e}{\textrm{out}} - \vectbf{e}{\textrm{in}}\right) &= l \lambda
        \end{aligned}
        \end{cases}
        \label{bragg_wolf_order}
    \end{equation}

\noindentгде $h,\:k,\:l$ --- индексы Миллера представленные целыми числами, $\vectbf{D}{i}$ --- вектор, соединяющий соседние узлы решетки вдоль направления $i$, $\vectbf{e}{\textrm{in}}$ --- единичный вектор направления падающего излучения, $\vectbf{e}{\textrm{out}}$ --- единичный вектор направления прошедшего излучения. Переходя к сферическим координатам, связанными с $\vectbf{e}{\textrm{in}}$ так, что в декартовом представлении $\vectbf{e}{\textrm{in}} = \vectbf{e}{z}$, \autoref{bragg_wolf_order} можно преобразовать следующим образом, учитывая, что $|\vectbf{D}{x}| = |\vectbf{D}{y}| = |\vectbf{D}{z}| = d$ для рассматриваемой кубической решетки:

    \begin{equation}
        \begin{cases}
            \cos{\theta_0}\sin{\Delta \theta}\cos{\left( \Delta \varphi - \varphi_0 \right)} - \sin{\theta_0} \left( \cos{\Delta \theta} - 1 \right) = \cfrac{h^{\prime} \lambda}{d}
            \\
            \sin{\Delta \theta} \sin{\left( \Delta \varphi - \varphi_0 \right)} = \cfrac{k^{\prime} \lambda}{d}
            \\
            \sin{\theta_0}\sin{\Delta \theta}\cos{\Delta \varphi} + \cos{\theta_0} \left( \cos{\Delta \theta} - 1 \right)= \cfrac{l^{\prime} \lambda}{d}
        \end{cases}
        \label{bragg_wolf_order_spherical}
    \end{equation}

\noindentгде $\Delta \theta,\:\Delta \varphi$ --- углы, характеризующие отклонение направления прошедшего излучения относительно падающего, $\theta_0,\:\varphi_0$ --- углы, характеризующие поворот мишени (решётки) в пространстве, $h^\prime,\:k^\prime\:,l^\prime$ --- новые индексы Миллера (\autoref{3ddiffr:image}).

    \begin{figure}[ht]
        \subimgtwo[components/img/3ddiffrxz]{Проекция на плоскость $xz$.}{3ddiffr:a}{0.4\textwidth}
        \hfil
        \subimgtwo[components/img/3ddiffrxy]{Проекция на плоскость $xy$.}{3ddiffr:b}{0.4\textwidth}
        \caption{Общая схема взаимодействия падающего излучения с решеткой. $\Delta \theta$ отсчитывается от положительного направления оси $z$, $\Delta \varphi$ --- от положительного направления оси $x$, против часовой стрелки.}\label{3ddiffr:image}
    \end{figure}

% При этом для отраженного излучения отклонение от исходного направления:

%     \begin{equation}
%         \begin{cases}
%             \Delta \theta_r = \pi - \Delta \theta
%             \\
%             \Delta \varphi_r = \pi - \Delta \varphi
%         \end{cases}
%     \end{equation}

Используя \autoref{bragg_wolf_order_spherical}, можем получить угловое распределение дифрагировавшего излучения. Наиболее интенсивные направления дифракции будут соответствовать минимальным по модулю индексам Миллера, тогда пусть $k^\prime = 0$:

    \begin{equation}
        \begin{cases}
            \Delta \varphi_1 = \varphi_0, \quad \Delta \varphi_2 = \pi + \varphi_0,
            \\
            \Delta \theta_1 = \theta_0 + \arcsin{\left( \cfrac{h^{\prime} \lambda}{d} - \sin{\theta_0} \right)},
            \\
            \Delta \theta_2 = \arcsin{\left( \cfrac{h^{\prime} \lambda}{d} - \sin{\theta_0} \right)} - \theta_0,
            \\
            l^{\prime}_{1,2} = \cfrac{\lambda}{d}\:\sin{\theta_0}\sin{\Delta \theta_{1,2}}\cos{\Delta \varphi_{1,2}} + \cos{\theta_0} \left( \cos{\Delta \theta_{1,2}} - 1 \right)
        \end{cases}
        \label{bragg_wolf_sol_0}
    \end{equation}

Используя \autoref{bragg_wolf_sol_0}, можно построить решения, соответствующие целым значениям $l^\prime$, которые отвечают различным прошедшим и отраженным порядкам излучения (\autoref{phi0_theta0_lprime:image}).

    \begin{figure}[ht]
        \subimgtwo[components/img/celes/phi0_theta0_lprime_d_2l_h_1]{$d = 2\lambda$, $h^\prime = 1$.}{phi0_theta0_lprime:a}{0.4\textwidth}
        \hfil
        \subimgtwo[components/img/celes/phi0_theta0_lprime_d_3l_h_1]{$d = 3\lambda$, $h^\prime = 1$.}{phi0_theta0_lprime:b}{0.4\textwidth}
        % \subimgtwo[components/img/celes/phi0_theta0_lprime_d_2l_h_2]{$d = 2\lambda$, $h^\prime = 2$.}{phi0_theta0_lprime:c}{0.4\textwidth}
        % \hfil
        % \subimgtwo[components/img/celes/phi0_theta0_lprime_d_3l_h_2]{$d = 3\lambda$, $h^\prime = 2$.}{phi0_theta0_lprime:d}{0.4\textwidth}
        \caption{Кривые, отвечающие различным дифракционным порядкам по $l^\prime$ при $k^\prime = 0$, $\Delta \varphi = \varphi_0$.}\label{phi0_theta0_lprime:image}
    \end{figure}

% Смоделируем стационарное взаимодействие в случае регулярной решётки с радиусом кластеров $a = 20$ нм и $d = 2\lambda_{10}$ при $\Delta \varphi = \varphi_0 = 0^{\circ}$, $\theta_0 = 20^{\circ}$, $\lambda = \lambda_{10} = 83$ нм, ширина гауссова пучка $w = 1700$ нм. Наиболее интенсивные направления дифракции будут соответствовать минимальным по модулю индексам Миллера, тогда по \autoref{bragg_wolf_sol_0} для $l^\prime = 0$ имеем только $\Delta \theta = 0^\circ$, соответствующий $h^\prime = 0$. Это прослеживается на \autoref{20rad_1st_check:image}, что показывает соответствие описанной ранее теории дифракции.

    \img[components/img/celes/20_rad_1st_check]{Рассеяние гауссового пучка ширины $w = 1700$ нм на слое регулярной решетке кластеров размера $a = 20$ нм, $\theta_0 = 20^{\circ}$. Границы газового слоя обозначены пурпурным цветом. Амплитуда $|\vectbf{E}{s}|^2$ построена в плоскости поляризации падающей волны, нормированая на собственный максимум.}{20rad_1st_check:image}{0.4\textwidth}

\subsubsection{Резонансное рассеяние лазерной гармоники}

%Для того, чтобы найти оптимальный угол рассеяния при помощи численного моделирования, введена следующая интегральная характеристика:

Для того, чтобы найти оптимальные углы рассеяния при помощи численного моделирования, введена следующая интегральная характеристика:

    \begin{align}
        E_{\textrm{int}} \left(\theta,\:\varphi, \:w, \:b, \:\eta\right) = \mathop{\int}\limits_{V\left(\theta,\:\varphi, \:w, \:b\right)} \left( |\vectbf{E}{s}|^2_{\eta \:= \eta} \right) dV,
        \label{e_int}
    \end{align}
    % \begin{equation}
    %     S\left(\Delta\theta,\:\Delta\varphi, \:w\right) = \left\{\:\left(i,\, j,\, k\right) : i\tan{\Delta\theta} - \frac{w}{2\cos{\Delta\theta}} \leq N - j \leq i\tan{\Delta\theta} + \frac{w}{2\cos{\Delta\theta}},\:\: N - j > -i\cot(\Delta\theta) - \frac{F\left(a, d, e\right)}{2\sin{\Delta\theta}}\:\right\},
    %     \label{S_for_e_int}
    % \end{equation}
    \begin{align}
        \vectbf{C}{} = \vectbf{C}{}\left(x,\:y,\:z,\:\theta,\:\varphi \right) = \begin{pmatrix}C_{x}\\C_{y}\\C_{z}\end{pmatrix} = M_y(\theta)\,M_z(\varphi)\begin{pmatrix}x\\y\\z\end{pmatrix},
        \label{c_for_e_int}
    \end{align}
    \begin{align}
        V\left(\theta,\:\varphi, \:w, \:b \right) = \left\{\:x,\:y,\:z : C_{x}^2 + C_{y}^2 \leq w, \:\: 0 < z \leq b \right\},
        \label{V_for_e_int}
    \end{align}
    % \begin{align*}
    %     F\left(a, d, e\right) = 2a + d(e - 1),
    % \end{align*}

\noindentгде $M_y(\theta)$ --- матрица поворота вокруг декартовой оси $y$ на угол $\theta$ против часовой стрелки, $M_z(\varphi)$ --- матрица поворота вокруг декартовой оси $z$ на угол $\varphi$ против часовой стрелки, $b$ --- граница области моделирования по $z$, $\eta$ --- коэффициент нерегулярности решетки. В таком случае \autoref{V_for_e_int} представляет собой область внутри цилиндра с радиусом $w$, наклоненного в соответствии с углами $\theta$, $\varphi$ и ограниченного плоскостями $z = 0$, $z = b$, что описывавает дифрагировавший пучок ширины $w$, отклоненный в направлении, заданном углами $\theta$, $\varphi$.

Варьируя $\theta_0$, был обнаружен оптимальный для резонансного рассения угол $\theta_0 = 14.32^{\circ}$, соответствующий наиболее интенсивному рассеянию в направлении дифракционного максимума $100$ по $h^\prime k^\prime l^\prime$ при $d = 2\lambda_{10}$, $w = 1700$ нм (\autoref{energy_vs_theta:image}).

    \img[components/img/celes/energy_vs_theta]{Зависимость относительной характеристики \autoref{integrate_sc_E} от угла падения $\theta_0$ при $\eta = 0$, $w = 1700$ нм.}{energy_vs_theta:image}{0.55\textwidth}

Рассеянные поля, полученные при моделировании, представлены на \autoref{random_ka0.7:image}. В этом случае мишень более реалистична, так как состоит из материала с реалистичной электронной плотностью $n_{el} = 5.7 \cdot 10^{23}\:\,\textrm{см}^{-3} \approx 4.4 n_{c}$ для $\lambda_{10} = 83$ нм. В качестве падающего поля был использован гауссов пучок с той же интенсивностью, что и в случае с одиночным кластером $I_{L} \approx 10^{18}\:\,\textrm{Вт/см}^2$, $I_h = I_{10} \approx 10^{14}\:\,\textrm{Вт/см}^2$, параметром ширины $w = 1700$ нм, направленный вдоль оси $z$ и поляризованный вдоль оси $x$.

На \Autoref{random_ka0.7:a, random_ka0.7:b} видна значительная разница между резонансным и нерезонансным случаем --- рассеяное поле $10$-ой гармоники четко ограничено, хорошо видно рассеяние в двух направлениях, соответствующих порядкам дифракции $000$ и $100$ по $h^\prime k^\prime l^\prime$ (\autoref{bragg_wolf_sol_0}), амплитуда поля превышает таковую в отсутствии резонанса более, чем в 10 раз при найденном угле $\Delta\theta$, что соответствует описанной ранее теории дифракции.

    \begin{figure}[ht]
        \subimgtwo[components/img/celes/mean_field_0_42]{Рассеяние $10$-ой гармоники, $\lambda_{10} = 83$ нм, $\eta = 0.43$.}{random_ka0.7:a}{0.4\textwidth}
        \hfil
        \subimgtwo[components/img/celes/mean_field_0_42_1harm]{Рассеяние лазерной гармоники, $\lambda_{L} = 830$ нм, $\eta = 0.43$.}{random_ka0.7:b}{0.4\textwidth}
        \subimgtwo[components/img/celes/reference_regular_14.324]{Рассеяние $10$-ой гармоники, $\lambda_{10} = 83$ нм, $\eta = 0$.}{random_ka0.7:c}{0.4\textwidth}
        % \hfil
        % \subimgtwo[components/img/celes/check20_rad]{Рассеяние $10$-ой гармоники, $\lambda_{10} = 83$ нм, $|\Delta d|_{\max} = 0$.}{random_ka0.7:c}{0.4\textwidth}
        \caption{Рассеяние гауссового пучка ширины $w = 1700$ нм на слое квазирегулярно расположенных кластеров размера $ka = 0.7$ ($a \approx 8.9$ нм), $\theta_0 = 14.32^{\circ}$. Границы газового слоя обозначены пурпурным цветом. Амплитуда $|\vectbf{E}{s}|^2$ построена в плоскости поляризации падающей волны, нормирована на максимальную амплитуду в случае рассеяния 10 гармоники.}\label{random_ka0.7:image}
    \end{figure}

    \img[components/img/celes/energy_vs_nonreg]{Зависимость относительной характеристики \autoref{integrate_sc_E} от нерегулярности $\eta$ при $w = 1700$ нм, $\theta_0 = 14.32^{\circ}$.}{energy_vs_nonreg:image}{0.55\textwidth}

% \subsubsection{Учет квазимонохроматичности падающего поля}

% Гармоническое излучение состоит из множества частот с хорошо определенными фазами, зависящими от природы излучающей среды. Для каждой гармоники условия рассеяния разные, так как нормированные величины определяют картину рассеянного поля. Была получена обобщенная картина рассеяного поля в случае волнового пакета, включающего в себя гармоники с 8 по 12.

\subsubsection{Направленная энергия в зависимости от нерегулярности расположения кластеров}

Для того, чтобы определить, как нерегулярность расположения кластеров в слое влияет на количество излучения, отклоненного от направления падения, было смоделировано рассеяние на множествах кластеров с различным показателем нерегулярности $\eta$ в соответствии с \autoref{random_shifts} и посчитана нормированная характеристика \autoref{e_int} на прямоугольной области с шириной $w$, соответствующей ширине падающего пучка, вне газового слоя в направлении дифракционного максимума $\Delta\theta = 2\theta_0$ (\autoref{energy_vs_nonreg:image}):

    \begin{equation}
        E_{\textrm{int}}^{\textrm{norm}} \left( 2\theta_0,\: 0, \:w, \:\eta\right) = \cfrac{E_{\textrm{int}} \left( 2\theta_0,\: 0, \:w, \:\eta\right)}{E_{\textrm{int}} \left( 2\theta_0,\: 0, \:w, \:0\right)}
        \label{integrate_sc_E}
    \end{equation}