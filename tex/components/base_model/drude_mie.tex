Рассмотрим одиночный сферический кластер радиуса $a$, облученный коротким фемтосекундным импульсом длительностью $\tau$ и интенсивностью порядка $I_{h} \approx 10^{14}$ $\textrm{W/cm}^2$, полученной в результате преобразования лазерной гармоники с коэффициентом преобразования $10^{-4}$. Модель Друде даёт представление диэлектрической функции плазмы:

    \begin{equation}
		\varepsilon (\omega) = 1 - {\left( \frac{\omega_{pe}}{\omega} \right)}^2 \frac{1}{1+i \beta_{e}}, \qquad \omega_{pe} = \sqrt{\frac{4 \pi e^2 n_e}{m_e}},
		\label{eps_plasma}
    \end{equation}

\noindent где $\omega$ --- рассмариваемая гармоническая частота, $\omega_{pe}$ --- электронная плазменная частота, $e$ и $m_e$ --- заряд и масса электрона, $n_e = Z n_i$ --- электронная плотность, где $Z$ --- средняя степень ионизации, $n_i$ --- ионная плотность. $\beta_{e} = v_e / \omega$ и $v_e$ коэффициент электрон-ионных столкновений в приближении Спитцера. В условиях твердотельной плазмы ионная плотность кластера порядка $n_i = 6 \cdot 10^{22}$ $\textrm{cm}^{-3}$, при этом электронная плотность кластера должна быть выше критической для заданной частоты $n_c = \omega^2 m_e / 4 \pi e^2$. Для 10-ой гармоники лазерного излучения $\lambda_{10} = 83$ nm мы получаем условие $n_e > n_c = 1.3 \cdot 10^{23}$ $\textrm{cm}^{-3}$, что согласуется с условием на ионную плотность при средней степени ионизации $Z > 2$.

Теория Ми может быть использована для описания упругого рассеяния электромагнитных волн частицами произвольного размера в случае линейных взаимодействий, а также позволяет получить описание рассеянного поля и поля внутри рассеивающего объекта. Основной шаг --- решение скалярного уравнения Гельмгольца в правильной системе координат (в данном случае сферической) и получение векторных решений. Для сферического кластера можем записать решение соответствующего уравнения, используя сферические функции Бесселя и Ханкеля $n$-ого порядка, включая присоединенные полиномы Лежандра~\cite{boren_huffman}.

Возьмем плоскую волну, распространяющуюся вдоль оси $z$ декартовой системы координат, поляризованную вдоль оси $x$, что может быть записано как:

    \begin{equation}
        \vectbf{E}{i} = E_0\:e^{i\omega t-ikz}\:\vectbf{e}{x},
        \label{E_i_sph}
    \end{equation}

\noindentгде $k = \omega / c$ --- волновое число, $\vectbf{e}{x}$ --- единичный вектор оси $x$, также являющийся вектором поляризации (\autoref{single_sph_scheme:image}). 

    \img[components/img/single_sph_scheme]{Схема базовой модели.}{single_sph_scheme:image}{0.6\textwidth}

    \begin{equation}
		\vectbf{E}{s} = \sum_{n = 1}^{\infty}E_n \left[ i a_n\left(ka, m\right) \vectbf{N}{}^{(3)}_{e1n} - b_n\left(ka, m\right) \vectbf{M}{}^{(3)}_{o1n} \right], \qquad E_n = i^{n} E_0 \frac{2n + 1}{n \left(n + 1\right)}
        \label{E_s_sph}
    \end{equation}

Далее эту плоскую можно разложить в ряд, используя обобщённое разложение Фурье. В случае изотропной среды имеем следующий вид коэффициентов рассеянного поля~\cite{boren_huffman}:


    \begin{equation}
		a_n(x,\:m) = \frac{m \func{\psi}{n}{\prime}{x} \func{\psi}{n}{}{mx} - \func{\psi}{n}{\prime}{mx} \func{\psi}{n}{}{x}}{m \func{\xi}{n}{\prime}{x} \func{\psi}{n}{}{mx} - \func{\psi}{n}{\prime}{mx} \func{\xi}{n}{}{x}},
		\label{an_bessel}
    \end{equation}

    \begin{equation}
        b_n(x,\:m) = \frac{\func{\psi}{n}{\prime}{x} \func{\psi}{n}{}{mx} - m \func{\psi}{n}{\prime}{mx} \func{\psi}{n}{}{x}}{\func{\xi}{n}{\prime}{x} \func{\psi}{n}{}{mx} - m \func{\psi}{n}{\prime}{mx} \func{\xi}{n}{}{x}},
        \label{bn_bessel}
    \end{equation}
    \begin{equation*} % artificial indent after the equation
    \end{equation*}

\noindentгде $\func{\psi}{n}{}{\rho} = z \func{j}{n}{}{\rho}$, $\func{\xi}{n}{}{\rho} = z \func{h}{n}{}{\rho}$ --- функции Риккати-Бесселя, $h_n = j_n + i \gamma_n$ --- сферические функции Ханкеля первого рода, $x = ka$ --- безразмерный радиус кластера, $ m = \sqrt{\varepsilon} $ --- комплексный коэффициент преломления (\autoref{eps_plasma}).