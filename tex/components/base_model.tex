\section{Базовая модель}

Рассмотрим одиночный сферический кластер радиуса $a$, облученный коротким фемтосекундным импульсом длительностью $\tau$ и интенсивностью порядка $I_{h} \approx 10^{14}$ $\textrm{Вт/см}^2$, полученной в результате преобразования лазерной гармоники с коэффициентом преобразования $10^{-4}$. Модель Друде даёт представление диэлектрической функции плазмы:

    \eq
		\varepsilon (\w) = 1 - \left( \frac{\w_{pe}}{\w} \right)^2 \frac{1}{1 + i \beta_{e}}, \qquad \w_{pe} = \sqrt{\frac{4 \pi e^2 n_e}{m_e}},
		\label{eps_plasma}
	\qe

\noindent где $\omega$ -- рассмариваемая гармоническая частота, $\omega_{pe}$ -- электронная плазменная частота, $e$ и $m_e$ -- заряд и масса электрона, $n_e = Z n_i$ -- электронная плотность, где $Z$ - средняя степень ионизации, $n_i$ -- ионная плотность. $\beta_{e} = v_e / \omega$ и $v_e$ коэффициент электрон-ионных столкновений в приближении Спитцера. В условиях твердотельной плазмы ионная плотность кластера порядка $n_i = 6 \cdot 10^{22}$ $\textrm{см}^{-3}$, при этом электронная плотность кластера должна быть выше критической для заданной частоты $n_c = \omega^2 m_e / 4 \pi e^2$. Для 10-ой гармоники лазерного излучения $\lambda_{10} = 83$ нм мы получаем условие $n_e > n_c = 1.3 \cdot 10^{23}$ $\textrm{см}^{-3}$, что согласуется с условием на ионную плотность при средней степени ионизации $Z > 2$.

Теория Ми может быть использована для описания упругого рассеяния электромагнитных волн частицами произвольного размера в случае линейных взаимодействий, а также позволяет получить описание рассеянного поля и поля внутри рассеивающего объекта. Основной шаг - решение скалярного уравнения Гельмгольца в правильной системе координат (в данном случае сферической) и получение векторных решений. Для сферического кластера можем записать решение соответствующего уравнения, используя сферические функции Бесселя и Ханкеля $n$-ого порядка, включая присоединенные полиномы Лежандра~\cite{boren_huffman}.

Возьмем плоскую волну, распространяющуюся вдоль оси $z$ декартовой системы координат, поляризованную вдоль оси $x$, что может быть записано как:

    \eq
        \vectbf{E}{i} = E_0\:e^{i\w t - ikz}\:\vectbf{e}{x},
        \label{E_i_sph}
    \qe

\noindentгде $k = \omega/c$ - волновое число, $\vectbf{e}{x}$ - единичный вектор оси $x$, также являющийся вектором поляризации (\autoref{single_sph_scheme:image}). 

    \img[components/img/single_sph_scheme]{Схема базовой модели.}{single_sph_scheme:image}{0.73\textwidth}

    \eq
		\vectbf{E}{s} = \sum_{n = 1}^{\infty}E_n \left[ i a_n\left(ka, m\right) \vectbf{N}{}^{(3)}_{e1n} - b_n\left(ka, m\right) \vectbf{M}{}^{(3)}_{o1n} \right], \qquad E_n = i^{n} E_0 \frac{2n + 1}{n \left(n + 1\right)}
        \label{E_s_sph}
	\qe

Далее эту плоскую можно разложить в ряд, используя обобщённое разложение Фурье. В случае изотропной среды имеем следующий вид коэффициентов рассеянного поля \cite{boren_huffman}:


    \eq
		a_n(x,\:m) = \frac{m \func{\psi}{n}{\prime}{x} \func{\psi}{n}{}{mx} - \func{\psi}{n}{\prime}{mx} \func{\psi}{n}{}{x}}{m \func{\xi}{n}{\prime}{x} \func{\psi}{n}{}{mx} - \func{\psi}{n}{\prime}{mx} \func{\xi}{n}{}{x}},
		\label{an_bessel}
	\qe

    \eq
        b_n(x,\:m) = \frac{\func{\psi}{n}{\prime}{x} \func{\psi}{n}{}{mx} - m \func{\psi}{n}{\prime}{mx} \func{\psi}{n}{}{x}}{\func{\xi}{n}{\prime}{x} \func{\psi}{n}{}{mx} - m \func{\psi}{n}{\prime}{mx} \func{\xi}{n}{}{x}},
        \label{bn_bessel}
    \qe
    \eqc % artificial indent after the equation
    \cqe %

\noindentгде $\func{\psi}{n}{}{\rho} = z \func{j}{n}{}{\rho}$, $\func{\xi}{n}{}{\rho} = z \func{h}{n}{}{\rho}$ -- функции Риккати-Бесселя, $h_n = j_n + i \gamma_n$ -- сферические функции Ханкеля первого рода, $x = ka$ -- безразмерный радиус кластера, $ m = \sqrt{\varepsilon} $ -- комплексный коэффициент преломления (\autoref{eps_plasma}).

\subsection{Асимптотическое приближение коэффициентов рассеянного поля}

В случае сферической симметрии амплитуда рассеянного поля максимальна для $m^2 = - (n+ 1) / n$ при $ka \ll 1$, что даёт соответствующий набор резонансных плотностей в бесстолкновительном случае $n_e = n_c(2n + 1) / n$. Это можно получить, используя нулевое асимптотическое приближение функций Бесселя, в результате чего коэффициенты (\Autoref{an_bessel, bn_bessel}) значительно упрощаются:

    \eq
        a_n\left( x \to 0,\:m \right) = \left( 1 + 2i \frac{ (2n - 1)! (2n + 1)!}{4^n \: n! (n + 1)!} \frac{\left(m^2 + \frac{n + 1}{n} \right)}{(m^2 - 1)} \frac{1}{x^{2n+1}} \right)^{-1}, \qquad b_n\left( x \to 0,\:m \right) = 0
        \label{ab_asymp}
    \qe

Такое приближение можно использовать вместо (\Autoref{an_bessel, bn_bessel}) для объектов достаточно маленького радиуса, но уже при $ka \sim 1$ оно перестаёт быть разумным, особенно для больших $n$. Вместо него в таком случае лучше подходит аппроксимация первого порядка:

    \eq
		a_n\left( x ,\:m \right) = \left( 1 + i \frac{ C_n x^{-1 -2n} \left( (4(1 + n + m^2 n) (-3 + 4n (1 + n)) - 2(m^2 - 1)(3 + n(5 + 2n + m^2 (2n - 1))) x^2) \right)}{\pi (m^2 - 1)(2n + 3)(n + 1)(4(2n + 3) - 2(m^2 + 1)x^2)} \right)^{-1}
		\label{an_sph_asymp1}
	\qe
	\eqc
		C_n = 2^{1 + 2n} \Gamma(n - \frac{1}{2}) \Gamma(n + \frac{5}{2})
	\cqe

На \autoref{ab_asymp:image} показана зависимость коэффициента рассеяния от электронной плотности для двух различных значений радиуса в рамках нулевого асимптотического приближения и сравнение первого и нулевого приближений. Видно, что с ростом $n$ ширина резонансного пика быстро уменьшается, а также б\'{о}льшим радиусам (безразмерным) $ka$ соответствует б\'{о}льшая их ширина. Помимо этого, с ростом радиуса растет и значение резонансной плотности, что видно на \autoref{nenc_123:image}.

    \begin{figure}[H]
        \subimgtwo[components/img/sph_base/sph_ka0.5_123]{$ka = 0.5$.}{ab_asymp:a}{0.66\textwidth}\\
        \subimgtwo[components/img/sph_base/sph_ka1.5_123]{$ka = 1.5$.}{ab_asymp:b}{0.66\textwidth}
		\caption{Коэффициенты сферических гармоник в приближении нулевого порядка, $\beta_e = 0$. Кривые "exact" построены с использованием полных разложений.}
		\label{ab_asymp:image}
	\end{figure}

    \img[components/img/sph_base/sph_ka1.5_123_1st]{$ka = 1.5$ в приближении первого порядка. $\beta_e = 0$. Кривые "exact" построены с использованием полных разложений.}{ab1:image}{0.66\textwidth}

Такие аппроксимации позволяют оценить резонансные параметры кластера, в частности электронную плотность и радиус. Устремив \autoref{an_sph_asymp1} к единице, можем определить выражение для резонансного коэффициента преломления, что в свою очередь дает выражение на резонансную электронную плотность, используя \autoref{eps_plasma}:

    \eq
        m^2 \left( x,\:n \right) = - \frac{8n^2 (n + 1) - 6n (x^2 + 1)}{2n x^2 (2n - 1)} \left[ 1 + \sqrt{ 1 - \frac{4n ( n - 3 + 4n^2 (n + 2)) (x^2 + 4n - 2)x^2}{8n^2 (n + 1) - 6n (x^2 + 1)} } \right]
        \label{m2_resonance}
    \qe

    \eq
        \frac{n_e}{n_c} = (1 - m^2) (1 + i \beta_e)
        \label{nenc_resonance}
    \qe

    \img[components/img/sph_base/nenc_123]{Резонансная электронная плотность в зависимости от радиуса. Кривые посчитаны при помощи \Autoref{m2_resonance, nenc_resonance}; $\beta_e = 0$.}{nenc_123:image}{0.66\textwidth}

В соответствии с \autoref{m2_resonance}, \autoref{nenc_123:image} резонанс рассеянного поля при $\lambda_{10} = \lambda_{L} / 10 = 83$ нм отвечает резонансной электронной плотности $n_{el} \approx 5.7 \cdot 10^{23}$ $\textrm{см}^{-3}$ для $ka = 0.7$ и $n_{el} \approx 3.9 \cdot 10^{23}$ $\textrm{см}^{-3}$ для $ka = 0.3$, со средними степенями ионизации $Z|_{ka = 0.7} \approx 9$ и $Z|_{ka = 0.3} \approx 6$.


