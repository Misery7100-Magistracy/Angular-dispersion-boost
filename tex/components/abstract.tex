\begin{abstract}

    Периодические поверхностные решетки и фотонные кристаллы являются отличными инструментами для дифракции и направления света. Однако этот метод менее эффективен в случае экстремального ультрафиолетового света из-за высокого поглощения любого материала в этом диапазоне частот. В работе исследуется возможность усиления угловой дисперсии излучения XUV-диапазона за счет рассеяния на подходящих сферических кластерах. В рамках работы была разработана аналитическая модель с использованием диэлектрической функции Друде плазмы и теории рассеяния Ми. Модель построена в квазистатическом приближении, так как время ионизации меньше длительности импульса, что значительно меньше времени разлета плазмы. В рамках модели мы используем предельные формы функций Бесселя, поскольку нас интересуют только частицы, размеры которых меньше длины волны падающего излучения. Оценены резонансные параметры мишени для десятой гармоники Ti:Sa лазер, найдено усиление рассеянного поля в резонансном случае по сравнению с лазерной гармоникой. Используя те же условия резонанса для одного кластера, мы моделируем дифракцию на массиве таких кластеров с помощью кода CELES. Полученные результаты показывают значительное усиление рассеянного поля в резонансном случае для больших углов и соответствуют теории дифракции Брэгга-Вульфа. При помощи particle-in-cell моделирования была подтвеждена квазистационарность плотности плазменных кластеров, что дает возможность значительно упростить расчет структур, подходящих для управления высокими гармониками лазерного излучения в XUV-диапазоне с помощью ионизированного кластерного газа.

\end{abstract}