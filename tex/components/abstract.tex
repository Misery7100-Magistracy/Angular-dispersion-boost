It is well known the interaction of intense coherent radiation with limited size targets can cause the excitation of surface plasmon oscillations. Absorption and scattering of the incident light in such case can be described good enough using the Mie theory, which predicts the multipole resonances corresponding to oscillations of a part of free electrons of the target with respect to positively charged ions. In this mode, effective excitation of surface plasmons can cause a significant increase
internal and external fields at eigenfrequencies, for example, gas os clusters, as well as the enhancement of scattered field at large angles with respect to the incident wave direction.

Mirrors and diffraction gratings can be used to manage spatial characteristics of the infrared and visible wavelength range. On the other side, crystals with regularly spaced scattering centers are suitable for X-ray radiation direction. However, there is a large gap between these wavelength ranges, hard or extreme ultraviolet radiation (XUV), corresponding to high-order laser harmonics in particular. XUV light turns out to be difficult to control due to high absorption of any material in this range.

\begin{figure}[H]
    \subimgtwo[components/img/celes/TE_10deg_check]{$\theta = 10^{\circ}$. Efficient scattering at angles $
    \theta_{\textrm{s}} = 0^{\circ}$, $-24^{\circ}$, $135^{\circ}$ with respect to the incidence direction.}{fig:a}{0.48\textwidth}
    \hfil
    \subimgtwo[components/img/celes/TE_15deg_check]{$\theta = 15^{\circ}$. Efficient scattering at angles $
    \theta_{\textrm{s}} = 0^{\circ}$, $-27^{\circ}$ with respect to the incidence direction.}{fig:b}{0.48\textwidth}
    \caption{10-th harmonic scattering by cubic lattice of clusters $12\times12\times12$. $\theta$ is the angle of lattice rotation around $y$ axis (equals to different incident light angles). $|\vectbf{E}{\textrm{s}}|$ plotted in the plane perpendicular to polarizaiton: $\vectbf{e}{\textrm{p}} = \vectbf{e}{y}$. Incident radiation represented by the gaussian beam with width $w = 300$ nm.}\label{fig:image}
\end{figure}

In this work we propose usage of a medium of spherical nano-clusters as a target for efficient directed scattering radiation in the XUV range. It was shown, with the linear Mie scattering theory, it is possible to estimate the resonance parameters of a single cluster and predict potential scattering directions for a system of many clusters.

Using a cubic lattice spatial configuration, options for diffraction control of 10-th titan-sapphire laser harmonic were calculated and the features of scattering with respect to the angle incidence of radiation were revealed (\autoref{fig:image}). The results show the correspondence of the Bragg-Wolfe diffraction theory for planar and spatial gratings, the ability to control high harmonics of laser radiation (XUV range) using an ionized cluster gas.
