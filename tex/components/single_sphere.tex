\section{Одиночный кластер}

В рамках теории рассеяния Ми известно, что амплитуду поля вблизи поверхности мишени можно значительно усилить. Для проверки этого было посчитаны значения $m $, отвечающие ранее рассмотренным условиям $\lambda = \lambda_{10}$, $ka = 0.5,\:0.7$, которые оказались равны $m_{0.5} = 1.635i$, $m_{0.7} = 1.851i$.

Были вычислены дальние и ближние результирующие электрические поля для этих двух случаев при $\lambda = \lambda_{L}$ и $\lambda = \lambda_{10}$ с целью сравнения профилей и амплитуды. Видно, что рассеяние первой гармоники в обоих случаях очень близко к рэлеевскому (\Autoref{1h_ka0.5:b, 1h_ka0.7:b}) - профиль плоской падающей волны практически не изменяется.

Совсем другая ситуация в случае $\lambda = \lambda_{10}$ -- профиль волны искажен в результате рассеяния и становится похож на расходящуюся сферическую волну (\Autoref{10h_ka0.5:b, 10h_ka0.7:b}). Амплитуда поля в окрестности рассеивающего кластера выше, чем при $\lambda = \lambda_{L}$ (примерно в 5 раз для обоих случаев) (\Autoref{10h_ka0.5:a, 10h_ka0.7:a}).

Случай $ka = 0.7$ был также сравнён с аналогичной ситуацией для одиночного наноцилиндра \cite{andreev_lecz} (\autoref{10h_ka0.7:c}). Видно, что картины поля похожи, в том числе и область локализованного поля в направлении рассеяния на угол $0^{\circ}$ относительно направления распространения плоской волны.

    \begin{figure}[H]
        \subimgtwo[components/img/mph/830nm_ka0.5_near]{Near-field.}{1h_ka0.5:a}{0.42\textwidth}
        \hfil
        \subimgtwo[components/img/mph/830nm_ka0.5_far]{Far-field.}{1h_ka0.5:b}{0.42\textwidth}
        \caption{Рассеяние лазерной гармоники. $\lambda = \lambda_{L}$, $a \approx 6.4$~nm ($ka = 0.5$); $|\vectbf{E}{}|$ построено в плоскости поляризации падающей волны.}
        \label{1h_ka0.5:image}
    \end{figure}

    \begin{figure}[H]
        \subimgtwo[components/img/mph/83nm_ka0.5_near_k_broken]{Near-field.}{10h_ka0.5:a}{0.42\textwidth}
        \hfil
        \subimgtwo[components/img/mph/83nm_ka0.5_far_k_broken]{Far-field.}{10h_ka0.5:b}{0.42\textwidth}
        \caption{Рассеяние $10$-ой гармоники. $\lambda = \lambda_{10}$, $a \approx 6.4$~nm ($ka = 0.5$); $|\vectbf{E}{}|$ построено в плоскости поляризации падающей волны.}
        \label{10h_ka0.5:image}
    \end{figure}

    \begin{figure}[H]
        \subimgtwo[components/img/mph/830nm_ka0.7_near]{Near-field.}{1h_ka0.7:a}{0.42\textwidth}
        \hfil
        \subimgtwo[components/img/mph/830nm_ka0.7_far]{Far-field.}{1h_ka0.7:b}{0.42\textwidth}
        \caption{Рассеяние лазерной гармоники. $\lambda = \lambda_{L}$, $a \approx 8.9$~nm ($ka = 0.7$); $|\vectbf{E}{}|$ построено в плоскости поляризации падающей волны.}
        \label{1h_ka0.7:image}
    \end{figure}

    \begin{figure}[H]
        \subimgtwo[components/img/mph/83nm_ka0.7_near_k_broken]{Ближнее поле.}{10h_ka0.7:a}{0.42\textwidth}
        \hfil
        \subimgtwo[components/img/mph/83nm_ka0.7_far_k_broken]{Дальнее поле.}{10h_ka0.7:b}{0.42\textwidth}
        \\
        \subimgtwo[components/img/external/oe-28_screen]{Рассеяние одиночным наноцилиндром \cite{andreev_lecz}.}{10h_ka0.7:c}{0.45\textwidth}
        \caption{Рассеяние $10$-ой гармоники. $\lambda = \lambda_{10}$, $a \approx 8.9$~nm ($ka = 0.7$); $|\vectbf{E}{}|$ построено в плоскости поляризации падающей волны. Качественное сравнение для такого же значения $ka$ в цилиндрических координатах (c) --- падающая волна распространяется слева направо (вдоль отрицательного направления оси $x$), $y$-поляризована.}
        \label{10h_ka0.7:image}
    \end{figure}