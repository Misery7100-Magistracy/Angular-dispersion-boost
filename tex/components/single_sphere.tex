\section{Single cluster}

Within the Mie theory, it is well-known that we can significantly enhance the field amplitude near the target. To check this, $m$ values corresponding previously considered conditions $\lambda = \lambda_{10}$, $ka = 0.5,\:0.7$ was caltulated: $m_{0.5} = 1.635i$, $m_{0.7} = 1.851i$.

Total near- and far-field waa calculated in two cases: for $\lambda = \lambda_{L}$ and $\lambda = \lambda_{10}$ --- to compare their amplitudes and scattering profiles. We can see, that the scattering of the laser harmonic (first harmonic) is very close to Rayleigh scattering (\Autoref{1h_ka0.5:b, 1h_ka0.7:b}) --- the incident plane wave profile almost does not change. Also scattering indicatrices in the plane of polarization correspond to the Rayleigh dependency~\cite{boren_huffman} (\Autoref{ka0.5_far_field:b, ka0.7_far_field:b}).

A completely different situation in the case of $\lambda = \lambda_{10}$ --- the incident plane wave profile is distorted as a result of scattering and becomes like a diverging spherical wave (\Autoref{10h_ka0.5:b, 10h_ka0.7:b}). The near-field amplitude is higher than for $\lambda = \lambda_{L}$ about 5 times for both radius cases (\Autoref{10h_ka0.5:a, 10h_ka0.7:a}). Also scattering indicatrices in the plane of polarization have larger amplitudes, which suggests more efficient far-field scattering (\Autoref{ka0.5_far_field:a, ka0.7_far_field:a}). We can see back-scattering enhancement on angles $\theta \approx 180^{\circ},\:120^{\circ},\:-240^{\circ}$ relative to the direction of the incident wave propagation.

The case $ka = 0.7$ compared with similar situation for scattering by a single nanocylinder~\cite{andreev_lecz} (\autoref{10h_ka0.7:image}c). We can see, that field distributions are similar include spherical outgoing far-field wave and localized near-field area in $0^{\circ}$ scattering direction relative to the direction of the incident wave propagation.

    \begin{figure}[H]
        \subimgtwo[components/img/mph/830nm_ka0.5_near]{Near-field.}{1h_ka0.5:a}{0.42\textwidth}
        \hfil
        \subimgtwo[components/img/mph/830nm_ka0.5_far]{Far-field.}{1h_ka0.5:b}{0.42\textwidth}
        \caption{Laser harmonic scattering. $\lambda = \lambda_{L}$, $a \approx 6.4$~nm ($ka = 0.5$); $|\vectbf{E}{}|$ plotted in the plane of polarization.}
        \label{1h_ka0.5:image}
    \end{figure}

    \begin{figure}[H]
        \subimgtwo[components/img/mph/83nm_ka0.5_near_k_broken]{Near-field.}{10h_ka0.5:a}{0.42\textwidth}
        \hfil
        \subimgtwo[components/img/mph/83nm_ka0.5_far_k_broken]{Far-field.}{10h_ka0.5:b}{0.42\textwidth}
        \caption{$10$-th harmonic scattering. $\lambda = \lambda_{10}$, $a \approx 6.4$~nm ($ka = 0.5$); $|\vectbf{E}{}|$ plotted in the plane of polarization.}
        \label{10h_ka0.5:image}
    \end{figure}

    \begin{figure}[H]
        \subimgtwo[components/img/mph/830nm_ka0.7_near]{Near-field.}{1h_ka0.7:a}{0.42\textwidth}
        \hfil
        \subimgtwo[components/img/mph/830nm_ka0.7_far]{Far-field.}{1h_ka0.7:b}{0.42\textwidth}
        \caption{Laser harmonic scattering. $\lambda = \lambda_{L}$, $a \approx 8.9$~nm ($ka = 0.7$); $|\vectbf{E}{}|$ plotted in the plane of polarization.}
        \label{1h_ka0.7:image}
    \end{figure}

    \begin{figure}[H]
        \subimgtwo[components/img/mph/83nm_ka0.7_near_k_broken]{Near-field.}{10h_ka0.7:a}{0.42\textwidth}
        \hfil
        \subimgtwo[components/img/mph/83nm_ka0.7_far_k_broken]{Far-field.}{10h_ka0.7:b}{0.42\textwidth}
        \\
        \subimgtwo[components/img/external/oe-28_screen]{Scattering by a single nanocylinder \cite{andreev_lecz}.}{10h_ka0.7:c}{0.45\textwidth}
        \caption{$10$-th harmonic scattering. $\lambda = \lambda_{10}$, $a \approx 8.9$~nm ($ka = 0.7$); $|\vectbf{E}{}|$ plotted in the plane of polarization. Qualitative comparison for the same $ka$ in cylindrical coordinates (c) --- the incident wave propagates from right to left (along negative $x$ axis direction), polarization is along $y$ axis.}
        \label{10h_ka0.7:image}
    \end{figure}

    \begin{figure}[H]
        \subimgtwo[components/img/mph/ka0.5_far_field_83]{$\lambda = \lambda_{10}$.}{ka0.5_far_field:a}{0.4\textwidth}
        \hfil
        \subimgtwo[components/img/mph/ka0.5_far_field_830]{$\lambda = \lambda_{L}$.}{ka0.5_far_field:b}{0.4\textwidth}
        \caption{Single cluster scattering indicatrices. $a \approx 6.4$~nm.}
        \label{ka0.5_far_field:image}
    \end{figure}

    \begin{figure}[H]
        \subimgtwo[components/img/mph/ka0.7_far_field_83]{$\lambda = \lambda_{10}$.}{ka0.7_far_field:a}{0.4\textwidth}
        \hfil
        \subimgtwo[components/img/mph/ka0.7_far_field_830]{$\lambda = \lambda_{L}$.}{ka0.7_far_field:b}{0.4\textwidth}
        \caption{Single cluster scattering indicatrices. $a \approx 8.9$~nm.}
        \label{ka0.7_far_field:image}
    \end{figure}