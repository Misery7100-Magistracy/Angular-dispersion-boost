\section{Одиночный кластер}

В рамках теории рассеяния Ми известно, что амплитуду поля вблизи поверхности мишени можно значительно усилить. Для проверки этого были вычислены значения комплексного коэффициента преломления $m$, отвечающие ранее полученным условиям на резонансную электронную плотность при $\lambda = \lambda_L / 10 = \lambda_{10} = 83$, $ka = 0.7$ -- $m_{0.7} = 1.851i$.

Были посчитано результирующее электрическое поле для этого случая при $\lambda = \lambda_{L}$ и $\lambda = \lambda_{10}$ с целью сравнения профилей и амплитуды. Видно, что рассеяние первой гармоники очень близко к рэлеевскому (\Autoref{1h_ka0.7:b}) - профиль плоской падающей волны практически не изменяется. Совсем другая ситуация в случае $\lambda = \lambda_{10}$ -- профиль волны искажен в результате рассеяния и становится похож на расходящуюся сферическую волну (\Autoref{10h_ka0.7:b}). Амплитуда поля в окрестности рассеивающего кластера выше, чем при $\lambda = \lambda_{L}$ (примерно в 5 раз) (\Autoref{10h_ka0.7:a}).

    % \begin{figure}[H]
    %     \subimgtwo[components/img/mph/830nm_ka0.5_near]{Near-field.}{1h_ka0.5:a}{0.42\textwidth}
    %     \hfil
    %     \subimgtwo[components/img/mph/830nm_ka0.5_far]{Far-field.}{1h_ka0.5:b}{0.42\textwidth}
    %     \caption{Рассеяние лазерной гармоники. $\lambda = \lambda_{L}$, $a \approx 6.4$~nm ($ka = 0.5$); $|\vectbf{E}{}|$ построено в плоскости поляризации падающей волны.}
    %     \label{1h_ka0.5:image}
    % \end{figure}

    % \begin{figure}[H]
    %     \subimgtwo[components/img/mph/83nm_ka0.5_near_k_broken]{Near-field.}{10h_ka0.5:a}{0.42\textwidth}
    %     \hfil
    %     \subimgtwo[components/img/mph/83nm_ka0.5_far_k_broken]{Far-field.}{10h_ka0.5:b}{0.42\textwidth}
    %     \caption{Рассеяние $10$-ой гармоники. $\lambda = \lambda_{10}$, $a \approx 6.4$~nm ($ka = 0.5$); $|\vectbf{E}{}|$ построено в плоскости поляризации падающей волны.}
    %     \label{10h_ka0.5:image}
    % \end{figure}

    \begin{figure}[H]
        \subimgtwo[components/img/mph/830nm_ka0.7_near]{Поле вблизи кластера.}{1h_ka0.7:a}{0.37\textwidth}
        \hfil
        \subimgtwo[components/img/mph/830nm_ka0.7_far]{Поле вдали кластера (общий вид).}{1h_ka0.7:b}{0.37\textwidth}
        \caption{Рассеяние лазерной гармоники. $\lambda = \lambda_{L} = 830$ нм, $a \approx 8.9$~нм ($ka = 0.7$); $|\vectbf{E}{}|$ построено в плоскости поляризации падающей волны.}
        \label{1h_ka0.7:image}
    \end{figure}

    \begin{figure}[H]
        \subimgtwo[components/img/mph/83nm_ka0.7_near_k_broken]{Поле вблизи кластера.}{10h_ka0.7:a}{0.37\textwidth}
        \hfil
        \subimgtwo[components/img/mph/83nm_ka0.7_far_k_broken]{Поле вдали кластера (общий вид).}{10h_ka0.7:b}{0.37\textwidth}
        \\
        \subimgtwo[components/img/external/oe-28_screen]{Рассеяние наноцилиндром \cite{andreev_lecz}.}{10h_ka0.7:c}{0.4\textwidth}
        \caption{Рассеяние $10$-ой гармоники. $\lambda = \lambda_{10} = 83$ нм, $a \approx 8.9$~нм ($ka = 0.7$); $|\vectbf{E}{}|$ построено в плоскости поляризации падающей волны. Качественное сравнение для такого же значения $ka$ в цилиндрических координатах (c) --- падающая волна распространяется справа налево (противоположно направлению оси $x$), $y$-поляризована.}
        \label{10h_ka0.7:image}
    \end{figure}

Случай $ka = 0.7$ был также сравнён с аналогичной ситуацией для одиночного наноцилиндра \cite{andreev_lecz} (\autoref{10h_ka0.7:c}). Видно, что картины поля похожи, в том числе и область локализованного поля в направлении рассеяния на угол $0^{\circ}$ относительно направления распространения плоской волны.