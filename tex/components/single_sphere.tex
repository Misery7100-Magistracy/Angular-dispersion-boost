\section{Стационарные вычисления}

\subsection{Одиночный кластер}

В рамках теории рассеяния Ми известно, что амплитуду поля вблизи поверхности мишени можно значительно усилить. Для проверки этого были вычислены значения комплексного коэффициента преломления $m$, отвечающие ранее полученным условиям на резонансную электронную плотность $n_{el}$ при $\lambda_{10} = 83$ nm, $ka = 0.7$: $m = 1.851i$ (\autoref{m2_resonance}). Комплексный коэффициент преломления чисто мнимый, так как столкновительный коэффициент $v_e$ в рассматриваемом случае несколько ниже частоты гармоники, поэтому взаимодействие можно считать бесстолкновительным~\cite{andreev_lecz}.

Были посчитано результирующее электрическое поле для этого случая при $\lambda = \lambda_{L}$ и $\lambda = \lambda_{10}$ с целью сравнения профилей и амплитуды. Видно, что рассеяние первой гармоники очень близко к рэлеевскому (\Autoref{1h_ka0.7:b}) --- профиль плоской падающей волны практически не изменяется. Совсем другая ситуация в случае $\lambda = \lambda_{10}$ --- профиль волны искажен в результате рассеяния и становится похож на расходящуюся сферическую волну (\Autoref{10h_ka0.7:b}). Амплитуда поля в окрестности рассеивающего кластера выше, чем при $\lambda = \lambda_{L}$ (примерно в 5 раз) (\Autoref{10h_ka0.7:a}).

    \begin{figure}[htbp]
        \subimgtwo[components/img/mph/830nm_ka0.7_near]{Поле вблизи кластера.}{1h_ka0.7:a}{0.37\textwidth}
        \hfil
        \subimgtwo[components/img/mph/830nm_ka0.7_far]{Поле вдали кластера.}{1h_ka0.7:b}{0.37\textwidth}
        \caption{Рассеяние лазерной гармоники. $\lambda = \lambda_{L} = 830$ нм, $a \approx 8.9$~нм ($ka = 0.7$); $|\vectbf{E}{}|$ построено в плоскости поляризации падающей волны.}\label{1h_ka0.7:image}
    \end{figure}

    \begin{figure}[htbp]
        \subimgtwo[components/img/mph/83nm_ka0.7_near_k_broken]{Поле вблизи кластера.}{10h_ka0.7:a}{0.34\textwidth}
        \hfil
        \subimgtwo[components/img/mph/83nm_ka0.7_far_k_broken]{Поле вдали кластера.}{10h_ka0.7:b}{0.34\textwidth}
        \\
        \subimgtwo[components/img/external/oe-28_screen]{Рассеяние наноцилиндром~\cite{andreev_lecz}.}{10h_ka0.7:c}{0.37\textwidth}
        \caption{Рассеяние $10$-ой гармоники. $\lambda = \lambda_{10} = 83$ нм, $a \approx 8.9$~нм ($ka = 0.7$); $|\vectbf{E}{}|$ построено в плоскости поляризации падающей волны. Качественное сравнение для такого же значения $ka$ в цилиндрических координатах (c) --- падающая волна распространяется справа налево (противоположно направлению оси $x$), $y$-поляризована.}\label{10h_ka0.7:image}
    \end{figure}

Случай $ka = 0.7$ был также сравнён с аналогичной ситуацией для одиночного наноцилиндра~\cite{andreev_lecz} (\autoref{10h_ka0.7:c}). Видно, что картины поля похожи, в том числе и область локализованного поля в направлении рассеяния на угол $0^{\circ}$ относительно направления распространения плоской волны.