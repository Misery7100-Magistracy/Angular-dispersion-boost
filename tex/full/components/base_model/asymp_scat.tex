\subsection{Асимптотическое приближение коэффициентов рассеянного поля}

В случае сферической симметрии амплитуда рассеянного поля максимальна для $m^2 = - (n+ 1) / n$ при $ka \ll 1$, что даёт соответствующий набор резонансных плотностей в бесстолкновительном случае $n_e = n_c(2n + 1) / n$. Это можно получить, используя нулевое асимптотическое приближение функций Бесселя~\cite{boren_huffman}, в результате чего коэффициенты (\Autoref{an_bessel, bn_bessel}) значительно упрощаются:

    \begin{equation}
        a_n\left( x \to 0,\:m \right) = {\left( 1 + i \frac{m^2n + n + 1 }{n \left( m^2 - 1 \right)} \: \frac{C_n}{x^{2n + 1}}\right)}^{-1}, \qquad b_n\left( x \to 0,\:m \right) = 0
        \label{ab_asymp}
    \end{equation}
    \begin{align*}
		C_n = \cfrac{2^{2n + 1}\Gamma \left(2n \right) \Gamma \left(2n + 2\right)}{\Gamma \left(n + 1\right) \Gamma \left(n + 2\right)}
    \end{align*}

Такое приближение можно использовать вместо \Autoref{an_bessel, bn_bessel} для объектов достаточно маленького радиуса, но уже при $ka \sim 1$ оно перестаёт быть разумным, особенно для больших $n$. Вместо него в таком случае лучше подходит аппроксимация первого порядка, полученная с помощью первого приближения функций Бесселя~\cite{boren_huffman}:

    \begin{equation}
		a_n\left( x ,\:m \right) = {\left( 1 + i \frac{ 2(m^2 n + n + 1) (4n (1 + n) - 3) - (m^2 - 1)(3 + n(5 + 2n + m^2 (2n - 1))) x^2}{ (m^2 - 1)(2n + 3)(n + 1)(4n + 6 - (m^2 + 1)x^2)} \: \frac{C_n}{x^{2n + 1}}\right) }^{-1}
		\label{an_sph_asymp1}
    \end{equation}
	\begin{align*}
		C_n = \frac{2^{2n + 1}}{\pi}\:\Gamma \left(n - \frac{1}{2} \right) \Gamma \left(n + \frac{5}{2} \right)
    \end{align*}

На \autoref{ab_asymp:image} показана зависимость коэффициента рассеяния от электронной плотности для двух различных значений радиуса в рамках нулевого асимптотического приближения и сравнение первого и нулевого приближений. Видно, что с ростом $n$ ширина резонансного пика быстро уменьшается, а также б\'{о}льшим радиусам (безразмерным) $ka$ соответствует б\'{о}льшая их ширина. Помимо этого, с ростом радиуса растет и значение резонансной плотности, что видно на \autoref{nenc_123:image}.

    \begin{figure}[htb]
        \subimgtwo[../img/sph_base/sph_ka0.5_123]{$ka = 0.5$.}{ab_asymp:a}{0.45\textwidth}
        \hfil
        \subimgtwo[../img/sph_base/sph_ka1.5_123]{$ka = 1.5$.}{ab_asymp:b}{0.45\textwidth}
        \subimgtwo[../img/sph_base/sph_ka1.5_123_1st]{$ka = 1.5$ в приближении первого порядка.}{ab_asymp:b}{0.45\textwidth}
		\caption{Коэффициенты сферических гармоник в приближении нулевого и первого порядка, $\beta_e = 0$. Кривые ``exact'' построены с использованием полных разложений.}\label{ab_asymp:image}
	\end{figure}

    % \img[components/img/sph_base/sph_ka1.5_123_1st]{$ka = 1.5$ в приближении первого порядка. $\beta_e = 0$. Кривые ``exact'' построены с использованием полных разложений.}{ab1:image}{0.45\textwidth}

Такие аппроксимации позволяют оценить резонансные параметры кластера, в частности электронную плотность и радиус. Устремив \autoref{an_sph_asymp1} к единице, можем определить выражение для квадрата резонансного коэффициента преломления, что в свою очередь дает выражение на резонансную электронную плотность в критических единицах при помощи \autoref{eps_plasma}:

    \begin{equation}
        m^2 \left(x,\:n \right) = \frac{8n^2 (n + 1) - (6n + 3)x^2 - 6n}{2n x^2 (2n-1)} \left[ 1 + \sqrt{ 1 - \frac{4n (n-3 + 4n^2 (n + 2)) (x^2 + 4n-2) x^2}{{\left(8n^2 (n + 1) - (6n + 3)x^2 - 6n \right)}^{2}} } \right]
        \label{m2_resonance}
    \end{equation}

    \begin{equation}
        \frac{n_e}{n_c} = (1 - m^2) (1 + i \beta_e)
        \label{nenc_resonance}
    \end{equation}

    \img[../img/sph_base/nenc_123]{Резонансная электронная плотность в зависимости от радиуса. Кривые посчитаны при помощи \Autoref{m2_resonance}; $\beta_e = 0$.}{nenc_123:image}{0.45\textwidth}

В соответствии с \Autoref{m2_resonance, nenc_resonance} резонанс рассеянного поля при $\lambda_{10} = \lambda_{L} / 10 = 83$ нм отвечает резонансной электронной плотности $n_{el} \approx 5.7 \cdot 10^{23}$ $\textrm{cm}^{-3}$ для $ka = 0.7$ и $n_{el} \approx 3.9 \cdot 10^{23}$ $\textrm{cm}^{-3}$ для $ka = 0.3$, со средними степенями ионизации $Z|_{ka = 0.7} \approx 9$ и $Z|_{ka = 0.3} \approx 6$.