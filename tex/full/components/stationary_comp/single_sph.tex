\subsection{Одиночный кластер}

В рамках теории рассеяния Ми известно, что амплитуду поля вблизи поверхности мишени можно значительно усилить. Для проверки этого были вычислены значения комплексного коэффициента преломления $m$, отвечающие ранее полученным условиям на резонансную электронную плотность $n_{el}$ при $\lambda_{10} = 83$ nm, $ka = 0.7$: $m = 1.851i$ (\autoref{m2_resonance}). Комплексный коэффициент преломления чисто мнимый, так как столкновительный коэффициент $v_e$ в рассматриваемом случае несколько ниже частоты гармоники, поэтому взаимодействие можно считать бесстолкновительным~\cite{andreev_lecz}.

Для этого случая было посчитано рассеянное электрическое поле (\autoref{E_s_sph}) при $\lambda = \lambda_{L}$ и $\lambda = \lambda_{10}$ с целью сравнения между собой резонансного и нерезонансного случая. Видно, что в резонансном случае (\autoref{ka0.7:b}) рассеянное поле представляет собой расходящуюся сферическую волну, амплитуда поля в окрестности кластера значительно выше, чем в отсутствии резонанса (\autoref{ka0.7:a}), где рассеянных волн как таковых практически не наблюдается, что говорит о том, что падающая волна в нерезонансном случае практически не взаимодействует с кластером.

    \begin{figure}[H]
        \subimgtwo[../img/mph_new/es_ka0.7_1harm]{$\lambda = \lambda_{L} = 830$ nm.}{ka0.7:a}{0.4\textwidth}
        \hfil
        \subimgtwo[../img/mph_new/es_ka0.7_10harm]{$\lambda = \lambda_{10} = 83$ nm.}{ka0.7:b}{0.4\textwidth}
        \caption{$ka = 0.7$ ($a \approx 8.9$ nm); $|\vectbf{E}{s}|^2$ в плоскости поляризации падающей волны.}\label{ka0.7:image}
    \end{figure}

    \begin{figure}[H]
        \subimgtwo[../img/mph_new/es_ka1.7_10harm]{Рассеяние кластером.}{10h_ka0.7:a}{0.4\textwidth}
        \hfil
        \subimgtwo[../img/external/oe-28_screen_single.jpg]{Рассеяние наноцилиндром~\cite{andreev_lecz}.}{10h_ka0.7:b}{0.39\textwidth}
        \caption{$ka = 1.7$ ($a \approx 22.5$ nm), $\lambda = \lambda_{10} = 83$ нм; $|\vectbf{E}{s}|^2$ построено в плоскости поляризации падающей волны. Качественное сравнение для такого же значения $ka$ в случае цилиндров (б) --- падающая волна распространяется справа налево (противоположно направлению оси $x$), $y$-поляризована.}\label{10h_ka0.7:image}
    \end{figure}

Дополительно был смоделирован случай $ka = 1.7$ (\autoref{10h_ka0.7:a}) и сравнён с аналогичной ситуацией для одиночного наноцилиндра~\cite{andreev_lecz} (\autoref{10h_ka0.7:b}). Видно, что общие направления рассеянного поля сохраняются, видны слабые боковые порядки с углами отклонения, близкими к $90^\circ$ относительно направления падающей волны, что сходно с случаем цилиндрической симметрии. Различия в амплитуде рассеянных волн связаны с принципиальными отличиями в геометрии цилиндра и кластера. Наиболее интенсивное рассеяние наблюдается для направления, соответствующего направлению падающей волны в силу конструктивной интерференции, эффективность рассеяния в этом направлении порядка 5\%.