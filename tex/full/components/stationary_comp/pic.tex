\subsection{Оправдание стационарной модели}

В общем случае расчет взаимодействия высокоинтенсивного импульса лазерного излучения с группой плотных сферических кластеров, расположенных в трехмерном пространстве, требует длительных и сложных нестационарных вычислений ввиду того, что распределение электронной плотности кластеров в результате взаимодействия с лазерным импульсом изменяется с течением времени.

Для проверки масштабов изменения электронной плотности в рассматриваемом случае было проведено моделирование эволюции распределения электронной плотности в одномерном пространстве отдельного кластера. Для моделирования был взят код LPIC++~\cite{Pfund1998}.

В качестве источника был задан фронтальный линейно поляризованный лазерный импульс с длиной волны $\lambda_{10} = 83$ nm, длительностью $\tau$, формой огибающей амплитуды импульса во времени $\sin^2{(t)}$ и интенсивностью $I_{h} = 10^{14}\:\rm{W}/\rm{cm}^2$. Период лазерного излучения, соответствующий лазерной гармонике, равен $T = \lambda_{L}\:/\:c \approx 2.8$ fs, поэтому длина импульса в моделировании была взята $\tau = 10T = 28$ fs, время моделирования $t = 20T = 56$ fs. Плазма представлена 2000 частицами в каждой ячейке, занятой мишенью, расположенной в центре бокса шириной $w_{box} \approx 3\lambda_{10}$; электронная плотность мишени в критических единицах равна $n_{el} = 4.2 n_c$. Также было рассмотрено взаимодействие с первой гармоникой, для которой $\lambda_L = 830$ nm, $I_L = 10^{18}\:\rm{W}/\rm{cm}^2$. В качестве мишеней были взяты одиночные кластеры радиуса $a$ от 9 до 60 nm.

% Относительная амплитуда импульса $\alpha_{h}$ равна:

%     \begin{align}
%         I_{10} \lambda_{10}^2 = \alpha_{h}^2 \times 1.37 \cdot 10^{18}\:\rm{W}\cdot\rm{\upmu m}^2/\rm{cm}^2
%     \end{align}
%     \begin{equation*}
%         \alpha_{h} = \frac{\lambda_{10}}{\sqrt{1.37} \cdot 1\:\rm{\upmu m}} \approx 7.1 \cdot 10^{-4}
%     \end{equation*}


    \begin{figure}[H]
        \subimgtwo[../img/lpic/10st_harm_20nm_rad]{$a = 20$ nm, $\lambda = \lambda_{10}$, $I = I_h$.}{lpic_low_high:a}{0.42\textwidth}
        \hfil
        \subimgtwo[../img/lpic/1st_harm_40nm_rad]{$a = 40$ nm, $\lambda = \lambda_{L}$, $I = I_L$.}{lpic_low_high:b}{0.42\textwidth}
        \caption{Взаимодействие одномерной мишени с 1-ой и 10-ой гармониками, $\lambda_{1} = \lambda_{L} = 830$ nm, $\lambda_{10} = 83$ nm.}\label{lpic_low_high:image}
    \end{figure}

    \img[../img/lpic/htr_over_2a_a]{Асимптотика средней суммарной толщины переходного слоя при $0 \leq t \leq 10T$ относительно радиуса мишени. $n_c$, использованное при построении, соответствует критической плотности для длины волны $\lambda = \lambda_{10}$.}{lpic_htr:image}{0.55\textwidth}

По полученным результатам моделирования была расчитана средняя суммарная толщина переходного слоя в процессе взаимодействия со внешним импульсом $h_{tr}$ в зависимости от радиуса мишени $a$ (\autoref{lpic_htr:image}). Условие квазистационарности в таком случае принимает вид $h_{tr} \ll 2a$, что соблюдается при $a \geq 20$ nm. Таким образом, при использовании стационарных вычислений для оценки углового рассеяния допустимо использовать только кластеры с радиусом 20 nm и больше.
