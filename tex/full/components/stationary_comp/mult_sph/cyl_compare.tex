\subsubsection{Сравнение с цилиндрической симметрией}

Было проведено сравнение результатов моделирования рассеяния на массиве цилиндров и массиве сферических кластеров с аналогичными начальными параметрами: радиус рассеивателей, пространственное расстояние между ними, параметры падающего поля. Такое сравнение было проведено для двух случаев --- одного ряда рассеивателей и трёх рядов рассеивателей, длина волны падающего поля $\lambda \approx 89$ nm, радиус рассеивателей $a = 30$ nm, углы поворота $\varphi_0 = 0^\circ$, $\theta_0 = 0^\circ$, расстояние между рассеивателями $d = 3\lambda$.

По полученным результатам можно заметить общие сходства в локализованном поле вблизи пересечения падающего пучка и массива, при этом для \Autoref{cyl_compare:a, cyl_compare:b} различия меньше, чем для \Autoref{cyl_compare:c, cyl_compare:d} --- в отличие от \autoref{cyl_compare:d} на \autoref{cyl_compare:c} можно наблюдать, что интенсивность дифрагировавшего поля значительно выше, чем у отраженного. За счет более крупного масштаба построения на \Autoref{cyl_compare:a, cyl_compare:c} видны четкие направления дифракции, в частности два наиболее выделяющихся это прошедшее излучение ($h^\prime = 0,\:k^\prime = 0,\:l^\prime = 0$ по \autoref{bragg_wolf_order_spherical}), а также направление $\Delta \varphi = 180^\circ$, $\Delta \theta = 30^\circ$ ($h^\prime = -1,\:k^\prime = 0,\:l^\prime = -1$ по \autoref{bragg_wolf_order_spherical}).

\begin{figure}[H]
    \subimgtwo[../img/celes/plane_flat_to_compare.pdf]{Рассеяние однорядным массивом сфер.}{cyl_compare:a}{0.42\textwidth}
    \hfil
    \subimgtwo[../img/external/oe-28_screen_mult2.png]{Рассеяние однорядным массивом цилиндров~\cite{andreev_lecz}.}{cyl_compare:b}{0.375\textwidth}
    \\
    \subimgtwo[../img/celes/3plane_flat_to_compare.pdf]{Рассеяние трёхрядным массивом сфер.}{cyl_compare:c}{0.42\textwidth}
    \hfil
    \subimgtwo[../img/external/oe-28_screen_mult_3planes.jpg]{Рассеяние трёхрядным массивом цилиндров~\cite{andreev_lecz}.}{cyl_compare:d}{0.375\textwidth}
    \caption{Рассеяние гармоники с длиной волны $\lambda \approx 89$ nm массивом сфер (а) и массивом цилиндров (б), $\varphi_0 = 0^\circ$, $\theta_0 = 30^\circ$, радиус сферических кластеров и цилиндров $a = 30$ nm, расстояние между кластерами и цилиндрами $d = 3\lambda$, падающее поле направлено из левого нижнего угла в правый верхний под углом $\theta_0$ относительно нормали к мишени.}\label{cyl_compare:image}
\end{figure}

% plane_flat_to_compare