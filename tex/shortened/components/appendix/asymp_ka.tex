\subsection*{A\quadНулевое и первое приближение коэффициентов рассеянного поля}

\begin{figure}[H]
    \subimgtwo[../img/sph_base/sph_ka0.5_123]{$ka = 0.5$.}{ab_asymp:a}{0.45\textwidth}
    \hfil
    \subimgtwo[../img/sph_base/sph_ka1.5_123]{$ka = 1.5$.}{ab_asymp:b}{0.45\textwidth}
    \subimgtwo[../img/sph_base/sph_ka1.5_123_1st]{$ka = 1.5$ в приближении первого порядка.}{ab_asymp:b}{0.45\textwidth}
    \caption{Коэффициенты сферических гармоник в приближении нулевого и первого порядка, $\beta_e = 0$. Кривые ``exact'' построены с использованием полных разложений функций Бесселя и Ханкеля в ряд.}\label{ab_asymp:image}
\end{figure}

На \autoref{ab_asymp:image} показана зависимость коэффициента рассеяния от электронной плотности для двух различных значений радиуса в рамках нулевого асимптотического приближения и сравнение первого и нулевого приближений. Видно, что с ростом $n$ ширина резонансного пика быстро уменьшается, а также б\'{о}льшим радиусам (безразмерным) $ka$ соответствует б\'{о}льшая их ширина. Помимо этого, с ростом радиуса растет и значение резонансной электронной плотности.