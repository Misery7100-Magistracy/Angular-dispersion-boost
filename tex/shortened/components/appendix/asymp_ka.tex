\subsection*{A\quadАсимптотическое приближение коэффициентов рассеянного поля и приближение разложением в ряд}

\begin{figure}[H]
    \subimgtwo[../img/sph_base/sph_ka1.5_123]{Асимптотическое приближение.}{ab_asymp:b}{0.45\textwidth}
    \hfil
    \subimgtwo[../img/sph_base/sph_ka1.5_123_1st]{Разложение в ряд до первого члена.}{ab_asymp:b}{0.45\textwidth}
    \caption{Коэффициенты сферических гармоник в асимптотическом приближении и приближении разложением функций Ханкеля и Бесселя в ряд при $\beta_e = 0$, $\chi = 1.5$. Кривые ``exact'' соответствуют точным значениям коэффициентов рассеянного поля.}\label{ab_asymp:image}
\end{figure}

На \autoref{ab_asymp:image} представлена зависимость коэффициента рассеяния от электронной плотности для $\chi = 1.5$ в рамках асимптотического приближения и приближения разложением в ряд функций Ханкеля и Бесселя. Чётко видно значительное расхождение формы и положения кривых для разных $n$ при использовании асимптотического приближения, при этом приближение разложением в ряд обладает лишь неточностью в виде сдвига, который с ростом порядка $n$ стремится к нулю.