\section{Введение}

% Мишени конечного размера, взаимодействующие с высокоинтенсивным когерентным, излучением представляют собой хорошо изученное явление линейно возбужденных 
% поверхностных плазмонных колебаний. Поглощение и рассеяние падающего света в таком случае с хорошей точностью могут быть описаны при помощи теории Ми, которая предсказывает существование резонанса, соответствующего мультипольным колебаниям части свободных электронов мишени относительно положительно заряженных ионов. В режиме резонанса эффективное возбуждение поверхностных плазмонов может привести к значительному усилению внутреннего и внешнего поля на собственной частоте кластера. Что может привести к усилению поля, рассеянного на большие углы относительно исходного направления падающей волны.

% В пределах длин волн порядка микрометра могут быть использованы фотонные кристаллы и решетки для направления или дифракции электромагнитных волн~\cite{lin_zhang}, в то время как для подобных манипуляций с рентгеновским излучением могут быть использованы кристаллы с атомами, регулярно расположенными на расстоянии нескольких нанометров, в качестве рассеивающих центров~\cite{batterman_cole}. При этом большой промежуток между этими диапазонами длин волн, называющийся XUV (extreme-ultraviolet) или жесткий ультрафиолет, оказывается трудно манипулируемым.

% В данной работе предлагается использование массивов сферических нанокластеров для направленного рассеяния жесткого ультрафиолетового излучения. Аналогичная задача в случае цилиндрической симметрии --- массивов наноцилиндров в качестве рассеивателей --- была исследована ранее~\cite{andreev_lecz}. Полученные результаты показали эффектив-\\ность подхода, что делает рассмотрение сферической конфигурации многообещающим. Конечно, использование цилиндров более удобно с точки зрения контроля радиусов одиночных рассеивателей и дистанций между ними, но массивы сферических кластеров могут позволить оперировать направлением излучениея в трехмерном пространстве, а также могут быть собраны в более оптимальную пространственную конфигурацию, нежели цилиндры.

Как известно~\cite{lin_zhang, batterman_cole}, периодические поверхностные решетки и фотонные кристаллы являются эффективными инструментами для дифракции и направления света. Однако этот метод менее эффективен в случае экстремального ультрафиолетового света из-за высокого поглощения любого материала в этом диапазоне частот. Взаимодействие интенсивного когерентного излучения с мишенями конечного размера представляет собой достаточно изученное явление, в том числе с учетом линейно возбуждаемых, поверхностных плазмонных колебаний. Поглощение и рассеяние падающего света в таком случае с хорошей точностью могут быть описаны при помощи теории Ми, которая предсказывает существование резонанса, соответствующего мультипольным колебаниям части свободных электронов мишени относительно положительно заряженных ионов. В режиме резонанса эффективное возбуждение поверхностных плазмонов может привести к значительному усилению поля кластера, а также поля, рассеянного на большие углы относительно исходного направления падающей волны. 

Для пространственного управления электромагнитными волнами видимого и инфракрасного диапазонов могут быть использованы фотонные кристаллы и дифракционные решетки~\cite{lin_zhang}, в то время как для подобных манипуляций с рентгеновским излучением могут быть использованы кристаллы с атомами, регулярно расположенными на расстоянии нескольких нанометров, в качестве рассеивающих центров~\cite{batterman_cole}. При этом значительный промежуток между этими диапазонами длин волн, называющийся диапазоном жесткого ультрафиолета (EUV, extreme ultra-violet range), оказывается трудно манипулируемым. В данной работе предлагается использование массивов сферических нано-кластеров для направленного рассеяния жесткого ультрафиолетового излучения. 

Аналогичная задача в случае цилиндрической симметрии для массивов нано-цилиндров в качестве рассеивателей была исследована ранее~\cite{andreev_lecz}. Полученные результаты показали эффективность подхода, что делает рассмотрение сферической конфигурации многообещающим. Однако, изготовление набора цилиндров с оптимальными радиусами, длинами и дистанциями между ними достаточно сложная технологическая задача, тогда как массивы сферических кластеров формируемых из газовой струи достаточно просто сформировать~\cite{wftwfwqf} и они могут позволить оперировать направлением излучением в трехмерном пространстве, а также могут быть собраны в более оптимальную пространственную конфигурацию, нежели цилиндры.

% Обобщенная схема взаимодействия приведена на \autoref{plasma_area1:image}. Гармоники, которые содержит основной импульс, обладают различной интенсивностью под различными углами, что приводит к угловой зависимости формы выходного излучения.

% \img[components/img/plasma_area2]{Схема взаимодействия. Плоскость поляризации параллельна одной из граней кубической области. Размеры сферических кластеров порядка единиц нанометров, расстояния между ними не менее сотни нанометров. Распределение кластеров внутри кубической области в общем случае произвольно, кластеры не пересекают грани области.}{plasma_area1:image}{0.8\textwidth}
