
Особенность коэффициентов рассеянного поля (равенство знаменателя нулю) даёт набор резонансных электронных плотностей и соответствующих им комплексных коэффициентов рассеяния. Для $\chi \ll 1$ при помощи асимптотического приближения~\cite{andreev_lecz} можно получить точное аналитическое выражение для коэффициентов $a_n$, резонансный набор при этом принимает вид $m^2 = - (n + 1)\:/\:n$, $n_e = n_c\:(2n + 1)\:/\:n$ для любого натурального $n$. Для $\chi \sim 1$ асимптотическое приближение уже не подходит, так как вклад безразмерного радиуса $\chi$ в положение резонанса перестает быть пренебрежимо мал и необходим его учет в соответствующей зависимости (Приложение A). В таком случае необходимо использовать разложения в ряд функций Ханкеля и Бесселя~\cite{boren_huffman}.

В результате мы можем оценить резонансные параметры кластера, в частности электронную плотность и радиус. Резонансное значение $a_n$ соответствует нулю знаменателя соответствующего выражения, что отвечает $|a_n| = 1$, откуда можем переписать выражение для квадрата резонансного коэффициента преломления, что в свою очередь дает выражение на резонансную электронную плотность в критических единицах при помощи (\ref{eps_plasma}):

    \begin{equation}
        m^2 \left(\chi,\:n \right) = \frac{8n^2 (n + 1) - (6n + 3)\chi^2 - 6n}{2n \chi^2 (2n-1)} \left[ 1 + \sqrt{ 1 - \frac{4n (n-3 + 4n^2 (n + 2)) (\chi^2 + 4n-2) \chi^2}{{\left(8n^2 (n + 1) - (6n + 3)\chi^2 - 6n \right)}^{2}} } \right]
        \label{m2_resonance}
    \end{equation}

    \begin{equation}
        \frac{n_e}{n_c} = (1 - m^2) (1 + i \beta_e)
        \label{nenc_resonance}
    \end{equation}

    % \img[../img/sph_base/nenc_123]{Резонансная электронная плотность в зависимости от радиуса. Кривые посчитаны при помощи \Autoref{m2_resonance}; $\beta_e = 0$.}{nenc_123:image}{0.45\textwidth}

В соответствии с (\ref{m2_resonance}, \ref{nenc_resonance}) резонанс рассеянного поля при $\lambda_{10} = 83$ нм отвечает резонансной электронной плотности $n_{el} \approx 5.7 \cdot 10^{23}$ $\textrm{cm}^{-3}$ для $ka = 0.7$ и $n_{el} \approx 3.9 \cdot 10^{23}$ $\textrm{cm}^{-3}$ для $ka = 0.3$.