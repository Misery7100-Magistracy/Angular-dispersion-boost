Рассмотрим взаимодействие интенсивного короткого лазерного импульса длительностью $\tau = 28$ fs, интенсивностью $I_{h} = 1.2 \cdot 10^{14}$ W/cm$^2$ в результате конверсии с коэффициентом $10^{-4}$ с одиночным сферическим твёрдотельным кластером радиуса $a$. В результате взаимодействия из твердотельного кластера возникает плазменный, представление которого в рамках диэлектрической функции плазмы даёт Модель Друде:

    \begin{equation}
		\varepsilon (\omega_{h}) = 1 - {\left( \frac{\omega_{pe}}{\omega_{h}} \right)}^2 \frac{1}{1+i \beta_{e}}, \qquad \omega_{pe} = \sqrt{\frac{4 \pi e^2 n_e}{m_e}},
		\label{eps_plasma}
    \end{equation}

\noindent где $\omega_{h}$ --- рассмариваемая гармоническая частота, $\omega_{pe}$ --- электронная плазменная частота, $e$ и $m_e$ --- заряд и масса электрона, $n_e = Z n_i$ --- электронная плотность, где $Z$ --- средняя степень ионизации, $n_i$ --- ионная плотность. $\beta_{e} = \nu_e / \omega_{h}$ и $\nu_e$ --- частота электрон-ионных столкновений в приближении Спитцера. В условиях твердотельной плазмы ионная плотность кластера порядка $n_i = 6 \cdot 10^{22}$ $\textrm{cm}^{-3}$, при этом нам необходима электронная плотность кластера выше критической для заданной частоты $n_c = \omega_{h}^2 m_e / 4 \pi e^2$, так как в противном случае кластер будет прозрачен для излучения этой частоты. Для 10-ой гармоники лазерного излучения $\lambda_{h} = 83$ nm мы получаем условие $n_e > n_c = 1.3 \cdot 10^{23}$ $\textrm{cm}^{-3}$, что согласуется с условием на ионную плотность при средней степени ионизации $Z > 2$.

Мы используем Теория Ми для описания упругого рассеяния электромагнитных волн частицами произвольного размера в случае линейных взаимодействий, а также позволяет получить описание рассеянного поля и поля внутри рассеивающего объекта~\cite{andreev_lecz, boren_huffman}. 

%Основной шаг --- решение скалярного уравнения Гельмгольца в правильной системе координат (в данном случае сферической) и получение векторных решений. Для сферического кластера можем записать решение соответствующего уравнения, используя сферические функции Бесселя и Ханкеля $n$-ого порядка, включая присоединенные полиномы Лежандра~\cite{boren_huffman}.

Пользуясь разложением плоской волны по сферическим гармоникам~\cite{boren_huffman}, в случае $x$-поляризованной падающей волны, распространяющейся вдоль оси $z$, получаем:

    \begin{equation}
        \vectbf{E}{i} = E_0\:e^{i\omega t-ikz}\:\vectbf{e}{x} = \sum_{n = 1}^{\infty}E_n \left[ \vectbf{M}{}^{(1)}_{o1n} - i \vectbf{N}{}^{(1)}_{e1n} \right],
        \label{E_i_sph}
    \end{equation}

    \begin{equation}
		\vectbf{E}{s} = \sum_{n = 1}^{\infty}E_n \left[ i a_n\left(ka, m\right) \vectbf{N}{}^{(3)}_{e1n} - b_n\left(ka, m\right) \vectbf{M}{}^{(3)}_{o1n} \right], \qquad E_n = i^{n} E_0 \frac{2n + 1}{n \left(n + 1\right)}
        \label{E_s_sph}
    \end{equation}

Коэффициенты разложения Фурье по векторным гармоникам в случае изотропной среды являются коэффициентами рассеянного поля~\cite{boren_huffman}:

    \begin{equation}
		a_n(\chi,\:m) = \frac{m \func{\psi}{n}{\prime}{\chi} \func{\psi}{n}{}{m \chi} - \func{\psi}{n}{\prime}{m \chi} \func{\psi}{n}{}{\chi}}{m \func{\xi}{n}{\prime}{x} \func{\psi}{n}{}{m \chi} - \func{\psi}{n}{\prime}{m \chi} \func{\xi}{n}{}{\chi}},
		\label{an_bessel}
    \end{equation}

    \begin{equation}
        b_n(\chi,\:m) = \frac{\func{\psi}{n}{\prime}{\chi} \func{\psi}{n}{}{m \chi} - m \func{\psi}{n}{\prime}{m \chi} \func{\psi}{n}{}{\chi}}{\func{\xi}{n}{\prime}{\chi} \func{\psi}{n}{}{m \chi} - m \func{\psi}{n}{\prime}{m \chi} \func{\xi}{n}{}{\chi}},
        \label{bn_bessel}
    \end{equation}
    \begin{equation*} % artificial indent after the equation
    \end{equation*}

\noindentгде $\funccomp{\psi}{n}{}{\rho} = z \funccomp{j}{n}{}{\rho}$, $\funccomp{\xi}{n}{}{\rho} = z \funccomp{h}{n}{}{\rho}$ --- функции Риккати-Бесселя, $h_n = j_n + i \gamma_n$ --- сферические функции Ханкеля первого рода, $\chi = ka$ --- безразмерный радиус кластера, $ m = \sqrt{\varepsilon} $ --- комплексный коэффициент преломления (\ref{eps_plasma}).