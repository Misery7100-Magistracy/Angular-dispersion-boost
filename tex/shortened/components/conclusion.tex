\section{Заключение}

Мы обнаружили, что периодическая структура из плотных плазменных кластеров оказалась подходящим элементом для направленного эффективного рассеяния излучения в XUV диапазоне. Так как множество сферических рассеивателей требует вычислений в трёхмерном пространстве, была предложена стационарная модель и определен диапазон радиусов кластеров, в пределах которого электронная плотность квазистационарна во время взаимодействия с внешним полем. Когда ионизация такова, что концентрация электронов близка к резонансной для заданных начальных параметров, эффективность рассеяния значительно увеличивается и достигает нескольких процентов в случае одиночного кластера. Для множества кластеров эффективность угловой дисперсии растет с увеличением количества рядов кластеров и может достигать порядка нескольких десятков процентов в случае определенных направлений.

Полученные угловые распределения дифракционных максимумов для рассеяния при помощи множества регулярно расположенных кластеров хорошо описываются при помощи теории Лауэ, при этом внесение небольшой нерегулярности в распределение кластеров ослабляет наиболее интенсивные направления дифракции, отличные от направления прошедшего излучения, не более чем на 25\%. В случае немонохроматического излучения эффективность усиления угловой дисперсии падает в соответствии с спетральным распределением амплитуды поля, ослабление может достигать порядка трёх раз.