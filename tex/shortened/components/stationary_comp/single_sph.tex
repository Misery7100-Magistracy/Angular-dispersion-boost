\subsection{Одиночный кластер}

В рамках теории рассеяния Ми известно, что амплитуду поля вблизи поверхности мишени можно значительно усилить. Мы рассматриваем случай, когда частота электрон-ионных столкновений $\nu_e$ много меньше частоты гармоники, поэтому взаимодействие можно считать бесстолкновительным~\cite{andreev_lecz}. Для этого случая было посчитано рассеянное электрическое поле (\ref{E_s_sph}) при $\lambda = \lambda_{L}$ и $\lambda = \lambda_{h} = \lambda_{10}$ с целью сравнения между собой резонансного и нерезонансного случая, значение $n_{el}$ вычислено при помощи (\ref{m2_resonance}, \ref{nenc_resonance}) при $\lambda_{10} = 83$ nm, $\chi = 0.7$. Видно, что в резонансном случае (\autoref{ka0.7:b}) рассеянное поле представляет собой расходящуюся сферическую волну, амплитуда поля в окрестности кластера значительно выше, чем в отсутствии резонанса (\autoref{ka0.7:a}), где рассеяние рэлеевское.

    \begin{figure}[H]
        \subimgtwo[../img/mph_new/es_ka0.7_1harm]{$\lambda = \lambda_{L} = 830$ nm.}{ka0.7:a}{0.4\textwidth}
        \hfil
        \subimgtwo[../img/mph_new/es_ka0.7_10harm]{$\lambda = \lambda_{10} = 83$ nm.}{ka0.7:b}{0.4\textwidth}
        \caption{$a = 8.9$ nm; $|\vectbf{E}{s}|^2$ в плоскости поляризации падающей волны.}\label{ka0.7:image}
    \end{figure}

    \begin{figure}[H]
        \subimgtwo[../img/mph_new/es_ka1.7_10harm]{Рассеяние кластером.}{10h_ka0.7:a}{0.4\textwidth}
        \hfil
        \subimgtwo[../img/external/oe-28_screen_single.jpg]{Рассеяние наноцилиндром~\cite{andreev_lecz}.}{10h_ka0.7:b}{0.39\textwidth}
        \caption{$\chi = 1.7$ ($a = 22.5$ nm), $\lambda = \lambda_{10} = 83$ нм; $|\vectbf{E}{s}|^2$ построено в плоскости поляризации падающей волны. Качественное сравнение для такого же значения $ka$ в случае цилиндров (б) --- падающая волна распространяется справа налево (противоположно направлению оси $x$), $y$-поляризована.}\label{10h_ka0.7:image}
    \end{figure}

Дополительно был смоделирован случай $\chi = 1.7$ (\autoref{10h_ka0.7:a}) и сравнён с аналогичной ситуацией для одиночного наноцилиндра~\cite{andreev_lecz} (\autoref{10h_ka0.7:b}). Видно, что общие направления рассеянного поля сохраняются, видны слабые боковые порядки с углами отклонения, близкими к $90^\circ$ относительно направления падающей волны, что сходно с случаем цилиндрической симметрии. Различия в амплитуде рассеянных волн связаны с принципиальными отличиями в геометрии цилиндра и кластера. Наиболее интенсивное рассеяние наблюдается для направления, соответствующего направлению падающей волны в силу конструктивной интерференции, эффективность рассеяния в этом направлении около 5\%.