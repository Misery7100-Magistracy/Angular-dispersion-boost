\subsubsection{Рассеяние волнового пакета}

Рассмотрим рассеяние волнового пакета, образующегося в результате порождения из лазерного импульса первой гармоники. Амплитуда волнового пакета во времени описывается гауссовой функцией (\ref{gaussian_E0}), при этом исходный импульс первой гармоники достаточно короткий, чтобы пренебречь изменением его временной ширины (длительности) при преобразовании. В рамках периодического продолжения промежутка $[-\tau, \tau]$ этого импульса, где $\tau$ --- полуширина импульса, можем построить ряд Фурье (\ref{gaussian_E0_fourier}), откуда имеем коэффициенты Фурье (\ref{gaussian_E0_aj}), представляющие собой вклад каждой из гармоник в общий импульс.

    \begin{equation}
        A_0\left( t \right) = E_0\,\exp{\left( - \frac{t^2}{\tau^2}\right)}
        \label{gaussian_E0}
    \end{equation}

    \begin{equation}
        A_0\left( t \right) = \frac{\sqrt{\pi}}{2} + \sum_{j = 1}^{\infty}{ A_j \, \cos{\left(\omega_j t \right)}}, \quad \omega_j = \frac{2 \pi j}{\tau} = \frac{c}{\lambda_j}, \quad \lambda_{j} = \frac{\lambda_{L}}{j},
        \label{gaussian_E0_fourier}
    \end{equation}

    \begin{equation}
        A_j = \frac{1}{\tau} \int\limits_{-\tau}^{\tau} E_0\,\exp{\left( - \frac{t^2}{\tau^2}\right)} \cos{\left(\omega_j t \right)} dt.
        \label{gaussian_E0_aj}
    \end{equation}

Для того, чтобы построить диаграмму рассеяния волнового пакета, была использована новая интегральная характеристика, определенная с учетом коэффициентов разложения в ряд Фурье волнового пакета (\ref{wavepacket_eint}). Такая характеристика разумна для описания направлений рассеяния в силу аддитивности энергии как количественной характеристики. Область $V$ в данном случае представляет собой аналогичную той, что была использована для предыдущей интегральной характеристики (\ref{V_for_e_int}, \autoref{eint_scheme:image}).

    % \begin{equation}
    %     \vectbf{E}{i} = \sum_{n = 1}^{\infty} E_{0, \: n} \:e^{i n \omega_0 t-ikz}\:\vectbf{e}{x}\label{gaussian_E}
    % \end{equation}

    % \begin{equation}
    %     E_0 \left( \lambda \right) = \frac{1}{\sigma\sqrt{2\pi}}\,e^{-\frac{1}{2}{\left(\frac{\lambda - \mu}{\sigma}\right)}^2}\label{gaussian_E0}
    % \end{equation}

    \begin{equation}
        \EuScript{E}_{\textrm{int}} \left(V, \:\eta, \:\varphi_0, \:\theta_0 \right) = \sum\limits_{j\:=\:N_1\:>\:0}^{N_2}{E_{\textrm{int}} \left( \eta,\:\lambda_{j}, \:V, \:A_j, \:\varphi_0, \:\theta_0\right)}.
        \label{wavepacket_eint}
    \end{equation}

Определим наиболее интенсивные направления рассеянного поля для решётки с $d = 2\lambda_{10}$, радиусом кластеров $a = 20$~nm, $\theta_0 = 15^\circ$, $\varphi_0 = 0^\circ$, гармоники в волновом пакете с 8-ой по 12-ую, то есть $N_1 = 8$, $N_2 = 12$ в (\ref{wavepacket_eint}), гауссов импульс имеет полуширину $\tau \approx 17$ fs.

Сравнивая полученный результат с аналогичной диаграммой, вычисленной при помощи (\ref{e_int}) для 10-ой гармоники, можно заметить аналогичное значение для дифракционного максимума ($h^\prime = k^\prime = l^\prime = 0$), отвечающего за прошедшее излучение, и ослабление и расплывание остальных, что полностью соответствует (\ref{bragg_wolf_order_spherical}). В частности, относительная эффективность $E_{\textrm{frac}}$ первого дифракционного максимума ($\Delta \varphi = 0^\circ$, $\Delta \theta = 30^\circ$) ослабляется с 0.3 до 0.1.

Это происходит в силу того, что индексы Миллера, отвечающие дифракционным уравнениям для разных длин волн, будут связаны между собой коэффициентами пропорциональности, имеем масштабирование кривых, отвечающих целочисленным значениям индексов Миллера, что и приводит к размытию дифракционной картины (\Autoref{wavepacket1:a, wavepacket1:b}).

    \begin{figure}[ht]
        \subimgtwo[../img/celes/E_squared/eint_10harm_15deg_0.0nonreg.pdf]{Рассеяние 10-ой гармоники.}{wavepacket1:a}{0.45\textwidth}
        \hfil
        \subimgtwo[../img/celes/E_squared/eint_wavepacket2_15deg_0.0nonreg.pdf]{Рассеяние волнового пакета.}{wavepacket1:b}{0.45\textwidth}
        \caption{Угловая диаграмма рассеяния гауссового волнового пакета и 10-ой гармоники. $\theta_0 = 15^\circ$, $\varphi_0 = 0^\circ$, $d = 2\lambda_{10}$, радиус кластеров $a = 20$ nm.}\label{wavepacket1:image}
    \end{figure}