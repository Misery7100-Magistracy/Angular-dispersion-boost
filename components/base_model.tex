\section{Base model}

Let us consider a single cluster with radius $a$ irradiated by short femtosecond pulse with intensity about $I_{h} \approx 10^{14}$ $\textrm{W/cm}^2$. The Drude model yields the dielectric function of the plasma:

    \eq
		\varepsilon (\w) = 1 - \left( \frac{\w_{pe}}{\w} \right)^2 \frac{1}{1 + i \beta_{e}}, \qquad \w_{pe} = \sqrt{\frac{4 \pi e^2 n_e}{m_e}},
		\label{eps_plasma}
	\qe

\noindent where $\w$ --- harmonic (angular) frequency under consideration; $\w_{pe}$ --- the electron plasma frequency; $e$, $m_e$ --- electron charge and mass; $n_e = Z n_i$ --- the electron number density, where $Z$ --- average ionization degree, $n_i$ --- ion density. $\beta_{e} = v_e / \w$ and $v_e$ --- electron-ion collision rate in Spitzer approximation. As we are going to consider scattering of harmonic radiation, the cluster should have a density above the critical one for this harmonic: $n_c = \w^2 m_e / 4 \pi e^2$. Thus for example, for 10-th laser harmonic with wavelength $\lambda_{L} = 830$ nm one obtains condition $n_e > 1.3 \cdot 10^{23}$ $\textrm{cm}^{-3}$.

The Mie theory can be used for the description of elastic electromagnetic wave scattering by arbitrary sized particles in case of linear interactions and let obtain scattered and internal field. A main step is to solve the scalar Helmholtz Equation in suitable coordinate system andgain the vector solutions. For spherical cluster the solution of corresponding equation can be written in the form of Bessel and Hankel functions of $n$-th order~\cite{boren_huffman}.

Assume an incident plane wave propagating along $z$ axis of cartesian coordinate system and polarized along $x$ axis:

    \eq
        \vectbf{E}{i} = E_0\:e^{i\w t - ikz}\:\vectbf{e}{x},
        \label{E_i_sph}
    \qe

\noindent where $k = \w/c$ --- wavenumber, $\vectbf{e}{x}$ --- the unit vector of $x$ axis direction and polarization vector:

    \img[components/img/single_sph_scheme]{Base model scheme.}{single_sph_scheme:image}{0.73\textwidth}

Now we can expand the plane wave into series using generalized Fourier expantions. Assuming our media is isotropic we obtain following form of scattered field~\cite{boren_huffman}:

    \eq
		\vectbf{E}{s} = \sum_{n = 1}^{\infty}E_n \left[ i a_n\left(ka, m\right) \vectbf{N}{}^{(3)}_{e1n} - b_n\left(ka, m\right) \vectbf{M}{}^{(3)}_{o1n} \right], \qquad E_n = i^{n} E_0 \frac{2n + 1}{n \left(n + 1\right)}
        \label{E_s_sph}
	\qe

$n$ --- vector harmonic number after cartesian-spherical coordinate system transformation, $m = \sqrt{\varepsilon\left(\w\right)}$ --- refractive index of the target. Vector harmonics coefficients have the following form~\cite{boren_huffman}:


    \eq
		a_n(x,\:m) = \frac{m \func{\psi}{n}{\prime}{x} \func{\psi}{n}{}{mx} - \func{\psi}{n}{\prime}{mx} \func{\psi}{n}{}{x}}{m \func{\xi}{n}{\prime}{x} \func{\psi}{n}{}{mx} - \func{\psi}{n}{\prime}{mx} \func{\xi}{n}{}{x}},
		\label{an_bessel}
	\qe

    \eq
        b_n(x,\:m) = \frac{\func{\psi}{n}{\prime}{x} \func{\psi}{n}{}{mx} - m \func{\psi}{n}{\prime}{mx} \func{\psi}{n}{}{x}}{\func{\xi}{n}{\prime}{x} \func{\psi}{n}{}{mx} - m \func{\psi}{n}{\prime}{mx} \func{\xi}{n}{}{x}},
        \label{bn_bessel}
    \qe
    \eqc % artificial indent after the equation
    \cqe %

\noindent $\func{\psi}{n}{}{z} = z \func{j}{n}{}{z}$, $\func{\xi}{n}{}{z} = z \func{h}{n}{}{z}$ --- Riccati-Bessel functions, $h_n = j_n + i \gamma_n$ --- spherical Hankel functions of the first kind. 

In case of spherical symmetry amplitude of the scattered field is maximum for $m^2 = - (n+ 1) / n$ when $ka \ll 1$, that gain corresponding set of resonance densities in collision-less case: $n_e = n_c(2n + 1) / n$. It can be obtained using zero-order (asymptotic) approximation of Bessel functions, after which coefficients (\Autoref{an_bessel, bn_bessel}) are significant simplified:

    \eq
        a_n\left( x \to 0,\:m \right) = \left( 1 + 2i \frac{ (2n - 1)! (2n + 1)!}{4^n \: n! (n + 1)!} \frac{\left(m^2 + \frac{n + 1}{n} \right)}{(m^2 - 1)} \frac{1}{x^{2n+1}} \right)^{-1}, \qquad b_n\left( x \to 0,\:m \right) = 0
        \label{ab_asymp}
    \qe

This approximation can be used instead of (\Autoref{an_bessel, bn_bessel}) for scatterers with quite small radius, but for $ka \sim 1$ the approximation ceases to be reasonable already, particularly for large $n$. In this case the first-order approximation is better suited:

    \eq
		a_n\left( x ,\:m \right) = \left( 1 + i \frac{ C_n x^{-1 -2n} \left( (4(1 + n + m^2 n) (-3 + 4n (1 + n)) - 2(m^2 - 1)(3 + n(5 + 2n + m^2 (2n - 1))) x^2) \right)}{\pi (m^2 - 1)(2n + 3)(n + 1)(4(2n + 3) - 2(m^2 + 1)x^2)} \right)^{-1}
		\label{an_sph_asymp1}
	\qe
	\eqc
		C_n = 2^{1 + 2n} \Gamma(n - \frac{1}{2}) \Gamma(n + \frac{5}{2})
	\cqe

\autoref{ab_asymp:image} shows dependence of the scattering coefficient on the electron density for two different values of the radius in zero-order approximation. We can compare it with the first-order for $ka = 1.5$ (\autoref{ab1:image}). We can see, that with increase of $n$ width of the resonance peak decreases rapidly. Larger $ka$ corresponds to larger peak width. Besides that, with increase of $ka$ value of the resonance density increases, that shown in \autoref{nenc_123:image}.

    \begin{figure}[H]
        \subimgtwo[components/img/sph_base/sph_ka0.5_123]{$ka = 0.5$.}{ab_asymp:a}{0.66\textwidth}\\
        \subimgtwo[components/img/sph_base/sph_ka1.5_123]{$ka = 1.5$.}{ab_asymp:b}{0.66\textwidth}
		\caption{Spherical harmonics coefficients in zero-order approximation, $\beta_e = 0$. "Exact" curves were plotted using full expansions of the Bessel and Hankel fucntions.}
		\label{ab_asymp:image}
	\end{figure}

    \img[components/img/sph_base/sph_ka1.5_123_1st]{$ka = 1.5$ in first-order approximation. $\beta_e = 0$. "Exact" curves were plotted using full expansions of the Bessel and Hankel fucntions.}{ab1:image}{0.66\textwidth}

Such approximations allow us to estimate the resonance cases for a material with pre-defined refractive index $m$ as well as estimate refractive index corresponding to the required wavelength. As we consider XUV range radiation (20-120 nm), radiuses of spherical scatterers should be about few nanometers, that causes $ka \sim 1$. Obviously, for such $ ka $ the resonance values of the electron density can be large in considering $n = 1$ as term with the largest contribution to the scattered field. Staying within high-temperature plasma we should use only $n_e < 10^{24}$ $\textrm{cm}^{-3}$. Thus for $ka > 0.9$ it is more reasonable to estimate the resonance electon density using $n = 2$.

Using first-order approximation (\ref{an_sph_asymp1}) with wavelength $\lambda_{10} = \lambda_{L} / 10 = 83$ nm we get $n_e \approx 5 \cdot 10^{23}$ $\textrm{cm}^{-3}$ for $ka \approx 0.5$ and $n_e \approx 5.7 \cdot 10^{23}$ $\textrm{cm}^{-3}$ for $ka \approx 0.7$ to reach efficient scattering.

    \img[components/img/sph_base/nenc_123]{Resonance electron density depending on radius. Curves were calculated in maximum points of (\ref{an_sph_asymp1}), $\beta_e = 0$.}{nenc_123:image}{0.66\textwidth}



