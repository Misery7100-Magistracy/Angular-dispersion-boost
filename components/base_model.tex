\\\noindent(- - 8 - -)\\~\\
\section{Base model}

\noindent(- - 9 - -)\\~\\
Рассмотрим одиночный кластер радиусом $a$, облученный интенсивным фемтосекундным импульсом. Предположим, что падающая плоская волна распространяется вдоль оси $z$ декартовой системы координат и поляризована вдоль оси $x$. 

Модель Друде дает диэлектрическую функцию плазмы с соответствующими параметрами. Теория Ми может быть использована для описания рассеяния упругих электромагнитных волн в случае линейных взаимодействий и позволяет получить рассеянное и внутреннее поля. Основной шаг - решение скалярного уравнения Гельмгольца и получиние векторные решения. Для сферического кластера решение соответствующего уравнения можно записать в виде сферических функций Бесселя и Ганкеля $n$ -го порядка с соответствующими коэффициентами.

\noindent(- - 10 - -)\\~\\
Рассматривая отдельный кластер, мы можем использовать нулевое приближение функций Бесселя, если радиус кластера намного меньше длины волны. Для значений нормированного радиуса около $0.5$ такое приближение все еще применимо, но для $ka \ sim 1$ приближение уже перестает быть разумным, особенно для больших порядков векторных гармоник. 

\noindent(- - 11 - -)\\~\\
Вместо этого в этом случае лучше подходит приближение первого порядка (получается с учетом первого члена в разложениях функций Бесселя). Здесь мы можем увидеть, насколько лучше это разложение --- кривые первого приближения намного ближе к кривым точных решений, в отличие от нулевого. 

\noindent(- - 12 - -)\\~\\
Такие аппроксимации позволяют оценить случаи резонанса для материала с заранее заданным показателем преломления $m$, а также оценить показатель преломления, соответствующий требуемой длине волны.


