\section{Multiple clusters}

\noindent(- - 19 - -)\\~\\

Также у нас есть результаты для рассеяния на простой кубической решетке в рамках случя нескольких сферических кластеров.

\noindent(- - 20 - -)\\~\\
Пространственная решетка описывается следующими параметрами: радиус узла $a$, количество узлов на ребре и длина ребра элементарной ячейки $b$. Параметры падающего поля такие же, как для одиночного кластера.

\noindent(- - 21 - -)\\~\\
Были рассмотрены два случая: с $b$, равным одной длине волны, и с $b$, равным утроенной длине волны.

В случае одной длины волны наблюдается эффективное рассеяние на гранях пространственной решетки. Большая часть поля локализована в области мишени.

\noindent(- - 22- -)\\~\\
Для утроенной длины волны повышенное разрежение между кластерами позволяет избавиться от сильного рассеяния назад, которое видно из дальнего поля. 

Таким образом, было показано, что с помощью линейной теории Ми можно количественно оценить параметры резонанса отдельного кластера и предсказать потенциальные направления рассеяния для нескольких кластеров.

Результаты показывают возможность управления высокими гармониками лазерного излучения в XUV-диапазоне с помощью ионизированного кластерного газа.

\noindent(- - 24 - -)\\~\\

Спасибо за внимание.