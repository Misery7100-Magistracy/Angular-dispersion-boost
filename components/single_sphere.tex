\section{Single cluster}

\noindent(- - 13 - -)\\~\\

Далее мы рассмотрим несколько вычислительных экспериментов.

\noindent(- - 14 - -)\\~\\
В рамках теории Ми хорошо известно, что мы можем значительно увеличить амплитуду поля вблизи мишени. Для проверки рассмотрим первую и десятую гармоники лазера в двух случаях радиуса кластера: 0.5 и 0.7.

\noindent(- - 15 - -)\\~\\
Было рассчитано полное ближнее и дальнее поле для сравнения их амплитуд и профилей рассеяния. Мы видим, что рассеяние лазерной гармоники (первой гармоники) очень близко к рэлеевскому рассеянию - профиль падающей плоской волны практически не меняется. Максимальное значение амплитуды ближнего поля составляет около четырнадцати.

\noindent(- - 16 - -)\\~\\
Совершенно другая ситуация для 10-й гармоники - профиль падающей плоской волны искажается в результате рассеяния и становится похожим на расходящуюся сферическую волну. Амплитуда ближнего поля примерно в 5 раз выше, чем у первой гармоники. 

\noindent(- - 17 - -)\\~\\
Здесь ситуация аналогична рассмотренному ранее случаю с первой гармоникой - почти рэлеевское рассеяние без искажения профиля, лишь меньший максимум амплитуды поля.

\noindent(- - 18 - -)\\~\\
Данный случай дополнительно был сравнён с аналогичной ситуацией для рассеяния на одиночном наноцилиндре. Мы можем видеть, что распределения поля похожи, включая сферическую расходящуюся волну в дальней зоне и локализованную область ближнего поля в направлении рассеяния на $0^{\circ}$ относительно направления распространения падающей волны.