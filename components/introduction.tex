\noindent(- - 2 - -)\\~\\

\section{Introduction}

\noindent(- - 3 - -)\\~\\
Мишени ограниченного размера, взаимодействующие с мощным когерентным излучением, представляют собой хорошо изученное явление линейных возбужденных поверхностных плазмонных колебаний. Поглощение и рассеяние падающего света в таком случае хорошо описывается теорией Ми, предсказывающей существование резонанса, соответствующего мультипольным колебаниям части свободных электронов мишени относительно положительно заряженных ионов. В резонансном режиме эффективное возбуждение поверхностных плазмонов может привести к значительному увеличению внутреннего и внешнего поля на основной частоте кластера (собственной частоте). В свою очередь, это может вызвать усиление поля, рассеянного на больш\'{и}е углы относительно направления падающей волны.

\noindent(- - 4 - -)\\~\\
В микрометровых длинах волн фотонные кристаллы и решетки могут использоваться для направления или дифракции электромагнитных волн, в то время как для рентгеновского излучения можно использовать реальные кристаллы с регулярно расположенными центрами рассеяния (атомами) на расстоянии нескольких нанометров. В то же время большим интервалом между этими порядками длин волн, называемым XUV (жесткий или экстремальный ультрафиолет), достаточно трудно управлять.

\noindent(- - 5 - -)\\~\\
В рамках данной работы мы рассматриваем возможность направленного рассеяния коротковолнового излучения в XUV-диапазоне за счет рассеяния на сферических кластерах. Подобный случай с цилиндрической симметрией (массивы наноцилиндров в качестве рассеивателей) исследовался ранее. Конечно, наноцилиндры лучше подходят для управления параметрами размера и расстояния на стадии изготовления мишени, но массивы сферических кластеров могут позволить управлять направлением света в трехмерном пространстве и обеспечить более оптимальную пространственную конфигурацию за счет своей геометрии.

\noindent(- - 6 - -)\\~\\
Известно, что короткий интенсивный лазерный импульс может генерировать гармоники высокого порядка, взаимодействуя с плотными твердыми поверхностями. Но интенсивность высоких гармоник, генерируемых в газах, как минимум на 4 порядка меньше, что недостаточно для ионизации мишени и генерации плазмы с полностью мнимым показателем преломления, который нам нужен - в нашем случае сферические кластеры - это ионизированный кластерный газ (рисунок 1). Для решения этой проблемы мы предлагаем использовать интенсивный предимпульс для предварительной ионизации мишени и достижения необходимых условий генерации.

\noindent(- - 7 - -)\\~\\
Общая схема взаимодействия представлена на рисунке 2. Гармоники в основном импульсе имеют разную интенсивность в зависимости от угла, что приводит к угловой зависимости формы выходного излучения. Рассеяние на отдельном кластере можно полностью описать в сферической симметрии, а взаимодействие можно легко смоделировать с помощью particle-in-cell метода. Мы предлагаем использовать линейное приближение в рамках теории Ми в качестве оценки для дальнейшего моделирования. В частности, мы концентрируемся на теоретическом исследовании, подкрепленном компьютерным моделированием, и указываем на применимость для экспериментальной реализации.
